\def\Arccosh{{\rm Arccosh}}
\advance\sectioncount by -1
\section{Introduction}

This paper documents a rigorous computer-assisted procedure for analyzing 
hyperbolic $3$-manifolds.

The first two applications of this procedure was published as part of [GMT],
as Propositions 1.28 and 2.8. The proof of Proposition 1.28 is reproduced here
with only minor changes, including most of Section 1 and the entirety of Sections 2 through 5.
This document introduces techniques useful for "finding killerwords" in Section 6,
and in so doing borrows from material that was originally published as remarks
in [GMT].

k
that every hyperbolic $3$-manifold has a non-coalescible insulator family,
and therefore that homotopy hyperbolic $3$-manifolds are hyperbolic. Formally

Informally, the result was first published as Conjecture 1.31 of [GMT],
and recently become theorem:

\proclaim{Theorem} Let $M$ be an orientable hyperbolic $3$-manifold, and let $\delta$ be
a shortest geodesic. Then, either ${\it tuberadius}(\delta) > \ln(3)/2$, or
$M$ is isometric to one of $N_0, \cdots, N_6$, where:
\begin{itemize}
\ritem{i)} $N_0$ is Vol3.
\ritem{ii)} $N_1$ is FIXME.
\ritem{iii)} $N_2$ is FIXME.
\ritem{iv)} $N_3$ is FIXME.
\ritem{v)} $N_4$ is FIXME.
\ritem{vi)} $N_5$ is FIXME.
\ritem{vii)} $N_6$ is FIXME.
\end{itemize}
\endproclaim
\demo{Proof}
Let $X_0, \cdots, X_6$ denote the exceptional regions of Theorem 2.XX,
with $N_i$ the corresponding conjectural manifold.
[GMT] showed that $N_0$=Vol3 is the unique manifold in the region
and [JR] showed that Vol3 covered no 3-manifold.
[JR] also proved that $N_5$ and $N_6$ are isometric.
[CLLMR] and [L] identify a manifold in each region
and [CLLMR] show that these manifolds are the unique ones in its region.
[CLLMR] show that $N_1$, $N_5$ and $N_6$ cover no manifold.
In [GT] the proof is completed by showing that $N_3$ covers no manifold
and each of $N_2$ and $N_4$ 2-fold cover manifolds,
however the quotients are all non exceptional,
i.e. each shortest geodesic has a $log(3)/2$ tube.
\enddemo

Other applications of the procedure:

Theorem [G]  (Smale conjecture of hyperbolic 3-manifolds)
If N is a closed hyperbolic 3-manifold,
then the natural inclusion $Isom(N)\to Diff(N)$ is a homotopy equivalence.

The proof made essential use of the fact that a shortest geodesic of a closed hyperbolic 3-manifold satisfies the insulator condition [GMT]

Theorem [GMM], [Mi] The Week's manifold is the
unique closed orientable hyperbolic 3-manifold of minimal volume.

This result relies on 

Theorem  [ACS] Suppose that M is a closed, orientable hyperbolic 3Ðmanifold,
and that C is a shortest geodesic in M.
Set N D drillC .M/. If $tuberad(C) \ge .log 3/2$ then
volN < 3:0177 volM.

This is based on a result of [ADST], that makes essential use of Perelman's [Pe] monotonicty result, a key element of his proof of geometrization and the log(3)/2 theorem of [GMT].  Theorem C is used to prove other results such as 

Theorem [ACS] Suppose that M is a closed, orientable, hyperbolic
3Ðmanifold such that $vol(M)\le 1.22$.
Then $H_1(M:Zp)$ has dimension at most 3 for every prime p.

(The statement is sharpened slightly to take into account [GT])

Theorem [LM] Let M be a compact orientable 3-manifold with boundary a
torus, and with interior admitting a complete finite-volume hyperbolic structure.
Then the number of exceptional slopes on M is at most 10.

This result, known as the Gordon conjecture, relies on rigorous computer assistance using the AffApprox technology.

Finally, the procedure is currently in use by Gabai, Thurston, and Yarmola
to rigorously analyze small orientable cusped hyperbolic $3$-manifolds,
and to rigorously analyze small non-orientable closed hyperbolic $3$-manifolds.

* ORGANIZATION
This paper is organized as follows.  

In Section 1 we introduce [GMT]'s Proposition 1.28 as Theorem 2.XX, and sketch its proof.

In Section 2 we provide a formal statement and proof of Theorem 2.XX.

INCREMENTSECTIONNUMBER(
In Sections 2 through 5 we address the computer-related aspects of the 
proof.   In Section 2, the method for describing the decomposition of the 
parameter space ${\cal W}$ into sub-regions is given, and the conditions 
used to eliminate all but seven of the sub-regions are discussed.  Near the 
end of this chapter, the first part of a detailed example is given.  Eliminating 
a sub-region requires that a certain function is shown to be bounded 
appropriately over the entire sub-region.  This is carried out by using a 
first-order Taylor approximation of the function together with a 
remainder 
bound.  Our computer version of such a Taylor approximation with remainder bound 
is called an {\it AffApprox} and in Section 3, the relevant theory is 
developed.  At this point, the detailed example of Section 2 can be 
completed.

In Sections 4 and 5, round-off error analysis appropriate to our 
set-up is introduced.  Specifically, in Section 5, round-off error is 
incorporated 
into the {\it AffApprox} formulas introduced in Section 3.  The proofs here 
require an analysis of round-off error for complex numbers, which is carried 
out in Section 4.
)

In Section 7 we WORDSMITH(breadcrumbs about finding killerwords).

Finally, in Section 8, we present an updated version of
the code used to check that the proof is valid,
with self-contained copies of the proofs required for the reader check its own validity.

{\it Acknowledgements}.
FIXME
