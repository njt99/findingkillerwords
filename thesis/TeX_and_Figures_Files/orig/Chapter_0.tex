
\advance\sectioncount by -1
\section{Introduction}
 
This paper introduces a rigorous computer-assisted procedure for analyzing 
hyperbolic
$3$-manifolds.  This procedure is used to complete the proof of several 
long-standing
rigidity conjectures in $3$-manifold theory as well as to provide a new lower
bound for
the volume of a closed orientable hyperbolic $3$-manifold. 

\proclaim{Theorem} Let $N$ be a closed hyperbolic $3$\/{\rm -}\/manifold. Then
 \begin{itemize}
\ritem{i)} If $f\colon M \to N$ is a homotopy equivalence{\rm ,} where $M$ is a closed
irreducible $3$\/{\rm -}\/manifold{\rm ,} then $f$ is homotopic to a homeomorphism.

\ritem{ii)} If $f,g\colon M\to N$ are homotopic homeomorphisms{\rm ,} then $f$ is
isotopic to $g$.

\ritem{iii)}  The space of hyperbolic metrics on $N$ is path connected.
\end{itemize}
\endproclaim



\demo{{R}emarks}  Under the additional hypothesis that $M$ is 
hyperbolic,  conclusion i) follows from
Mostow's rigidity theorem [Mo].  
Under the hypothesis that $N$ is Haken (and not necessarily
hyperbolic), 
conclusions i), ii) follow from Waldhausen [Wa].  
Under the hypothesis that $N$ is both Haken and
hyperbolic, conclusion iii) follows by combination of  [Mo] and [Wa].  
Because non-Haken
manifolds are necessarily orientable we will from now on assume that all
manifolds under discussion are orientable.
\enddemo

Theorem 0.1 with the added hypothesis that some closed geodesic
$\delta \subset N$ has a
{\it noncoalescable insulator family} was proven by  Gabai (see [G]).  Thus
Theorem 0.1 follows from [G] and the main technical result of this paper
which is:
 
\proclaim{Theorem}   If $\delta$ is a shortest geodesic in a closed 
orientable hyperbolic $3$\/{\rm -}\/manifold{\rm ,} then $\delta$ has a 
non{\rm -}coalescable insulator family.
\endproclaim

{\it Remarks}.   If $\delta$ is the core of an embedded hyperbolic 
tube of radius\break
$\ln(3)/2 =  0.549306\ldots$ then $\delta$ has a noncoalescable insulator 
family by
Lemma 5.9 of [G].  (See the Appendix to this paper for a review of insulator 
theory.)  In this 
paper we establish a
second condition, sufficient to guarantee the existence of a noncoalescable  insulator family
for $\delta:$ That {maxcorona}($\delta) < 2\pi/3$. 
(maxcorona($\delta)<2\pi/3$ if tuberadius($\delta)>\ln(3)/2$.)
We use the expression ``$N$ satisfies the {\it insulator condition}"
when there is a geodesic $\delta$ which has a noncoalescable  insulator 
family.

 
We prove Theorem 0.2 by first showing that
all closed hyperbolic $3$-manifolds, 
with seven families of exceptional cases, 
have embedded hyperbolic
tubes of radius $\ln(3)/2$ about their shortest geodesics. 
(Conjecturally, up to isometry, there are exactly six exceptional
manifolds associated to these seven families; see Conjecture 1.31 and 
Remarks 1.32.)
Second, we show that any shortest
geodesic $\delta$ in six of the seven families has
maxcorona($\delta)<2\pi/3$.
Finally, we show that the seventh family corresponds to Vol3, the closed 
hyperbolic $3$-manifold
with (conjecturally) the third smallest volume, and that the insulator 
condition holds for Vol3.  Each of the three parts of the proof is carried out 
with the assistance of a rigorous computer program.

Here is a brief description of why Theorem 0.2 might be amenable to 
computer-assisted proof.
If a shortest geodesic $\delta$ in a hyperbolic $3$-manifold $N$ does not have 
a $\ln(3)/2$
tube then there is a 2-generator subgroup $G$ of $\pi_1(N)\break =\Gamma$ which 
also does not have that property.  
Specifically, take 
$G$ generated by $f$ and $w$, with $f\in \Gamma$ a primitive
hyperbolic isometry
whose fixed axis $\delta_0 \subset {\bf H}^3$ projects to $\delta$, and 
with $w\in \Gamma$ a hyperbolic isometry
which takes $\delta_0$ to a nearest translate.
Then, after identifying $N={\bf H}^3/\Gamma$ and letting 
$Z={\bf H}^3/G$,
we see that the shortest geodesic in $Z$ (which corresponds to $\delta$)
does not have 
a $\ln(3)/2$ tube. 
Thus, to understand solid tubes around shortest geodesics in hyperbolic 
$3$-manifolds, we need to understand appropriate 2-generator groups, and this 
can be done by a parameter space analysis as follows.  (Parameter space 
analyses are naturally amenable to 
computer proofs.)

The space of {\it relevant} (see Definition 1.12) 2-generator groups 
in\break ${\rm Isom}_+({\bf H}^3)$ is naturally
parametrized by a subset ${\cal P}$ of ${\bf C}^3.$  
Each parameter corresponds to a
2-generator group $G$ with specified generators $f$ and $w$, and we call 
such a group a {\it marked group}. 
The marked groups of particular interest are those in which $G$ is
discrete, torsion-free, parabolic-free, $f$ corresponds to a shortest
geodesic $\delta$, and $w$ corresponds to a
covering translation of a particular lift of
$\delta$ to a nearest translate.  
We denote this set of particularly interesting marked groups by ${\cal T}.$
We show that if tuberadius($\delta) \le \ln(3)/2$ 
in a hyperbolic $3$-manifold $N,$ then 
$G$ must correspond to a parameter lying in one of seven small regions
${\cal R}_n,\ n=0,\ldots,6$ 
in ${\cal P}$.  
With respect to this notation, we have:

\vglue8pt {\elevensc Proposition 1.28}.
${\cal T} \cap ({\cal P} - \mathbold{\cup}_{n=0,\ldots,6}{\cal R}_n) = \emptyset.$
\vfill

The full statement of Proposition 1.28 explicitly describes the seven small
regions of the parameter space as well as some associated data.

Here is the idea of the proof.
Roughly speaking, we subdivide ${\cal P}$ into a billion small regions and 
show that all but the seven exceptional regions cannot contain 
a parameter corresponding to a  
``shortest/nearest" marked group.
For example we would know that 
a region ${\cal R}$ contained no such group if we knew that for each 
point $\rho\in {\cal R}$,
Relength($f_\rho) > {\rm Relength}(w_\rho).$  
(Here Relength($f_\rho$) (resp.\ Relength($w_\rho^{\phantom{|}}$))
denotes the real translation length of the isometry of ${\bf H}^3$ 
corresponding to
the element $f$ (resp.\ $w$) in the marked group with parameter~$\rho.$)
This inequality would contradict the fact that $f$ corresponds to $\delta$ 
which is a {\it shortest} geodesic.  Similarly, there are {\it nearest}
contradictions.

Having eliminated the entire relevant parameter space 
with the 
exception of seven small regions, we next ``perturb" the computer analysis to 
eliminate six of the seven regions, that is, all but the region ${\cal R}_0.$
We do this by proving:

\specialnumber{2.8} \proclaim{Proposition}  \hglue-9pt
 If $\delta$ is a shortest geodesic in closed hyperbolic $3$\/{\rm -}\/manifold $N,$ and
maxcorona($\delta) \ge 2\pi/3,$ then $\pi_1(N)$ has a marked subgroup 
whose associated parameter lies in
${\cal R}_0.$
\endproclaim

The proof
involves a second computer analysis similar to the first.  Here we 
are interested in discrete, torsion-free, parabolic-free, 
marked groups $\{G,f,w\}$ where $f$ corresponds to an oriented
shortest geodesic $\delta,$ and $w$ corresponds  to\break an isometry which 
maximizes
the function ${\cal C}(d(\delta_0,h(\delta_0)))$ where 
$h\in\break  G-\{f^k\},$ and finally
${\cal C} (d(\delta_0,w(\delta_0)))\ge 2\pi/3.$  Here $\delta_0$ is a lift of 
$\delta,$
$\ d(\delta_0,h(\delta_0))$ is the complex distance between the two 
oriented
geodesics $\delta_0, h(\delta_0),$ and ${\cal C}$ is a function called the 
corona function.
If $\{G,f,w\}$ is as above (but need not satisfy the condition 
${\cal C} (d(\delta_0,w(\delta_0)))\ge 2\pi/3$), then we define
${\rm maxcorona}(\delta) = {\cal C} (d(\delta_0,w(\delta_0))).$
Thanks to the first computer analysis, 
we need only analyze a vastly smaller parameter
space than ${\cal P}$.

Finally, a detailed analysis of the region ${\cal R}_0$ 
enables us to prove:

\specialnumber{3.1} 
\proclaim{Proposition}
${\cal T} \cap {\cal R}_0$ contains a 
unique parameter and the quotient of ${\bf H}^3$ by the group associated to this parameter is 
{\rm Vol3}.   Further if $N$ is a closed hyperbolic $3$\/{\rm -}\/manifold with 
shortest geodesic $\delta$ and 
{\rm maxcorona(}$\delta) \ge 2\pi/3,$ then $N = {\rm Vol3}.$  
\endproclaim

Then, through a direct analysis of the
geometry of Vol3 we prove:

\specialnumber{3.2} \proclaim{Proposition} {\rm Vol3} satisfies the insulator condition.
\endproclaim

This completes the proof of Theorem 0.2.
 
This paper is organized as follows.  
In Sections 1,2,3 we prove Propositions 1.28, 2.8, 3.1/3.2, 
respectively.  In addition,
in Section 1 we describe the  space 
${\cal P}^\prime \subset {\bf C}^3$ which
naturally parametrizes all relevant marked groups.  
We explain how a theorem of Meyerhoff as
well as elementary hyperbolic geometry considerations imply that we need
only consider a
compact portion ${\cal P}$ of ${\bf C}^3$. We will actually
be working in the parameter space ${\cal W} \supset \exp({\cal P})$.  
The technical reasons for
working in ${\cal W}$ rather than in ${\cal P}$ are
described near the end of Section 1.  In Section 2 we
describe and prove the necessary results about the 
corona function ${\cal C}$.
In Section 4, we prove some applications, one of which is discussed briefly 
below.

In Sections 5 through 8 we address the computer-related aspects of the 
proof.   In Section 5, the method for describing the decomposition of the 
parameter space ${\cal W}$ into sub-regions is given, and the conditions 
used to eliminate all but seven of the sub-regions are discussed.  Near the 
end of this chapter, the first part of a detailed example is given.  Eliminating 
a sub-region requires that a certain function is shown to be bounded 
appropriately over the entire sub-region.  This is carried out by using a 
first-order Taylor approximation of the function together with a 
remainder 
bound.  Our computer version of such a Taylor approximation with remainder bound 
is called an {\it AffApprox} and in Section 6, the relevant theory is 
developed.  At this point, the detailed example of Section 5 can be 
completed.

As an aside,  we note that at  the time of this research we believed (based 
largely on discussions with experts in the field) that there were no available 
appropriate Taylor approximation packages.  
Since carrying out our research, we 
have discovered that L. Figueiredo and J. Stolfi have independently 
developed an Affine Arithmetic package which is different in spirit and in 
the specifics of the implementation from ours, but covers similar 
ground.  One can consult [FS] for an alternate approach to ours.  

Finally, in Sections 7 and 8, round-off error analysis appropriate to our 
set-up is introduced.  Specifically, in Section 8, round-off error is 
incorporated 
into the {\it AffApprox} formulas introduced in Section 6.  The proofs here 
require an analysis of round-off error for complex numbers, which is carried 
out in Section 7.

We used two rigorous computer programs in our proofs---{\it verify} and 
{\it corona}.  These programs are provided at the {\it Annals} web 
site.  It should be noted that {\it corona} is a small variation of {\it verify} 
and as such only a small number of sections differ from those of {\it verify}.  
The proofs of Propositions 1.28, 2.8, and 3.2 amount to having {\it verify} 
and {\it corona} analyze several computer files.  These computer files are 
also available at the {\it Annals} web site. Details about how to get 
them and the programs can be found there.
 
One consequence of our work is: 

\specialnumber{4.1}
\proclaim{Theorem} If $\delta$ is a shortest geodesic in
the closed orientable hyperbolic $3$\/{\rm -}\/manifold $N,$ then either
 \vglue2pt
{\rm i)} {\rm tuberadius(}$\delta) > \ln(3)/2${\rm ,} or
 \vglue2pt
{\rm ii)} $1.0953/2 > {\rm tuberadius}(\delta) > 1.0591/2$
and {\rm Relength(}$\delta) > 1.059${\rm ,} or
\vglue2pt
  {\rm iii)} {\rm tuberadius(}$\delta) = 0.8314\ldots/2$ and $N = {\rm Vol3}.$\pagebreak
\endproclaim
 

Combining Theorem 4.1 with a result of F. Gehring and G. Martin (see [GM2]), which implies that if the closed 
orientable hyperbolic $3$-manifold $N$ has a geodesic with a $\ln(3)/2$ tube
 then the volume of $N$ is greater than $0.16668\ldots,$ we obtain:

\vglue3pt {\elevensc Corollary 4.3.}  {\it If $N$ is a closed orientable hyperbolic 
$3$\/{\rm -}\/manifold{\rm ,} then
the volume of $N$ is greater than  $0.16668\ldots\ $.}
\vglue3pt
\advance\theoremcount by 1
 
{\it Remarks}.   i) The previous best lower bound for volume was 
0.001 by [GM1], which
improved the lower bound   0.0008 of [M2].
\vglue2pt
ii) Using Theorem 4.1 and expanding on work of Gehring and Martin, A. Przeworski has recently extended the lower bound 
to 0.276796 (see [P]).\footnote{{\it Note added in proof} (January 2003): Przeworski (see [P2]) has improved the volume
lower bound to 0.3315 by combining his tube packing results with Theorem 4.1 and work of I. Agol (see [A]).} 
\vglue6pt

Given the fundamental use of computers in our proof, we need to 
discuss issues related to their use.  For an introduction to this topic we 
suggest [La], especially the concluding remarks.

We pose the simple question: why should one have confidence in our proof?  
%\vglue3pt 

First, the non-computer part of the proof has been analyzed in the 
traditional way by the authors, referees, individuals, and in seminars.
% \vglue3pt 

Second, the computer programs we have written can be checked just as  
mathematical proofs can be checked, and have been so checked. However, 
one must be prepared for subtleties that the computational approach 
introduces.  For example, if we wish to have the computer show that 
the result $x$ of a calculation is less than 2, 
it is not equivalent to show that $x$ is not greater than or equal to 2. 
That is because the
output of the computation may be ``NaN" (not a number) and said output is
not ``greater than or equal to 2".  This may arise, for example, if at some 
point of the (theoretical)
computation one takes the quotient of two numbers, both of which are 
extremely small.  The computer will view both the numerator and the 
denominator as 0 and hence produce the NaN output.   See Sections 5, 6,  
and 7 of [IEEE]  for more  details.  

We note that our programs are not complicated. In fact, the {\tt main} part 
of both programs is extremely simple conceptually.  The bulk of the 
programming is taken up with constructing 
first-order Taylor approximations 
with round-off error built in.  This is interesting, but not deep.   Although 
the proofs can get complicated, an undergraduate could easily check this 
material. 
 
% \vglue3pt 
Third,  we asked the computers to do simple things.
We were able to organize our proof so that the only mathematical operations 
used are $+,-,\times,/,\sqrt{}$ and these operations are governed by the 
IEEE-754 standards (see [IEEE]).  Thus, if the computer verifying  our proof 
adheres to 
the IEEE-754 standards for these operations then we have a valid proof. 
(Here, ``adhering to the IEEE-754 standards" requires that the computer run 
properly, that the version of C++ used adheres to the ANSI-C standards, and 
so on.)  We note that, according to the IEEE-754 standards, one typically has 
to tell the computer to check for occurrences of underflow and overflow.
%\vglue3pt

Fourth, we successfully ran our verification programs on several machines 
with different compilers and different architectures.  
Of course, despite the manufacturers's claims of IEEE-754 compliance,  bugs 
can exist.  However, we did run our verifications on machines believed to be 
reliable.  Further, having run the verifications on quite different machines, 
we have a significant increase in confidence
(see [K2] for an entertaining example). 

Because of the size of our data set (between one-half and one gigabyte in 
compressed form), the verification program takes quite a few CPU hours to 
run and we ran it first (successfully)
in about 60 CPU days.   Here, the term ``CPU day" refers to 24 hours of 
running an SGI Indigo 2 workstation with the R4400 chip, and the estimate 
of 60 days refers to 20 to 30 computers running 80 to 90 percent of the time 
over the course of 3 or 4 days.  We also had access to the suite of eight SUN 
computers in the Physics Department at Boston College and successfully ran 
the verification program on the data set.

We note that
one can gain further confidence by testing the machines/ compilers using 
available ``vetting" programs. Such programs are not infallible, but they do 
test in obvious trouble spots.  One active area of research in computer 
science involves effective checking of machines/compilers.
However, we note that this is a developing field, and at this point not an 
entirely satisfying tool.  The vetting program {\it ucbtest} was successfully 
run on the SGI's. 
%\vglue3pt

Fifth,  we employed various common-sense checks on our work.  For 
example, a program employing  a completely different, very geometric, 
approach to the theory was used on a rich set of data points and gave results 
consistent with our rigorous program.  Further, difficult cases were checked 
by another program differing from the rigorous program (although 
employing the same Taylor approximation method) and run on a 
completely different platform (Macintosh). 
\pagegoal=50pc

We also note that one set of referees did extensive 
robustness checks on the data.  
In the numerous regions they analyzed, they found that {\it verify} 
could be run successfully despite significant reduction in precision
(reducing, for instance, from 52 bits to 30 bits of precision)
without having to change the depth of subdivision.
This robustness is not surprising to us: 
we did a heuristic analysis of operations 
performed which indicated that round-off error would not affect 
computations  significantly until about 26 subdivisions in each of six
dimensions were performed (the box ${\cal W}$ was 
recursively subdivided in half by hyperplanes).  
We note that in general, nowhere near this number of 
subdivisions was needed, and in fact, it turned out to be  the maximum 
number of subdivisions ever used.
%\vglue3pt

Sixth, there was a significant amount of internal consistency in our work.  
For example, we found by our computer analysis that a certain hyperbolic\break 
$3$-manifold has  a particularly small maximal tube around its 
shortest geodesic.  This turned out to be the well-known hyperbolic 
$3$-manifold Vol3.  Further, 
our computer analysis discovered six (more) exceptional regions in the 
parameter space ${\cal W}$ which led us to (probable) hyperbolic 
$3$-manifolds with interesting properties.  J. Weeks's program {\it SnapPea} 
(see [W1]) later provided confirming evidence  for much of this.
\pagegoal=48pc
  
These are some of our reasons for having confidence in our proof.  Now we 
move on to the issue of archiving of research.

Our proof utilizes between one-half and one gigabyte (compressed) of data.  
The {\it Annals} has set up a secure system to store the data and make it 
available for future researchers.
The verification computer programs also reside in the {\it Annals} web site, if 
only for easy retrieval.  We note that copies of the programs are {\it not} in 
the paper proper.

Certain details of the proof are tedious, probably of limited use,  and would 
be better off archived than put in the paper proper.  Specifically, proofs of 
the propositions in Sections 7 and 8 are important but tedious in their 
similarity.  We provide representative examples in the paper, but relegate 
the full collection to the archive.  Further,  we found that skipping steps in 
the proofs of these propositions was a sure avenue to disaster.  As such, we 
proved each proposition in stupefying detail.  In the paper, this detail is 
sometimes pruned, but the full versions exist in the archive.

One could also ask what would happen if the large data set used to prove the 
theorem was lost.  
If one had access to the roughly 13000 member ``conditionlist" then it would 
be relatively easy to reconstruct the data set (that is, the decomposition of 
the parameter space into sub-boxes together with killerwords/conditions), 
because the hardest part of constructing the data set was the search for 
killerwords.  Presumably, this search could also be done fairly quickly by 
breaking the parameter space up into small pieces and farming these pieces 
out to various computers.  Also, the locations in the parameter space that 
caused the most trouble are explicitly described in Proposition 1.28.  
Knowing these trouble spots and how to deal with them ahead of time, would 
be a time-saver.

In the unlikely event that  ``conditionlist" was not available, the task of 
reconstructing the data set would be considerably harder.  Although, the 
facts that it {\it has} been done, that faster and faster computers will be 
plentifully available, and that Proposition 1.28 saves some work, indicate 
that the reconstruction process would not be too horrendous.  Further, it is 
also possible that improved proof techniques to the main theorem of this 
paper will be developed.  In fact, Section 2 describes a process, {\it corona}, 
that has the potential to handle easily  the worst of the trouble spots in the 
parameter space.
 
 \pagebreak
{\it Acknowledgements}.  We thank The Geometry Center and
especially Al Marden and David Epstein for the vital and multifaceted roles
they played in  this work.   We also thank the Boston College Physics 
Department for allowing us to use their suite of computers.

Jeff Weeks and {\it SnapPea} provided  valuable data and
ideas.  In
fact, the data from an undistributed version of
{\it SnapPea} encouraged us to pursue a computer-assisted proof of Theorem 
0.2.  

Bob Riley specially tailored his program
{\it Poincar{\rm \'{\it e}}}
to directly address the needs of our project.  His work provided
many leads in our search for
killerwords.  Further, he provided the first proof to show (experimentally) 
that the six
exceptional regions (other than the Vol3 region) correspond to
closed orientable $3$-manifolds.  The authors are deeply grateful for his
help. 

The first-named author thanks
the NSF for partial support.  Some of the first author's preliminary ideas
were formulated while visiting
David Epstein at the University of Warwick Mathematics Institute.  The
second-named author thanks the NSF and Boston College for
partial support; the USC and Caltech Mathematics Departments for 
supporting
him as a visitor while much of
this work was done; and Jeff Weeks, Alan  Meyerhoff, and especially Rob
Gross for computer assistance.  The
third-named author thanks the NSF for partial support, and the Geometry 
Center and the Berkeley Mathematics Department for their support. 

Finally, we thank the referees for the magnificent job they did.  The first set 
of referees read our paper thoroughly and made numerous excellent 
suggestions for improving the exposition.  Further, their discussion of issues 
related to computer-aided proofs crystallized many of these topics in our 
minds.

The second set of referees also read the paper thoroughly, and we are 
grateful for their elegant suggestions concerning the exposition. 
They also checked the programs in great detail, and approached this task 
with a desire to understand what was really going on behind the scenes.  
Their ingenious robustness checks raise the confidence level in our proof, 
and their thought-provoking comments should help us when we attempt to 
use the computer to help us push across the frontier of our current results.
 






