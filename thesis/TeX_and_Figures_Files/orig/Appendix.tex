 \def\dcirc{\leavevmode\setbox0=\hbox{h}\dimen2=\ht0 \advance\dimen2
by-1ex\rlap{\raise.67\dimen2\hbox{\char'27}}D}
\def\ncirc{\leavevmode\setbox0=\hbox{h}\dimen2=\ht0 \advance\dimen2 by-1ex\rlap{\raise.67\dimen2\hbox{\char'27}}N}

\vglue8pt
\centerline{\bf Appendix}
\vglue2pt
\centerline{\bf Review of the theory of insulators in hyperbolic 3-manifolds}
\vglue8pt
 
Given that Theorem 0.2
is the main technical result of this paper, we herewith present an appendix
which
briefly recalls from [G] various definitions,  examples and properties of
insulators in hyperbolic 3-manifolds.  Also, we briefly recall  why
the existence of a noncoalescable insulator family for a geodesic $\delta$ in a
closed orientable hyperbolic 3-manifold $N$ allows us to conclude that  any
irreducible
3-manifold
$M$, homotopy equivalent to $N$, is indeed homeomorphic to $N$.

\demo{Definition {\rm A.1}}  Let $G$ be a group of homeomorphisms of
$S^2$ and ${\cal A}=\{A_i\}$ a countable set of pairwise disjoint $G$-equivariant
pairs of
points  of $S^2$, i.e.\ if  $g\in G, A_i\in {\cal A}$, then $g(A_i)\in {\cal
A}$.  Let $\{\lambda_{ij}\}$ be a collection of smooth simple closed curves
in $S^2$.  The set $\{\lambda_{ij}\}$ is called a $(G,{\cal A})$ {\it
insulator  family} and each $\lambda_{ij}$  is called an {\it insulator} if
\begin{itemize}
\item[i)] {\it Separation}: If $i\neq j$, then $\lambda_{ij}$ separates
$A_i$ from $A_j$.
\item[ii)]  {\it Equivariance}: If $g\in G$, then $g(\lambda_{ij})$ is the
curve
associated
to the pair $g(A_i),g(A_j)$.  Also $\lambda_{ij}=\lambda_{ji}.$

\item[iii)] {\it Convexity}: To each $\lambda_{ij}$ there exist round
circles respectively
containing
$A_i$ and $A_j$ and disjoint from $\lambda_{ij}$.

\item[iv)]  {\it Local Finiteness}: For every $\epsilon>0$ there exist only
finitely many $\lambda_{ij}$ such
that $i$ is fixed and diam$(\lambda_{ij})>\epsilon$.
\end{itemize}

\enddemo

\vglue-8pt
{\it Definition} A.2.   A $(G,{\cal A})$ insulator family is {\it
noncoalescable}
if it satisfies the following {\it no tri\/{\rm -}\/linking} property.  For no $i$,
do  there exist
$\lambda_{ij_1},\lambda_{ij_2},  \lambda_{ij_3}$ whose union separates the
points of
$A_i$.   A hyperbolic 3-manifold satisfies the {\it insulator condition} if
there
exist  a geodesic $\delta$ in $N$ and a
$(\pi_1(N),\{\partial\delta_i\})$
noncoalescable insulator family.   Here $\{\delta_i\}$ is the set of lifts of
$\delta$ to ${\bf H}^3$. \eject 

\demo{{E}xample {\rm A.3}}  
Let $\delta$ be a simple closed geodesic in the
hyperbolic\break $3$-manifold $N$.  Let $\delta$ lift  to a collection
$\Delta=\{\delta_i\}$ of hyperbolic lines in ${\bf H}^3$.  To each pair
$\delta_i,\delta_j$, there exists the {\it midplane} $D_{ij}$, i.e.\ the
hyperbolic halfplane orthogonal to and cutting the middle of the {\it
orthocurve}
(i.e.\ the shortest line segment) between $\delta_i$ and $\delta_j$.
Each
$D_{ij}$  extends to a circle $\lambda_{ij}$ on $S^2_{\infty}$, which
separates  $\partial \delta_i$ from $\partial\delta_j$. We call the
insulator family
$\{\lambda_{ij}\}$ the {\it Dirichlet insulator family}.    If
tuberadius$(\delta)>\log(3)/2$,
then this family is noncoalescable, the idea being that from the point of
view of
$\delta$ each $\lambda_{ij}$ takes up less than 120 degrees of visual
angle;  thus there can
be no tri-linkings among the $\lambda_{ij}$'s. 
\enddemo

The content of Section 2 is a construction of a new insulator family for $\delta$ called the Corona insulator family.  If
tuberadius($\delta)>{\log}(3)/2$, then this family is just the Dirichlet family.  Lemma 2.5 provides a sufficient condition for the
Corona family to be noncoalescable. 

Theorem 0.2 is established by showing that the Corona family for $\delta$ is noncoalescable if $N\neq $ Vol3 while if $N=$ Vol3, then an insulator which 
is a hybrid of the Dirichlet and Corona insulators is noncoalescable.

Noncoalescable insulator families are important because of the following
result.

\vglue4pt {\elevensc Theorem  A.4} [G].  {\it Let $N$ be a closed  orientable hyperbolic
$3$\/{\rm -}\/manifold containing a geodesic $\delta$ having a noncoalescable
insulator family{\rm ,}   then\/{\rm :}}
\begin{itemize}
\ritem{i)}  {\it If $f:M\to N$ is a homotopy equivalence where $M$ is
an irreducible $3$\/{\rm -}\/manifold{\rm ,} then $f$
is  homotopic to a homeomorphism.}
\ritem{ii)} {\it If $f,g:N\to N$ are homotopic  homeomorphisms{\rm ,} then $f$ is
isotopic to $g$.}
\ritem{iii)} {\it The space of hyperbolic metrics on $N$ is path connected.}
\end{itemize}
 

{\it Idea of the proof of} i).  The plan is to find a simple closed curve $\gamma$ in $M$ and a homotopy of $f$ to $g:M\to N$ so that
$g|N(\gamma):N(\gamma)\to N(\delta)$ is a homeomorphism, 
$g(M-\ncirc(\gamma))=N-\ncirc (\delta)$ and $g|M- \ncirc (\gamma):M- \ncirc (\gamma)\to
N- \ncirc (\delta)$ is a homotopy equivalence.  Since $N- \ncirc (\delta)$ is
Haken we invoke
Waldhausen [Wa] to conclude that $g|M- \ncirc (\gamma)$ is homotopic to a
homeomorphism rel
boundary and hence $f$ is homotopic to a homeomorphism.

There are many technical difficulties in carrying out the above plan, the
primary one being how to
find the curve $\gamma$.  Here is another view of this problem.  Given a
hyperbolic 3-manifold $Q,$ it
is well known that any nontrivial element $\alpha$ of $\pi_1(Q)$ is
represented by a unique geodesic
$\eta$ in $Q.$  Suppose that we are given $Q$ with some random Riemannian
metric and we do not know the
hyperbolic metric. Then \pagebreak how are we supposed to find $\eta$?
  In our setting if
$f_*:\pi_1(M)\to\pi_1(N)$ is the induced map on the fundamental group then our goal is to find the $\gamma$ which represents
$f^{-1}_*([\delta])$.  We outline how to solve this problem if $\delta$ has a noncoalescable insulator family. 
 
 
 
\nonumproclaim{Theorem A.5 {\rm [G]}} If $f:M\to N$ is  a homotopy
equivalence{\rm ,} where
$M$ is an irreducible  $3$\/{\rm -}\/manifold  and $N$ is a closed
hyperbolic $3$\/{\rm -}\/manifold{\rm ,} then there exist  a closed hyperbolic $3$\/{\rm -}\/manifold $X$ and
regular covering maps  $p_1:X\to M,$ $ q_1:X\to N$ such that $f\circ
p_1$ is
homotopic to $q_1$.  A lift $\tilde f: {\bf H}^3\to{\bf H}^3$ extends to
${\rm id}:S^2_{\infty} \to S^2_{\infty}$.
Furthermore the action of $\pi_1(M)$ on ${\bf H}^3$
extends to a Mobius action on $S^2_{\infty}$
which
is identical to the action of $\pi_1(N)$ on $S^2_{\infty}$.\endproclaim


In other words this theorem says that $\tilde M$ and $\tilde N$ have common spheres at
infinity.  Thus $\lambda_{ij}$  is a $\{\pi_1(M),\{\partial\delta_i\}\}$
noncoalescable insulator family.

Here is how to construct $\gamma$.  To each smooth simple closed
curve
$\lambda_{ij}$ in  $S^2_{\infty}$, there exists a lamination
$\sigma_{ij} \subset {\bf H}^3$ by
embedded least area planes,  with limit set
$\lambda_{ij}$ such that $\sigma_{ij}$ lies in a 
uniformly fixed width
hyperbolic regular
neighborhood of the hyperbolic convex
hull of $\lambda_{ij}$.   Here ${\bf H}^3$ is given the Riemannian metric induced from $M$,
while the  fixed width measurement is via the hyperbolic metric
induced from $N.$
Fix $i$.  Let
$H_{ij}$ be the ${\bf H}^3$-complementary region of $\sigma_{ij}$
containing the ends of $\delta_i$.
It turns out that $\mathbold{\cap}_j H_{ij}$ contains a component
$V_i=(\dcirc)^2 \times {\bf R}$
which projects to an open  solid torus in $M$.  Define
$\gamma$ to be the core of this solid torus
and
$\gamma_i$ the lift which lives in $V_i$.
Up to isotopy,
$\gamma$ is independent of all choices, in particular the choice of
Riemannian metric on
$M$.

To first approximation think of $\sigma_{ij}$ as a properly embedded plane
and $V_i$ as
the intersection of topological half spaces in ${\bf H}^3$.  
(It is not known if the leaves of
$\sigma_{ij}$ must be properly embedded.)
Without the
no-tri-linking
condition this intersection might be empty.  Note that if ${\bf H}^3$ is given the
hyperbolic metric and $\lambda_{ij}$ is the Dirichlet insulator, then each
$\sigma_{ij}$ is a totally geodesic plane and the projection of $V_i$ into
$N$ is a
solid torus regular neighborhood of $\delta$.  This statement is also true if
$\lambda_{ij}$ is any noncoalescable insulator family; however, the proof
requires the
convexity condition.   Thus the above construction using the Riemannian
metric induced
from $M$ (resp.\ $N$) yields the curves $\{\gamma_i\}\subset
{\bf H}^3$ (resp.\ $\{\delta_i\})$.  
These collections $\{\gamma_i\}=\Gamma$ and 
$\{\delta_i\}=\Delta$ are
${\bf B}^3$-links (i.e.\ sets of pairwise disjoint properly embedded arcs in
${\bf B}^3$ whose
restriction to ${\bf H}^3$ is locally finite). 
\vglue4pt
We now give a hint as to how to find the desired $g:M\to N$.
  The
Riemannian metric
$\mu_0$ on $X$ induced from $M$ and the hyperbolic metric
$\mu_1$ on $X$ induced from $N$
lift
to
$\pi_1(X)$ equivariant metrics $\tilde\mu_t, t\in \{0,1\}$ on  ${\bf H}^3$; so
the above construction  applied to the  $(\pi_1(X),\{\partial\delta_i\})$
insulator family
$\{\lambda_{ij}\}$ with respect
to the  $\tilde\mu_t$ metric  yields   a link $\tau_t$  in $X$.
Since the
isotopy class of $\tau_t$ is independent of $t$,  $\tau_0=p_1^{-1}(\gamma)$  is isotopic to  $\tau_1=q_1^{-1}(\delta)$.  We  conclude
that the
${\bf B}^3$-link $\Gamma$ is  equivalent to the ${\bf B}^3$-link
$\Delta$; i.e.\ there exists a homeomorphism $k:( {\bf B}^3,\Gamma) \to
({\bf B}^3,\Delta)$ so that $k\mid S^2_{\infty}=$ id.

The above mentioned $g$ arises in proving the next result.  Here
$p: {\bf H}^3\to M$, and $q: {\bf H}^3\to N$ are the universal covering maps arising
from A.5.

\nonumproclaim{Proposition A.6 {\rm [G]}}  Let $f:M\to N$ be a homotopy
equivalence
between the closed orientable hyperbolic $3$\/{\rm -}\/manifold $N$ and
the
irreducible manifold
$M$.
If there exist  a simple closed curve  $\gamma\subset M${\rm ,} a
geodesic $\delta \subset N$ and a homeomorphism
$k:({\bf B}^3,p^{-1}(\gamma))\to
({\bf B}^3,q^{-1}(\delta))$ such that $k\mid \partial {\bf B}^3  = {\rm id}${\rm ,}
then
$f$ is
homotopic to
a homeomorphism. 
\endproclaim

{\it Remark} A.7.  As mentioned above, if $\rho_0$ is a hyperbolic
metric on $N$, then   associated to a  nontrivial element $\alpha$ of
$\pi_1(N) $,
there exists a unique  geodesic $\delta_0 \subset N$ representing
$\alpha$.  A second hyperbolic metric $\rho_1$ on $N$ will give rise to the
geodesic $\delta_1$.  The Mostow rigidity theorem implies that there exists an
isometry of $N_{\rho_0}$ to $N_{\rho_1}$ homotopic to the identity.  A consequence of
Theorem 0.1 is that  the isometry is {\it isotopic} to the identity.
Thus Theorem
0.1 is a strengthening of Mostow rigidity.  To appreciate the difference,
note that the
Mostow rigidity theorem does not imply that $\delta_0$ is isotopic to
$\delta_1$,
while Theorem 0.1 does.    The proof is a
consequence of the insulator technology.   Here is a hint.  The
construction of
$\gamma$ is independent, up to isotopy, of all choices.  Thus applying the
insulator
construction to $N$ respectively using the metrics $\rho_0$ and $\rho_1$
yields the
curves
$\delta_0$ and
$\delta_1$, which are necessarily isotopic. 


 



