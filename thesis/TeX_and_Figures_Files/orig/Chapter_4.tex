\section{Applications}
 
We provide some applications.  First, we give a partial answer to Problem 1.16.  Second a  lower bound for the volume of hyperbolic 3-manifolds is produced.  Finally, we give a relationship between isotopic closed curves in a hyperbolic 3-manifold and essential links in $B^3.$

\nonumproclaim{Theorem 4.1} If $\delta$ is a shortest geodesic in
the closed orientable hyperbolic $3$\/{\rm -}\/manifold $N,$ then either
\begin{itemize}
\ritem{i)} {\rm tuberadius(}$\delta) > \ln(3)/2${\rm ,} or
 
\ritem{ii)} $1.0953/2 > {\rm tuberadius}(\delta) > 1.0591/2$
and {\rm Relength(}$\delta) > 1.0595${\rm ,} or

\ritem{iii)} {\rm tuberadius(}$\delta) = 0.8314\ldots/2$ and $N = {\rm Vol3}.$
\end{itemize}
\endproclaim
        
{\it Proof}.  If 
$0.5 < {\rm tuberadius}(\delta)=d/2 \le \ln(3)/2,$ then by Corollary 1.29 and Table 1.2 we see that 
$1.0953/2 > {\rm tuberadius}(\delta) > 1.0591/2$
and Relength($\delta) > 1.0595.$
If ${\rm tuberadius}(\delta)=d/2 \le 0.5$ 
then the  parameter
$(L,D,R)\in {\cal P}$ 
associated to $N$ must be in 
${\cal R} \cap {\cal T} = \exp^{-1}(X_0 \cap {\cal S}).$ 
It then 
follows by Corollary 3.15 that $N = {\rm Vol3}.$\hfill\qed

\vglue12pt {\it Remark} 4.2.    The previous best lower bound for the volume of hyperbolic 3-manifolds was on the
order of 0.001 (see [GM1]).  Using the results of the present paper and the method of [M1] it is easy to improve this
to 0.1.  However, Gehring and Martin provide an improved tube-volume formula in [GM2] and we use their formula to
get a lower bound of $0.16668\ldots.$ The Gehring-Martin tube-volume formula for manifolds (as opposed to
orbifolds) is 
$${\cal V}(t) = \sqrt 3 \tanh(t) \cosh(2t) {\rm Arcsinh}^2(\sinh(t)/\cosh(2 t))$$
where $t$ is the radius of the embedded solid tube.  Note that the length of the core geodesic is irrelevant. 

\nonumproclaim{{C}orollary 4.3} ${5 \over 2 \sqrt 3} {\rm Arcsinh}^2\left({ \sqrt 3 \over 5}\right) = 0.16668\ldots$
is a lower bound for the volume of closed hyperbolic $3$\/{\rm -}\/manifolds.
\endproclaim

\demo{Proof} Corollary 1.7 of [GM2] applies the tube-volume formula ${\cal V}(t)$ to tubes of radius at least
$\ln(3)/2$ and produces 
$${\rm Vol}(N) \ge {\cal V}(\ln(3)/2) = {5 \over 2 \sqrt 3} {\rm Arcsinh}^2\left({ \sqrt 3 \over 5}\right) =
0.16668\ldots\; .$$

So, if a hyperbolic $3$-manifold $N$ has a shortest geodesic with tuberadius
greater than or equal to $\ln(3)/2$ then we are done.  
If tuberadius is less than $\ln(3)/2$ then Theorem 4.1 implies that
either $N$ is Vol3, 
or $l$ and $d$ are bounded as follows: $l \ge 
1.0595$  and $d \ge 1.0591.$
 In the first case, 
  ${\rm Vol}(N)  = {\rm Vol}({\rm Vol3}) = 1.01\ldots,$
while in the second case, plugging the $l$ and $d$ bounds into the tube-volume formula $\pi l \sinh^2(d/2)$
 we get ${\rm Vol}(N) > 1.02$.
\enddemo

{\it Remark} 4.4. Recently, A. Przeworski has proved a lower bound for volume of 0.276796 (see [P]).  His method is
to develop an improved version of Gehring and Martin's tube-volume formula and then to apply it to our tuberadius
results.\footnote{{\it Note added in proof} (January 2003): Przeworski (see [P2]) has improved the volume
lower bound to 0.3315 by combining his tube packing results with Theorem 4.1 and work of I. Agol (see [A]).} 

\nonumproclaim{Theorem 4.5}  Let $k_1, k_2$  be simple closed curves in $N$ such that $k_1$ is
a geodesic.  Then $k_1$ is isotopic to $k_2$ if and only if as $B^3$\/{\rm -}\/links
$q^{-1}(k_1)$ is equivalent to $q^{-1}(k_2)$ where $q : {\bf H^3} \to N$ is the universal covering
projection.
\endproclaim

{\it Proof}.  Apply Corollary 5.6 of [G]. 
Recall that $\Gamma$ and $\Delta$ are equivalent 
$B^3$-links  if there is a homeomorphism of $B^3$ which takes $\Gamma$ to
$\Delta$ and fixes $S^2$ pointwise.\hfill\qed\vglue6pt

{\it Remark} 4.6.  A similar argument extends Theorem 4.5 to homotopy essential
links which lift to trivial $B^3$-links.  The general case of Conjecture 5.5A of [G] is still open.

 

 

