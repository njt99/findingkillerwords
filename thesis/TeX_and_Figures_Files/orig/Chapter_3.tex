 \def\distance{{\rm distance}}
\def\length{{\rm length}}


\section{Vol3}
 
In Section 1, we showed how a certain 3-complex-dimensional
parameter space can be used to attack the question of whether all
closed, oriented hyperbolic $3$-manifolds have a tube of radius
$\ln(3)/2$ around a shortest geodesic and hence have noncoalescable
insulator families.  In particular, Proposition 1.28 showed that a
hyperbolic 3-manifold has such a tube unless its fundamental group
contains a marked  group associated to a point in the seven exceptional
boxes
$X_0,\ldots , X_6$.

In Section 2, we showed that any shortest geodesic in a closed oriented
hyperbolic 3-manifold has a noncoalescable insulator family unless its
fundamental group contains a marked  group associated to a point in
the exceptional box $X_0.$   In this chapter, we show that the box 
$X_0 \subset {\cal W},$ or equivalently,
the region ${\cal R} = \exp^{-1}(X_0) \subset {\cal P},$ is associated to a unique closed manifold and that manifold has a noncoalescable insulator family. 

\nonumproclaim{Proposition 3.1}  
The point $(\omega, \omega, 0)$ is the unique point in 
${\cal T} \cap {\cal R};$ here $$\omega=
\ln((-1 - i\sqrt3)/2 - (-6 + 2i\sqrt3)^{(1/2)}/2)
\approx 0.83144 - 1.94553i.$$ This point corresponds to {\rm Vol3,} the closed hyperbolic
 $3$\/{\rm -}\/manifold of conjecturally third smallest volume.  Further{\rm ,}
 if $N$ is a closed orientable hyperbolic
$3$\/{\rm -}\/manifold{\rm ,} either $N = {\rm Vol3}$ or {\rm maxcorona(}$\delta) < 2\pi/3$
 for $\delta$ a shortest geodesic in $N.$
\endproclaim

\nonumproclaim{Proposition 3.2} Any shortest geodesic in  {\rm Vol3} satisfies the insulator condition.
\endproclaim

  {\it Remark} 3.3.
Topologically, Vol3 is (3,1)  surgery on manifold m007 in the census of
cusped hyperbolic 3-manifolds (see [CHW]).  It is also $(-3,2)$\break $(-6,1)$ surgery on the left-handed Whitehead
link,  link $5_2^2$ in the standard knot tables.  The program {\it SnapPea} (see [W]) gives an
experimental proof that Vol3 is hyperbolic and that its volume is that of the
regular ideal $3$-simplex.  A rigorous proof can be found in [JR] or [CGHN].  Previously,
Hodgson and Weeks (unpublished) had found an exact Dirichlet domain for Vol3; that is,  the face
pairings were expressible as explicit matrices with coefficients in a finite
extension $F$ of ${\bf Q}$ and they obtained equations in $F$ for the various faces.  See Remark 3.14.

\demo{{R}emark {\rm 3.4}}
Outline of Proof of Proposition 3.1: By the proof of Proposition 2.8, if $N$ has maxcorona$(\delta) \ge 2\pi/3$, then it must have an $(L,D,R)$ parameter in
${\cal R}.$   Using Proposition 1.28, we can further assert that this parameter
is in ${\cal T} \cap {\cal R}.$   
A geometric argument (Lemma 3.7) which utilizes Lemmas 3.5 and 3.6 shows that $R = 0$, and an algebraic argument (Lemmas 3.8, 3.9, 3.11, 3.13) shows that $L = D = \omega,$ where $\exp(\omega)$ is a root of the polynomial $z^4 + 2z^3 + 6z^2 + 2z + 1.$  This implies ([JR]) that ${\rm Vol3} = {\bf H^3}/G$ where $G$ is the subgroup of $\pi_1(N)$ generated by $f,w$ associated to the parameter $(L, D, R) = (\omega, \omega, 0).$ Therefore, $N$ is covered by Vol3.  Because Vol3 covers no 3-manifolds nontrivially ([JR]), it follows that $N = {\rm Vol3}.$
\enddemo

\nonumproclaim{Lemma 3.5}\hskip-8pt  Let ${\cal R} = \exp^{-1}(X_0).$ If
$\alpha= (L,D,R)=(l+it, d+ib, r+ia)  \in {\cal R} \cap {\cal T},$ then $f_\alpha$ and 
$w_\alpha$ satisfy the relations
\begin{itemize}
\ritem{i)}  $wFwfwfWfwf,$

\ritem{ii)} $wFwfwwfwFw.$
\end{itemize}
\noindent Here $W$ denotes $w^{-1},$ and $F$ denotes $f^{-1}$. 
\endproclaim

\demo{Proof}
  In i), ii) above and what follows below we suppress the subscripts
$\alpha.$   Because i), ii) are cyclic permutations of the quasi-relators $r_2,\ r_1$ corresponding to the $X_0$ box of Proposition 1.28, it follows that if 
$h=wFwfwfWfwf$     or $h=wFwfwwfwFw,$ then ${\rm Relength}(h) < {\rm Relength}(f)$ throughout ${\cal R}.$  Since
$\alpha\in {\cal T},$  $f$ is a
shortest element and so $h={\rm id}.$  \enddemo

\nonumproclaim{Lemma 3.6}  The following substitutions in Lemma {\rm 3.5}
 give rise to three sets of new relators.
\begin{itemize}
\ritem{a)} In {\rm i), ii)} exchange $f$ and $w$  {\rm (}\/hence{\rm ,} exchange $F$ and $W${\rm ).}

\ritem{b)} In {\rm i), ii)} exchange $f$ and $F$.

\ritem{c)} In {\rm i), ii)} exchange $w$ and $W$.
\end{itemize}

\endproclaim

\demo{Proof}  
a)  First, a cyclic permutation of relator i) gives relator i) with $f$
replaced by
$w$ and $w$ replaced by $f$.  Second, one readily obtains the relator $fWfwffwfWf$ from i), ii), because $fWfwf = (wFwfw)^{-1} = wfwFw = (fwfWf)^{-1}$ where the first and third equalities follow from i) and the second from ii).
\vglue4pt
b)  Again it is routine to obtain b) from i), ii).
\vglue4pt
c)  Conclusion c) follows from a) and b). \enddemo


\nonumproclaim{Lemma 3.7} If $(L,D,R) \in \cal R \cap {\cal T},$ then $R=0$.
\endproclaim

\demo{Proof}  Let $\{G,f,w\}$ be the marked group corresponding to the parameter $(L, D, R).$
Figure 3.1  shows a schematic picture of geodesics
$B_{(0;\infty)}$, $W(B_{(0;\infty)})$, $w(B_{(0;\infty)})$, $f(w(B_{(0;\infty)}))$, $f(W(B_{(0;\infty)})),$ where, in the figure we have, for convenience, abbreviated $ B_{(0;\infty)}$ to $B$. 
Also, it shows the images of the orthocurve $O$ from
$W(B_{(0;\infty)})$  to $ B_{(0;\infty)}$ after translation by $w, f,$ and $fw.$  
Finally, it shows the
orthocurves  $O_1$ from $fW(B_{(0;\infty)})$ to $W(B_{(0;\infty)})$ 
and $O_2$ from $w(B_{(0;\infty)})$ to $fw(B_{(0;\infty)}).$ Note that
Figure 3.1 displays the situation where ${\rm Re}(R)>0.$  It is also 
{\it a priori}
possible that $O_2$  might intersect $w(B_{(0;\infty)})$ on the other side of $w(O).$
There are other similar  possible inaccuracies.
 

Note that $\sigma_1=fwffw\in G$ sends
geodesic $W(B_{(0;\infty)})$ to $fw(B_{(0;\infty)})$ and $\sigma_2=wFFwF\in G$ sends the
geodesic $fW(B_{(0;\infty)})$ to $w(B_{(0;\infty)})$.  
Now $\sigma_2^{-1}\sigma_1=fWffWfwffw$ is a relator
of $G$, since it is a cyclic permutation of relator ii) with $f$ replaced by $w$
and $w$ replaced by $f.$  Thus $\sigma_1=\sigma_2$ and hence
$\sigma_1(O_1)=O_2.$




Figure 3.1 gives rise to some right-angled hexagons which we will study. 
For a careful development of the theory of right-angled hexagons, see
[F]; we give an abbreviated treatment here.
A right-angled hexagon consists of a cyclically
ordered 6-tuple of oriented geodesics $\lambda_1,\cdots,\lambda_6$
in ${\bf H^3}$ such that $\lambda_i$ intersects $\lambda_{i+1} \pmod 6$ orthogonally.  Each ``edge" of the hexagon is labeled by the
complex number $e_i = d_{\lambda_i}(\lambda_{i-1},\lambda_{i+1});$ 
see Definition 1.4 and Lemma 1.5 for the definition of $d_\alpha(\beta,\gamma)$ and some of its properties.
The following is true:

\begin{itemize}
\item[(3.1)] The effect of reversing the orientation of
$\lambda_i$ is to change $e_i$ to $-e_i, e_{i+1}$ to $e_{i+1} + \pi i,$ and
$e_{i-1}$ to $e_{i-1}+\pi i.$
\end{itemize}

Figure 3.1 gives rise to the two right-angled hexagons drawn in Figure 3.2.  (Figure 3.2a may be inaccurate for
the following reason.  It is not clear
 whether the head of $O_1$ should be placed in  front of the tail
of $O$ or behind the tail of $O.$  A similar statement holds for
the tail of $O_1$ and for Figure 3.2b.) Assume that in Figure
3.2a, $\lambda_1$  corresponds to $ B_{(0;\infty)}$ and the edges are
cyclically  ordered counterclockwise.  Then $e_6=D,\ 
e_1=L,\ e_2=-D.$     We now show that if $e_5$ has value $c,$ then
$e_3$ has value $c+\pi i.$ Observe that there is an order-2
rotation $\tau$ of ${\bf H^3}$ about an axis orthogonal to $ B_{(0;\infty)}$ which
reverses the orientation on $ B_{(0;\infty)}$ and takes the oriented
orthocurve $O$ to the oriented orthocurve $f(O).$  Since 
distance($W(B_{(0;\infty)}), B_{(0;\infty)})=\ $distance$(fW(B_{(0;\infty)}), B_{(0;\infty)})$ it follows that
$\tau(fW(B_{(0;\infty)}))=-W(B_{(0;\infty)})$ and $\tau(W(B_{(0;\infty)}))=-fW(B_{(0;\infty)})$ where the minus
sign indicates that the orientation has been reversed.   This in
turn implies   that $\tau(O_1)=-O_1$ and therefore by (3.1)
that  $e_3=e_5+\pi i.$ 
 





Let $\phi$ be the isometry of ${\bf H^3}$ which is an  $r+i(\pi+a)$
translation of $ B_{(0;\infty)}.$  Thus $\phi$ commutes with $f,$ and $\phi(B_{(0;\infty)})= B_{(0;\infty)}, \ \phi(O) = -w(O),\ 
\phi(f(O)) = -fw(O),$ $\phi(W(B_{(0;\infty)}))=w(B_{(0;\infty)})$
and $\phi(fW(B_{(0;\infty)}))=fw(B_{(0;\infty)})$  which in turn implies that
$\phi(O_1)=-O_2.$  If the hexagon of Figure 3.2b with
edges $\lambda^\prime_1,\ldots,\lambda^\prime_6$ is
counterclockwise cyclically oriented so that 
$\lambda^\prime_1$ denotes the oriented geodesic $ B_{(0;\infty)}$, then again
by (3.1),  $e_5^\prime=c$ and
$e_3^\prime=c+\pi i.$ 


Via elements of $G,$ we translate each of $W(B_{(0;\infty)})$, $fW(B_{(0;\infty)})$,  $w(B_{(0;\infty)})$,
$fw(B_{(0;\infty)})$ to $ B_{(0;\infty)},$ and after translation we obtain from Figure 3.2
the various distance relations 
$\{e_1, e_3, e_5, e_1^\prime, e_3^\prime, e_5^\prime\}$
schematically indicated on
Figure~3.3.  Each $O_i^*$ is a $g \in G$ translation of $O_i$ 
such that $O_i^*$ has an 
endpoint on $ B_{(0;\infty)}.$  There are two classes of such translates,
 one where $O_i^*$ points into $ B_{(0;\infty)}$ and one where
$O_i^*$ points out.  Call the former (resp.\ latter) a
pointing in (resp.\ out) $O_i^*.$  Actually Figure 3.3 includes
three more relations.   Because the oriented $O_1$ is a $G$-translate
of the oriented $O_2$ and $f$ is a primitive element of $G$ which
fixes $ B_{(0;\infty)},$ it follows that 
$${\rm distance}(({\rm pointing\ in}\ O_1^*),
({\rm pointing\ in }\ O_2^*))=0 \pmod L$$  and 
$${\rm distance}(({\rm pointing\ out}\ 
O_1^*), ({\rm pointing\ out}\ O_2^*))=0 \pmod L.$$  
Finally
distance($w(O),O)=R.$  
\vglue8pt
\figin{fig3.1n}{800}
\centerline{Figure 3.1.  Various geodesics around $B = B_{(0;\infty)}$ in ${\bf H^3}.$}
\pagebreak


\centerline{\BoxedEPSF{fig3.2n.eps scaled 700}}
\vglue9pt
\begin{quote}{Figure 3.2. Two right-angled hexagons arising from Figure 3.1.  Here $B$ denotes the axis
$B_{(0,\infty)}$.}
\end{quote}
\figin{fig3.3n}{700}
\begin{quote} Figure 3.3. The values shown are the distances along $B_{(0;\infty)}$, between the indicated
oriented ortholines and $O^\ast_i$ denotes a $g\in G$ translate of the orthocurve $O_i$.
\end{quote}


We therefore obtain the following two
equations
$$  c-R+L+c+\pi i = 0 \pmod L\ \ \hbox{ and } \ \  
 c-L+R+c+\pi i = 0 \pmod L\leqno{(3.2)}$$
 hence
$$
2R=0 \pmod L\ \ \hbox{ and }  \ \ 4 c =0 \pmod L.
\leqno{(3.3)} $$
By the $L^\prime$ and $R^\prime$ ranges for $X_0$ provided in Proposition 1.28, it is easy to compute that  for
each element of ${\cal R} = \exp^{-1}(X_0),\  |{\rm Re}(R)| < |{\rm Re}(L)/2|,$ and that in fact $R = 
\exp^{-1}(R^\prime)$ is close to $0$ (thereby eliminating the possible solution $R = \pi i$).  It then  follows  that
$R=0$ for $(L, D, R) \in {\cal R} \cap {\cal T}.$   \enddemo

\nonumproclaim{Lemma 3.8}  If $(L,D,R) \in \cal R \cap {\cal T},$ then $L=D.$
\endproclaim

\demo{Proof}  We will now  use the exponential coordinates $l=\exp(L)=L',\ d=\exp(D)=D'.$  These $l,d$ should not
be confused with the $l+it, d+ib$ used above.  In the following calculations
Mathematica [Math] was used to perform matrix
multiplication of $2\times 2$ matrices with coefficients
rational functions in the variables $\sqrt l, \ \sqrt d$.  
Plugging in $R = 0$ in Lemma 1.24 we get the following ${\rm SL}(2,{\bf C})$ representatives of $f,F,w,W$ (here $\{G,f,w\}$ is the marked group
corresponding to $(L,D,R)$). Because the $R$ term drops out, we can express the matrices of $w$ and $W$ as functions of $d$ alone.
\begin{eqnarray*}
f(l)& = &\left(\matrix{\sqrt{l} & 0 \cr 0 & \sqrt{l}\cr}\right)\\[4pt]
 F(l) &= &\left(\matrix{1/\sqrt{l} & 0 \cr 0 &
1/\sqrt{l}\cr}\right)\\[4pt]
 w(d) &=& \left(\matrix{(\sqrt{d}+1/\sqrt{d})/2 & (\sqrt{d}-1/\sqrt{d})/2\cr
(\sqrt{d}-1/\sqrt{d})/2 & (\sqrt{d}+1/\sqrt{d})/2\cr}\right)\\[4pt]
W(d)& = &\left(\matrix{(\sqrt{d}+1/\sqrt{d})/2 & 
(-\sqrt{d}+1/\sqrt{d})/2\cr (-\sqrt{d}+1/\sqrt{d})/2 & (\sqrt{d}+1/\sqrt{d})/2\cr}\right) 
\end{eqnarray*}


  Let $Y$ be the relator
$FwfwfWfwfw,$ which is a cyclic permutation of
relator i) of Lemma 3.5.   Multiplying
this product of 10 matrices we obtain the
following matrix entries for $Y(l,d) = Y$ which we know is 
the identity in ${\rm PSL}(2,{\bf C})$ and hence 
$Y = \pm I.$ (At the end of the proof of Lemma 3.9, we shall
see that $Y = I.$)
\begin{eqnarray*}
Y_{11}&=&((1 + d) (1 - 2d^2 + d^4 + 8dl - 16d^2l \\
              &&  +\ 8d^3l - 2l^2 + 4dl^2 - 4d^2l^2 + 4d^3l^2 \\
        &&-\ 2d^4l^2 + l^4 + 4dl^4 + 6d^2l^4 + 4d^3l^4\\
      &&  +\ d^4l^4))/(32d^{(5/2)}l^{(5/2)});\\
Y_{12}&=& ((-1 + d)
(1 + 4d + 6d^2 + 4d^3 + d^4 + 4dl \\
        &&+\       8d^2l + 4d^3l - 2l^2 - 12d^2l^2 - 2d^4l^2 \\
&&+\ 
        4dl^3 + 8d^2l^3 + 4d^3l^3 + l^4 + 4dl^4 \\
&&+\
        6d^2l^4 + 4d^3l^4 + d^4l^4))/(32d^{(5/2)}l^{(5/2)});\\
Y_{21}& =& ((-1 + d)
(1 + 4d + 6d^2 + 4d^3 + d^4 + 4dl \\
&&+\
                8d^2l + 4d^3l - 2l^2 - 12d^2l^2 - 2d^4l^2 \\
&&+\
        4dl^3 + 8d^2l^3 + 4d^3l^3 + l^4 + 4dl^4\\
&&+\
        6d^2l^4 + 4d^3l^4 + d^4l^4))/(32d^{(5/2)}l^{(3/2)});
\\ Y_{22}& =& ((1 + d)(1 + 4d + 6d^2 + 4d^3 + d^4 - 2l^2 + 4dl^2\\
&&- \ 
                4d^2l^2 + 4d^3l^2 - 2d^4l^2 + 8dl^3 \\
&&-\ 
        16d^2l^3 + 8d^3l^3 + l^4 - 2d^2l^4 + d^4l^4))         /(32d^{(5/2)}l^{(3/2)})
. \end{eqnarray*}


Since $G$ is generated by $f,$ an $L$ translation  along $B_{(0;\infty)},$
and by $w,$ a $D$ translation along $B_{(-1;1)},$ it follows from 
Lemma 3.6 that the relation $Y=I$ holds with $l$
and $d$ switched.  Thus $0=Y_{12},\ $ and $0=Y_{12}$ (with $l,d$ switched)
which implies that
\begin{eqnarray*}
0&=&(1 + 4d + 6d^2 + 4d^3 + d^4 + 4dl + 8d^2l\\
&&+\ 
                4d^3l - 2l^2 - 12d^2l^2 - 2d^4l^2\\
&&+\ 
                4dl^3 + 8d^2l^3 + 4d^3l^3 + l^4 + 4dl^4\\
&&+\ 
                6d^2l^4 + 4d^3l^4 + d^4l^4)\\
&&-\ 
                (1 + 4l + 6l^2 + 4l^3 + l^4 + 4ld + 8l^2d\\
&&+\ 
                4l^3d - 2d^2 - 12l^2d^2 - 2l^4d^2\\
&&+\ 
                 4ld^3 + 8l^2d^3 + 4l^3d^3 + d^4 + 4ld^4\\
&&+\ 
                6l^2d^4 + 4l^3d^4 + l^4d^4)\\
&=& 4(1 + d)^2(1 + l)^2(-d + l)(-1 + dl).
\end{eqnarray*}

This implies that $d=l$ and hence $D=L$, or we  obtain one
of the following solutions which contradicts the
conditions  ${\rm Re}(L)>0,\ {\rm Re}(D)>0.$ 
The solution $d=-1$ implies $D=\ln(d)=\ln(-1)=\pi i.$ 
The solution $l=-1$ implies $L=\pi i.$ 
The solution $dl=1$ implies that $D = -L$. \enddemo

\nonumproclaim{Lemma 3.9}  If $(L,D,R) \in \cal R \cap {\cal T},$ then $d = \exp(D)$ is a root of the polynomial
$$1+2d+6d^2+2d^3+d^4.$$
\endproclaim

\demo{Proof}  The equation $Y_{12}=0$ yields
\begin{eqnarray*}
0&=& 1 + 4d + 6d^2 + 4d^3 + d^4 + 4dl + 8d^2l + 4d^3l - 2l^2 \\
&&-\ 12d^2l^2 - 2d^4l^2 + 4dl^3 + 8d^2l^3 + 4d^3l^3 + l^4 +
4dl^4\\
&&+\  6d^2l^4 + 4d^3l^4 + d^4l^4.\end{eqnarray*}

Setting $l=d$ we obtain
\begin{eqnarray*}
0&=&
1 + 4d + 8d^2 + 12d^3 - 2d^4 + 12d^5 + 8d^6 + 4d^7 + d^8\\
&=&
(1 + 2d - 2d^2 + 2d^3 + d^4)(1 + 2d + 6d^2 + 2d^3 + d^4).\end{eqnarray*}

On the other hand, setting $l=d$ in the equation $Y_{11}=1$ we obtain
$$32d^5=(1+d)(1 + 4d^2 - 12d^3 + 6d^4 + 8d^5 + 4d^6 + 4d^7 + d^8)$$
which can be rewritten as 
$$0=(-1 + d)(1 + 2d + 6d^2 + 2d^3 + d^4)(-1 + 4d^3 + d^4).$$

The only solutions to these equations 
with $(\ln(d),\ln(d),0) \in {\cal R} \cap {\cal T}$ 
are the roots of the equation
$$0=(1 + 2d + 6d^2 + 2d^3 + d^4) .\leqno{(3.4)}
$$

The equation resulting from $Y_{11} = -1$ would be 
$$-32d^5=(1+d)(1 + 4d^2 - 12d^3 + 6d^4 + 8d^5 + 4d^6 + 4d^7 + d^8).$$
But 
it is easy enough to check that there are no solutions $d$ with 
$(\ln(d),\ln(d),0) \in {\cal R} \cap {\cal T}.$
\enddemo

{\it Remarks} 3.10.  i) From relator i) of Lemma 3.5 
and from $R=0$ 
we deduced that $Y=I$ and that $l$ and $d$ can be switched in $Y.$
Relator ii)  was used in the proof that $R=0.$
\vglue4pt
ii) Our matrix representatives given in the proof of Lemma 3.8 define a
lift to ${\rm SL}(2,{\bf C})$ of our representation of $\pi_1({\rm Vol3})$ 
into ${\rm PSL}(2,{\bf C})$.  Indeed
we showed above that the relator i) corresponds to the identity in 
${\rm SL}(2,{\bf C})$
and having solved $d=l=\omega$ and $R=0,$ a direct calculation shows that
relator ii) also corresponds to the identity in ${\rm SL}(2,{\bf C})$.

\nonumproclaim{Lemma 3.11}  The roots of $1+2d+6d^2+2d^3+d^4$ are
\begin{eqnarray*}
&&\left(-1 - i\sqrt3\right)/2 - \left(-6 + 2i\sqrt3\right)^{(1/2)}/2,\\
&& \left(-1 - i\sqrt3\right)/2 + \left(-6 + 2i\sqrt3\right)^{(1/2)}/2,\\
&&\left(-1 + i\sqrt3\right)/2 - \left(-6 - 2i\sqrt3\right)^{(1/2)}/2,\\
&& \left(-1 + i\sqrt3\right)/2 + \left(-6 - 2i\sqrt3\right)^{(1/2)}/2. 
\end{eqnarray*}
 \endproclaim

{\it Remarks} 3.12.  i)  If $x$ is a root of $1+2d+6d^2+2d^3+d^4,\ $ then so are
$\bar x,\  1/x\ $ and $1/\bar x.$
\vglue4pt
ii)  $-6 + 2i \sqrt{3} = 4 \sqrt{3} \exp(5\pi i/6)$ and hence
\begin{eqnarray*}
\pm \sqrt{-6 + 2i \sqrt{3}} &=& \pm 2 \root 4 \of {3} \exp(5\pi i/12)\\
&=& \pm \left(2 \root 4 \of {3}\right) \left(\sqrt{2}/4\right) \left(\left(\sqrt{3} - 1\right) + i \left(\sqrt{3} +
1\right)\right).
\end{eqnarray*}

\nonumproclaim{Lemma 3.13} If $(L,D,R) \in \cal R \cap {\cal T},$ then 
\begin{eqnarray*}
D=L&=&\ln\left(
\left(-1 - i\sqrt3\right)/2 - \left(-6 + 2i\sqrt3\right)^{(1/2)}/2\right)\\
&=&\omega
\approx 0.83144294552931 - 1.945530759503636i.
\end{eqnarray*}
\endproclaim

\demo{Proof}  The other three solutions to equation (3.4), $-\omega, \ \bar \omega, \ {\rm and} -\bar \omega,\ $
all lie 
 outside of ${\cal R}.$  The solution $\omega$ lies in ${\cal R}$.\enddemo

{\it Remark} 3.14. In our language, Hodgson and Weeks (see Remark 3.3)  knew that $\pi_1$(Vol3)
was generated by $f,w$ with $D=L=\omega$  and $R=0.$  

\demo{{P}roof of Proposition {\rm 3.1}}  
The previous lemmas established the first sentence of Proposition 3.1.
This, together with Remark 3.4, establishes the second sentence
of Proposition 3.1. The third sentence of Proposition 3.1 is 
established by the following argument.
By Proposition 2.8, if
maxcorona$(\delta)\ge 2\pi/3$, then the parameter for a 2-generator subgroup $G$ of $\pi_1(N)$ lies in
$X_0,$ and, using Proposition 1.28, we can further assert that 
this parameter lies in ${\cal S} \cap X_0.$  
Because $G= \pi_1$(Vol3), $N$ is covered by Vol3.  
By Jones-Reid (see [JR]) Vol3
only covers Vol3 (compare with Remark 3.20).  
Therefore $N = {\rm Vol3}.$\hfill\qed
\enddemo

\nonumproclaim{Corollary 3.15} {\rm Vol3} is the unique hyperbolic $3$\/{\rm -}\/manifold with
associated parameter values in  ${\cal S} \cap X_0.$\hfill\qed
\endproclaim

{\it Definition} 3.16. Let $\delta$ be a geodesic in the hyperbolic 3-manifold $N$
and $\{\delta_i\}_{i\ge 0}$ be the preimages of $\delta$ in ${\bf H^3}$.  Define an
equivalence relation on $\{\delta_i\}_{i\ge 1}$ by saying that $\delta_i$ and $\delta_j$ are equivalent if
there exists a $g\in\pi_1(N)$ taking $\{\delta_0,\delta_i\}$ to
$\{\delta_0,\delta_j\}$
(not necessarily preserving order).

Call an equivalence class an {\it orthoclass}.   Counting
with multiplicity,  consider the collection
$\cal O$ of complex numbers 
$\{\distance(\delta_0, \delta_i)\mid i>0\ $ and only one
$\delta_i\ $ is represented in each equivalence class $\}.$  
Now order $\cal O$ to obtain
the {\it ortholength spectrum of} $\delta, \ \ \{O(1), O(2), \cdots\}$ where $i\le j$ implies
${\rm Re}(O(i))\le {\rm Re}(O(j)).$
Let ${\cal O}(i)$ denote the equivalence class corresponding to $O(i).$

The {\it based ortholength spectrum of} $\delta$
consists of all distinct pairs of complex
numbers of the form $\{(\distance(B_{(-1;1)}, $ ortholine from $ B_{(0;\infty)}$ to
$g(B_{(0;\infty)}))$, $\distance(B_{(0;\infty)},g(B_{(0;\infty)})))\mid g\in \pi_1(N), g \ne f^n\}$.
Up to isometry of ${\bf H}^3,$ the
based ortholength spectrum gives the complete description of how $\{\delta_i\}$ is embedded in ${\bf H}^3.$

\demo{Proof of Proposition {\rm 3.2}} 
Let $\delta$ be a shortest geodesic in Vol3. Let
$\{\delta_i\}$ denote the preimages of $\delta$ in ${\bf H}^3$ with $\delta_0$ identified
with $B_{(0;\infty)}$  and $\delta_1$, a nearest translate, identified with $w(\delta_0).$ By
the proofs of Lemma 1.13 and Proposition 3.1 we can assume that the
associated parameters satisfy $L=D=\omega$ and $R=0.$  Define a
$(\pi_1({\rm Vol3}),{\partial \delta_i})$ 
insulator family $\{\lambda_{ij}\}$ which is a hybrid of
Dirichlet and corona insulators as follows.  
If $\delta_i\in {\cal O}(1),$ then
let $\lambda_{0i}$ be the Dirichlet insulator between $\delta_0$ and $\delta_i$
(see Definition 2.4).  Also let $\lambda_{0i}$ be the Dirichlet insulator between $\delta_0$ and $\delta_i$
if $\delta_i\in {\cal O}(k)$ and the 
visual angle at $\delta_0$ subtended by this Dirichlet insulator is
less than 113.16 degrees.
 (By [G], it is less than 113.16 when Re($O(k))>2 {\rm Arccosh}(1/\sin(113.16/2)$).)
Otherwise let  $\lambda_{0i}$ be the {\it corona} insulator 
$\kappa_{0i},$ as constructed in Proposition 2.3;  
in Lemma 3.17 we will show that also in this case ${\rm visualangle}_{\delta_0}(\lambda_{0i}) < 113.16.$
Now, extend equivariantly to
obtain the family $\{\lambda_{ij}\}$ which by construction satisfies conditions
A.1 of the appendix.   
To complete the proof of
Proposition 3.2 we now show that the insulator family $\{\lambda_{ij}\}$ is noncoalescable.

We begin by analyzing ${\cal O}(1)$ insulators.
   The geodesic plane $E^*$ midway between $ B_{(0;\infty)}$ and
$w^{-1}( B_{(0;\infty)})$ intersects the $ B_{(0;\infty)}$-$ B_{(-1;1)}$ plane in a geodesic $E$ at distance  ${\rm Re}(\omega)/2 = {\rm Relength}(f)/2$
from $B_{(0;\infty)}.$  Conjugate $\pi_1(N)$ via a rotation and homothety fixing $B_{(0;\infty)}$ 
 so that one endpoint of $E$ is at (1,0) and the other endpoint is at $(x,0),$
where $x >1.$ Because ${\rm Redistance}(E, B_{(0;\infty)})={\rm Re}(\omega)/2,$ formula (7.23.1) of [Bea]
implies $x < 23.815.$ Thus the Dirichlet insulator
$\lambda = \partial \bar {E^*}$
 (resp.\ $w(\lambda)$) between $ B_{(0;\infty)},\ w^{-1}( B_{(0;\infty)})\ $ (resp.\ $ B_{(0;\infty)}$, 
$w(B_{(0;\infty)}))\ $ is symmetric about the $x$-axis and lies within the circle passing through (1,0),
(23.815,0)  (resp.\ (-1,0), (-23.815, 0)).  
By  Lemma 4.7 of [G] this circle takes up a
visual angle of less than 133.68 degrees.  
Now $f$ is the composition of an
$\exp({\rm Re}(\omega))$ homothety centered about the origin and an ${\rm Im}(\omega)$ radian $\approx
-111.4707$ degree rotation. Because $\exp(4 {\rm Re}(\omega)) > 27.82 > 23.815$ it follows that the plane $E^*$ is taken ``beyond" $E^*$ by $w^4.$  Thus,
 $\lambda\cap (f^n(\lambda)\cup f^nw(\lambda))=\emptyset$ if $|n| \ge 4.$
Therefore if there was a
tri-linking among three insulators associated to ${\cal O}(1),$
the orthoclass of $w(B_{(0;\infty)}),$ there would be a tri-linking involving three circles from the collection 
$\{f^nw(\lambda),  f^n(\lambda)\mid -3\le n\le 0\},$ one of which is $\lambda.$

Because the absolute value of the rotational effect of $f^{\pm 1}$ is $|{\rm Im}(\omega)|$ radians, which is less than 111.48 degrees,  and because 
$\lambda$ takes up less than 133.68 degrees $ B_{(0;\infty)}$-visual
angle, 
it follows that
$f^{-1}(\lambda)\cup\lambda$ takes up less than $133.68+111.48=245.06$
degrees $ B_{(0;\infty)}$-visual angle.  Similar arguments show that if $\lambda$ and one of $f^n(\lambda)$ or $f^mw(\lambda)$ 
nontrivially intersect, then the
union cannot take up more $ B_{(0;\infty)}$-visual angle, in fact except for
$f^\pm(\lambda)$, it takes up less. See Figure 3.4 which shows the union of
$\{f^nw(\lambda),  f^n(\lambda)\mid -3\le n\le 0\}.$   Therefore, if the union of
three such circles was connected,  they would take up at most
$133.68+2(111.48)=356.64$ degrees of $ B_{(0;\infty)}$-visual angle and hence would not create a
tri-linking. Thus, ${\cal O}(1)$ cannot by itself create a tri-linking.
\figin{fig3.4n}{595}
\centerline{Figure 3.4. There is no tri-linking from ${\cal O}(1)$.}
\eject

Now assume Lemma 3.17.  There remain
three cases to consider: tri-linking involving no ${\cal O}(1)$ insulators; tri-linking involving exactly one ${\cal O} (1)$ insulator; 
tri-linking involving exactly two ${\cal O} (1)$ insulators.  The case of no ${\cal O} (1)$ insulators cannot occur because  $3(113.16) < 360$.  The case of exactly one ${\cal O} (1)$ insulator cannot occur because
the union of
three such insulators will take up less than $133.68 + 2(113.16)=360$
degrees of visual angle. 
Finally, the case of exactly two ${\cal O} (1)$ insulators cannot occur, because the two ${\cal O} (1)$ insulators would take up at most 245.06 degrees of visual
angle;
 so all three insulators would take up at most 
$245.06 + 113.16$ degrees of visual angle.
\enddemo

\nonumproclaim{Lemma 3.17} If $\delta_i \in {\cal O}(k),\ k>1,$ then ${\rm
visualangle}_{\delta_0}(\lambda_{0i}) < 113.16.$
\endproclaim

\demo{{P}roof} 
One might be tempted
to use a version of {\it SnapPea} to prove
this lemma by enumerating possibilities, and in fact we 
describe this approach later in this section.
However, to provide the necessary rigor, we found it 
convenient to use instead a modified version of the parameter space 
argument we have used twice before. 
The proof is in two steps.  The first step is to show that
Re$(O(2))>{\rm Re}(O(1))$, and the second step is
the parameter space argument.
\vglue8pt
{\it Step} 1.  Re$(O(2))>{\rm Re}(O(1))$.
\vglue8pt

{\it Proof of Step} 1.  
If ${\rm Re}(O(2)) = {\rm Re}(O(1)),$ then 
${\cal T}$ contains a parameter with $L = \omega$
and $D = O(2).$  By Proposition 1.28 this can only happen if the parameter
lies in $X_0$ and by Proposition 3.1 this can only happen if 
$O(2) = \omega = O(1).$

If $v\in
\pi_1$({\rm Vol3}) is an element with distance$(v(B_{(0;\infty)}), B_{(0;\infty)})=O(2) = \omega$ as above, then the group
generated by $v,f$ is conjugate (by an orientation-preserving isometry taking $ B_{(0;\infty)}$ to itself) to the group generated by $w,f.$
This implies that the $\pi_1({\rm Vol3})$ translates of $B_{(0;\infty)}$ are symmetrically placed about $B_{(0;\infty)}$.  
We now restrict our focus to 
$\delta_i\notin {\cal O}(1)$  with distance($\delta_i, B_{(0;\infty)})=\omega,$ and define $K$ to be the 
unoriented orthocurve between $v(B_{(0;\infty)})$ and $B_{(0;\infty)}.$ 
There are two cases to consider.

The first case is that $K$ hits $B_{(0;\infty)}$ at $(0,0,1).$ 
Now (after possibly rechoosing $\delta_i,$
and using the fact
that $R = 0$) we can assume that the angle between $K$ and
$K^\prime$ is $\pi/m,$ where $K^\prime$ is the orthocurve between $B_{(0;\infty)}$ and $w(B_{(0;\infty)})$ and
$m>1$ is a maximal integer.  
Using the hyperbolic law of cosines ([F, p.~83]), 
it follows that if $m=2,$ then there exists an $\bar \omega$
ortholength.  As in the first paragraph of this proof, we obtain a contradiction. See
Figure 3.5. If $m>2,$ then the cosine law shows
that there exists a real ortholength less than Re$(\omega).$


The second case is that
Redistance($K, K^\prime) = x > 0$
where $x$ is minimal among all possible choices. 
Then by symmetry we can
assume that there is an $m$ with $mx={\rm Re}(\omega)={\rm Re}(L).$  Indeed $\delta_i$ can be
chosen so that if $\delta_i$ is obtained from $w(B_{(0;\infty)})$ by an $x+iy$ translation
$\tau$ along $ B_{(0;\infty)},$ then $m(x+iy)=\omega.$  (This uses the fact that the $R$
associated to any orthoclass ${\cal O}(i)$ with $O(i)=\omega$ is equal to $0.$)
Therefore  an $x+iy$ 
translation $\tau$ along $ B_{(0;\infty)}$ descends to a free ${\bf Z}/m{\bf Z}$ action $\phi$ on Vol3.  This is free, because any
lift of $\phi^n$ is a conjugate (by an orientation-preserving isometry) of $\tau^n$ which is fixed-point free or the identity.
This  contradicts the fact that Vol3 only covers Vol3.
\enddemo

\figin{fig3.5}{700}
\begin{quote} Figure 3.5. $\cosh(P) = 
\cosh^2(\omega) + \sinh^2(\omega) \cosh(i (\pi/2 + \pi))$ has solution $P = \bar \omega.$
\end{quote}


{\it Step} 2. The parameter space argument.
 

\demo{Proof of Step {\rm 2}} Our parameter space ${\cal W}$ is the usual initial box.   As before, a parameter
gives rise to a marked group $\{G,f,v\}$.  (Here, $f$ is
the standard generator
of $\pi_1$(Vol3) fixing $ B_{(0;\infty)}$.)  We consider the 
set $U$ of parameters such that
$f, v$ generate a torsion-free, parabolic-free group $G$ where $f$ is of minimal length, 
${\rm Re}(D) > {\rm Re}(\omega),$ and
${\rm corona}(D) \ge 113.16.$
(It is easily checked that $U \subset {\cal W},$ and it is
interesting to note that $U$ contains parameters with ${\rm Re}(D)$ 
almost 1.24.)
We will partition the initial box ${\cal W}$ into sub-boxes ${\cal W}_i$ 
such that each ${\cal W}_i$ can be eliminated for one of
the following reasons.
\begin{itemize}
\item[y)]  ${\cal W}_i$ is outside of $U$.

\item[z)]  ${\cal W}_i = X_0.$

\item[a)]  There exists no $\beta\in {\cal W}_i$ such that 
$\exp(\length(f_\beta)) \in {\cal L}(X_0).$  
Hence, $\length(f_\beta) \neq \omega$ throughout ${\cal W}_i.$

\item[b)]       ${\cal W}_i$ has some $L'=\exp(L)$ values in  ${\cal L}(X_0)$ but
there exists no $\beta\in {\cal W}_i$ such that the real part of 
${\rm distance}(v_\beta(B_{(0;\infty)}), B_{(0;\infty)})$ is greater than the minimum $d$ value for $X_0.$
In particular,  ${\rm Re(distance}(v_\beta(B_{(0;\infty)}), B_{(0;\infty)})) \le {\rm Re}(\omega)$ throughout ${\cal W}_i.$

\item[c)]  There exists a killerword $h$ in $f,v,f^{-1},v^{-1},$ such that
$h_\beta \neq {\rm id}$ and
${\rm Relength}(h_\beta) < {\rm Relength}(f_\beta)$  
for all $\beta\in {\cal W}_i.$
\end{itemize}
 
There are files containing the partition of ${\cal W}$ and the associated conditions/killerwords, and the program {\it corona} verifies that they
indeed work.  As noted earlier, {\it corona} works with the exponentiated versions of the above conditions.
The computer methods (involving the corona function) 
needed to complete the proofs
in Sections 2 and 3 are sufficiently similar that it was 
natural to incorporate both proofs into one partitioning of ${\cal W}$,
one list of associated conditions/killerwords, and one computer
program ({\it corona}).

Lemma 3.17 follows from the fact that each 
sub-box can be eliminated.  Indeed
if $\delta_i \in {\cal O}(k),\ k>1,$ then  Step 1 implies that
${\rm Re}(O(k)) > {\rm Re}(O(1)) = {\rm Re}(\omega).$  
If $\lambda_{0i}$ is a Dirichlet insulator, then by
construction either $\delta_i \in {\cal O}(1)$ or 
${\rm visualangle}_{\delta_0}(\lambda_{0i}) < 113.16.$  
If $\lambda_{0i}$ is a corona
insulator and ${\rm visualangle}_{\delta_0}(\lambda_{0i})\ge 113.16,$ 
then the group generated by $f$ and $v$ gives rise to a parameter in $U$  
where $v\in
\pi_1({\rm Vol3})$ is an element taking $\delta_0$ to $\delta_i.$
The above program rules out this possibility.
\enddemo
 

Here is an experimental ``proof" of Lemma 3.17.  In [HW] an algorithm is given to
compute, with multiplicities,  the length spectrum of a hyperbolic 3-manifold $M,$
given a Dirichlet domain for $M.$   Weeks has observed that a very similar argument gives an algorithm to compute the
based ortholength spectrum.  In fact an analogue to Proposition 1.6.2 in [HW] (with an analogous proof) is the
following.

\nonumproclaim{Lemma 3.18 {\rm (Weeks)}} Let $M$ be a closed orientable $3$\/{\rm -}\/manifold having a Dirichlet
domain $\cal D$ with basepoint $x$ and with spine radius $r.$  
Let $\delta$ be a geodesic of length $l+it.$  
To compute all the based ortholengths of real length less than or equal to
$\lambda$
with basing less than or equal to $l/2$ from some point on $\delta_0$ {\rm (}\/a preimage of $\delta${\rm )} it
suffices to find all translates $g\cal D$ satisfying $\rho(x,gx)\le
2r+2{\rm Arccosh}(\cosh(l/2)\cosh(\lambda/2)).$
\endproclaim

\demo{Proof}  As in [HW] we can assume that $\rho(\delta_0,x)\le r.$   The 0-basing on
$\delta_0$ will be given by the oriented perpendicular $P$ from $\delta_0$ to $x.$
Figure 3.6 shows that if there is a translate $\delta_i=g(\delta_0)$ based at
distance $ \le l/2,$ at real distance $\le \lambda$ from $\delta_0,$ then
$\rho(g(x),x)\le \lambda +l+2r$.  As in [HW] an application of the hyperbolic
cosine law yields the better estimate of Lemma 3.18.    \enddemo

Provided one is given a Dirichlet domain, [HW] gives an efficient algorithm to find these $g$'s.   Finally to each such $g$ one
computes the basing and distance from $g(\delta_0)\  {\rm to}\  \delta_0.$

The collection of ortholengths with basings $\le l/2$ contains at least two
representatives for each ortholength class;  thus the Weeks algorithm can
be used
to give lower bounds on the various $O(i)$'s.  {\it SnapPea} (see [W]) computes a Dirichlet domain
for Vol3 with spine radius $\le 0.68.$  
Because $l < 0.83145,$ taking $\lambda=1.24$ we
obtain $2r+2{\rm Arccosh}(\cosh(l/2)\cosh(\lambda/2)) < 2.89.$  This algorithm was
implemented on an undistributed version of {\it SnapPea}, and provided the following estimates.  
Note that ${\rm Re}(O(2)) \ge 1.24$ is sufficient to guarantee that ${\cal C}(O(2)) < 113.16$ degrees.
\begin{eqnarray*}
O(1)&\approx& .83144 - 1.94553 i,\\
O(2)&\approx& 1.3170 - \pi i, \\
O(3)=O(4) &\approx& 1.4197 + 1.0963 i,\\
O(5) &\approx& 1.9769 -1.2995 i. \end{eqnarray*}
These
estimates were found using a ``tiling" of radius $3.00 > 2.89,$ which is
sufficient for ``proving" that for Vol3, ${\rm Re}(O(2)) > 1.24$.\hfill\qed

\figin{fig3.6}{800}
\centerline{Figure 3.6. Controlling the based ortholength spectrum.}
\vglue12pt


\demo{{R}emark {\rm 3.19}} This experimental  proof should be easy to make rigorous by
implementing Weeks's algorithm using exact arithmetic, say via 
the program {\it Snap} (see [CGHN]).
\enddemo

{\it Remark} 3.20.  Similar technology can probably be used to show that Vol3 only covers Vol3.  Because Vol3 is
non-Haken, and nonorientable 3-manifolds are Haken, 
Vol3 can only cover an orientable 3-manifold.
By Proposition 1.28 and the first sentence of Proposition 3.1, it follows that either a shortest geodesic $\delta$ in $M$ has ${\rm tuberadius}(\delta) 
= {\rm Re}(\omega)/2$ and ${\rm length}(\delta) = \omega,$ or $\delta$ has a $\ln(3)/2$ tube.  The latter implies that some geodesic $\beta$ in Vol3 has a $\ln(3)/2$ tube.  Elementary volume considerations together with length spectrum data provided by {\it SnapPea} imply that ${\rm length}(\beta) < 1$ and hence is one of
eight geodesics.  Tuberadius data provided by an unreleased version of {\it SnapPea} assert that such geodesics
have  tuberadius smaller than 0.43.   Thus, the length of $\delta$ is $\omega$ and so 
${\rm volume}(M) > \pi \sinh({\rm Re}(\omega)/2))^2 {\rm Re}(\omega) > 
0.476 > {\rm volume}({\rm Vol3})/3.$  This implies that there exists a free, orientation-preserving involution $\tau$ of Vol3 such that $\tau(\delta) \ne \delta.$  By  length spectrum data from {\it SnapPea}, there are only two distinct geodesics with the same length as $\delta.$  The above analysis shows that these correspond to $ B_{(0;\infty)}$ and $ B_{(-1;1)}$, which nontrivially intersect.  Thus, no such involution exists.  Using the exact arithmetic of {\it Snap} (see [CGHN]), one should be able to make rigorous these {\it SnapPea} calculations.
 



