 
\def\Relength{{\rm Relength}}
\def\length{{\rm length}}
\def\Arccosh{{\rm Arccosh}}
\def\trace{{\rm trace}}
\def\distance{{\rm distance}}
\def\Redistance{{\rm Redistance}}
\def\va{{\rm visualangle}}

\vglue-12pt
\section{The corona insulator family}
\vglue-4pt
 
The upshot of Proposition 1.28 is that if a closed orientable hyperbolic
3-manifold has a shortest geodesic which does not have an embedded
$\ln(3)/2$ tube then the parameters for its associated marked group(s)
$(G,f,w)$  must be in one of the exceptional boxes 
$X_0, X_1, \ldots, X_6$ listed in that proposition.
Nonetheless---as we shall see in this section and the next---such manifolds
have noncoalescable insulator families about their shortest geodesics,
although they might not be Dirichlet insulator families.  A review of the
theory of insulators in hyperbolic 3-manifolds is provided in the
appendix.

In this section, we describe a new insulator  family $\{\kappa_{ij}\}$
called the corona insulator, and
we describe a condition sufficient for this family to be
non\-coalescable---a condition which is weaker than the
${\rm tuberadius}(\delta) > \ln(3)/2$ sufficient condition for the
Dirichlet insulator family.

The reason Dirichlet insulator families for geodesics with solid tubes of
radius greater than $\ln(3)/2$ are noncoalescable is that the amount of
visual angle taken up by the various insulators is less than 120 degrees,
and thus there is no chance for tri-linking to occur.   The visual angle
(measured at one axis) for a member of the Dirichlet insulator family
associated to two axes depends only on the real distance between the two
axes.

In contrast, the visual angle for a member of the corona insulator family
associated to two axes depends on the complex distance between the two
axes.  We now define this function ${\cal C}$ of complex distance,
and name it the {\it corona} function.  After that we give a precise
definition of the visual angle function, and prove that the corona function
is the proper visual angle function for the corona insulator family.

\demo{Definition {\rm 2.1}} Let
${\cal C}: (0,\infty)\times [-\pi,\pi]\to (0,\pi)$ be defined by

$${\cal C}(u,v) = {\rm Abs}\left({\rm Im}
\left(\Arccosh\left(1-{4 \over 1 \pm \cosh(u+iv)} \right)\right)\right)$$
\noindent where $\pm$ is positive for $-\pi/2 \le v \le \pi/2$ and negative
otherwise.  The branch of $\Arccosh$ is chosen so that the values of the
corona function lie between $0$ and $\pi.$\enddemo

\figin{fig2.1}{600}
 
\begin{quote} Figure 2.1. The  102 to 120 degree contours for the corona function ${\cal C}(u,v)$ (for example, the 120 degree contour corresponds
to
where the corona function takes on value $2\pi/3$).
\end{quote}

 
In the following definition, it is helpful to imagine the geodesic $\sigma$
as being the $z$-axis in the upper-half-space model of ${\bf H}^3.$

\demo{Definition {\rm 2.2}}  If $\sigma\subset {\bf H}^3$ is a
geodesic, then $S^2_\infty - \partial\sigma$ can be
parametrized by $S^1\times {\bf R},$  where {\bf R} represents the set of
real numbers.
Each $x\times {\bf R}$ lies in the ideal boundary of
a hyperbolic half-plane bounded by $\sigma.$
Two such lines $x\times {\bf R},\ y\times {\bf R}$
are at distance $\theta$ in the $S^1$ factor if they meet at
$\partial \sigma$ at angle $\theta.$
Consider a region $R$ in $S^2_\infty-\partial \sigma,$ and define
$\va_\sigma(R)=\theta\in [0,2\pi]$ as follows.
Let $\theta$ be the infimum of lengths of closed subintervals $J$ of $S^1$
such that $R \subset J \times {\bf R},$ if there is such a subinterval, and
let $\theta = 2\pi$ if there is no such subinterval.
\enddemo

\nonumproclaim{Proposition 2.3}  Let $\delta_i, \delta_j$ be
disjoint oriented geodesics in ${\bf H}^3$.
Then there exists a smooth simple closed curve $\kappa_{ij}$ in $S^2_{\infty}$
separating $\partial\delta_i$ from $\partial\delta_j$ such that for $k\in
\{i,j\},\
\va_{\delta_k}(\kappa_{ij})={\cal C}(\distance(\delta_i, \delta_j)).$
\endproclaim

\demo{Proof}
Let $P$ be the orthocurve from  $\delta_i$ to  $\delta_j.$
Consider the half-plane with boundary $\delta_i$ containing $P,$ and the
half-plane with boundary $\delta_j$ containing $P.$   Allow these
half-planes to expand into wedges at the same rate. (A {\it wedge} is a
closed set in $B^3 = {\bf H}^3 \cup S^2_\infty$ bounded by two hyperbolic
half-planes which meet along a common geodesic.)  At first, the four
half-planes that bound these wedges intersect in ${\bf H}^3,$ but at some
angle $\theta$ these half-planes intersect only at infinity (that is, in
$S_\infty^2$). By reasons of symmetry they intersect in two or four points
(four points of intersection occur when ${\rm Im}(\distance(\delta_i,
\delta_j))$ is  $\pi/2$ or $-\pi/2$).

Let $A_i, A_j$ be the wedges which exist at angle $\theta.$ Let
$T_k=A_k\cap S^2_\infty$ for $k\in \{i,j\}.$  Let $\kappa_{ij}$ be a simple
closed curve in
$T_i\cap T_j$ which separates $\partial \delta_i$ from $\partial \delta_j.$  By
construction, for $k\in \{i,j\},\ \va_{\delta_k}(\kappa_{ij})=\theta,$
and $\theta$ is in $[0,\pi].$
See
Figure 2.2 (and compare with Figure 2.3).
\figin{fig2.2}{550}
\begin{quote} Figure 2.2.  Here dist$(\delta_i,\delta_j)$ is approximately $.9+(2\pi/5) i$ and\break $C({\rm
dist}(\delta_i,\delta_j))$ is approximately $2\pi/3$.  Note that the lightly shaded region is $T_i\cap T_j$ while the darkly
shaded region is
$S^2_\infty - T_j.$
\end{quote}


\noindent 
\figin{fig2.3}{625}
 \centerline{Figure 2.3. Two versions of the degenerate right-angled hexagon.}
\vglue6pt
To complete the proof of the proposition, we now show that $\theta =\break{\cal
C}(\distance(\delta_i,\delta_j)).$
To do this, we use hyperbolic trigonometry on a degenerate right-angled
hexagon in ${\bf H}^3.$  
 By [F], a degenerate right-angled
hexagon is a 5-tuple of oriented geodesics $S_1,\cdots, S_5$ in ${\bf H}^3$
such that
$S_i$ is orthogonal to $S_{i+1}$ and $S_1$ and $S_5$ limit at a common
point $S_0$ at
infinity.   These oriented geodesics give rise to complex numbers
$\sigma_0, \sigma_2,\sigma_3, \sigma_4$ representing signed edge lengths.
For $k\in \{2,3,4\}$, $\sigma_k= d_{S_k}(S_{k-1},S_{k+1})$  where a
$d_{S_k}(S_{k-1},S_{k+1})$ translation of ${\bf H}^3$ along the oriented
geodesic $S_k$
takes the oriented geodesic $S_{k-1}$ to the oriented geodesic $S_{k+1}$
(see Definition 1.4).  The remaining (degenerate) edge length is given by
$\sigma_0 = 0$ if the axes $S_1$ and $S_5$ either both point
 into $S_0$ or both point out of $S_0;$  otherwise $\sigma_0=\pi i.$
By [F; pg. 83] we have the following hyperbolic law of cosines.

\centerline{$\cosh(\sigma_0) = \cosh(\sigma_{2}) \cosh(\sigma_{4})
+ \sinh(\sigma_{2}) \sinh(\sigma_{4}) \cosh(\sigma_{3}).$}
\vglue6pt



We work in the upper-half-space model of hyperbolic 3-space, and normalize
so that the ortholine from $\delta_j$ to $\delta_i$ is $B_{(0;\infty)}$ 
(thus $\delta_i$ intersects $ B_{(0;\infty)}$ above $\delta_j$), while the
oriented axis $\delta_i$ is $ B_{(-1;1)}.$ Now,  $ B_{(0;\infty)}$ will be $S_3$
while the oriented geodesics $\delta_i$ and $\delta_j$ will be $S_2$ and  
$S_4$, respectively.



Let, $u+iv = \distance(\delta_i,\delta_j).$
  If $-\pi/2 < v < 0$ then the intersection points at infinity occur in the
second quadrant and the fourth quadrant.  See Figure 2.3a.    For
convenience, we work with the point in the second quadrant and send
(unique) perpendiculars from it to the geodesics $\delta_i$ and $\delta_j$.
The perpendicular to $\delta_i$ will be oriented towards $\delta_i$ and
then denoted $S_1$, while the perpendicular to $\delta_j$ will be oriented
away from $\delta_j$ and then denoted $S_5.$  The  intersection point at
infinity (in the second quadrant) is  $S_0.$

 
 


This is the proper set-up for applying the (degenerate) hyperbolic law of
cosines (see Figure 2.3b).  Note that $\sigma_3 = -(u+iv),$ and $\sigma_0 =
i \pi.$ By symmetry $\sigma_2 = \sigma_4 = (\alpha + i \beta)/2$ where
$(\alpha + i \beta)/2$ is $\distance(S_1, S_3).$  Plugging into the law of
cosines, using a half-angle formula ($\cosh(2z) = 2 \cosh^2(z) - 1 = 2
\sinh^2(z) + 1$), solving for $\cosh(\alpha + i \beta),$ and taking the
\Arccosh, we get the desired result.  Note that the visual angle in this
set-up is $-\beta,$ thus necessitating taking the absolute value.




When $0 < v < \pi/2$ our two intersection points occur in the first and third
quadrants, and we carry out the same procedure.  This
time $S_2$ and $S_4$ are traversed in the direction opposite to their
orientations (the attendant changes in sign drop out though). In this case,
the visual angle is $\beta.$

The special cases $v = -\pi/2,\ 0,\ \pi/2$ follow similarly.

The cases $-\pi \le v \le -\pi/2$ and $\pi/2 \le v \le \pi$ reduce to the
previous cases after adding or subtracting $\pi.$  The formula in
Definition 2.1 is then obtained after we note that $\cosh(z \pm i\pi) = -
\cosh(z)$. \enddemo

{\it Definition} 2.4.  Let $\delta$ be a simple closed
geodesic in the closed
orientable hyperbolic 3-manifold $N.$  Let $\{\delta_i\}_{i\ge 0}$ be the
lifts of
$\delta$ to ${\bf H}^3.$  For each $\pi_1(N)$-orbit of unordered pairs
$(\delta_i,
\delta_j)$ choose a representative where $i=0.$
If\break $\Redistance(\delta_0, \delta_i)\le \ln(3),$ then let
$\kappa_{0j}$ be a smooth simple closed
curve in $S^2$ separating $\partial \delta_0$ from $\partial \delta_j$ such
that for
$k\in \{0,j\},\ \  \va_{\delta_k}(\kappa_{0j})={\cal C}
(\distance(\delta_0, \delta_j)).$
 If
$\Redistance(\delta_0,\delta_i) > \ln(3),$ then let $\kappa_{0j}$ be the
Dirichlet
insulator, i.e. the boundary of the geodesic plane orthogonally bisecting
the orthocurve between $\delta_0, \delta_j.$

In either case, extend the collection $\pi_1(N)$-equivariantly to a family
$\{\kappa_{ij}\}$ defined for all
$i,j.$  This is the {\it corona family} for $\delta.$

\nonumproclaim{Lemma 2.5} The corona family $\{\kappa_{ij}\}$ is
\vglue2pt 
{\rm i)}  an insulator family for $\delta$
\vglue2pt 
{\rm ii)}  noncoalescable if $\max\{{\cal C}(\distance (\delta_0, \delta_j))\mid
j>0\}<2\pi/3.$
\endproclaim

\demo{Proof} We check that $\{\kappa_{ij}\}$ satisfies
the various conditions of
Definitions A.1 and A.2 in the appendix.
\vglue2pt
i)  By construction, $\kappa_{ij}$ separates $\partial \delta_i$
from $\partial \delta_j$ and $\{\kappa_{ij}\}$ is $\pi_1(N)$-equivariant.
For $k\in \{i,j\},\ $
$\delta_k$-visualangle$(\kappa_{ij})<\pi,$ and so $\{\kappa_{ij}\}$
satisfies the convexity
condition.

Modulo the natural action of $\pi_1(N)$ on  $\kappa_{ij},$
there are only finitely many insulators $\kappa_{ij}$ which are not
Dirichlet insulators.  Therefore, for fixed $i,$ there exist
only finitely many $\kappa_{ij}$ such that ${\rm
diam}(\kappa_{ij})>\epsilon.$  This
establishes local finiteness. 

\vglue2pt
ii) No tri-linking follows immediately from the ``less than
 $2\pi/3$" condition.\enddemo

{\it Definition} 2.6. If $\delta$ is a simple closed
geodesic in the hyperbolic
3-manifold $N,$ define ${\rm maxcorona}(\delta)=\max\{{\cal C}(\distance
(\delta_0, \delta_j))\mid j>0\}$.

\vglue4pt {\it Remarks {\rm 2.7}}. i)  It seems possible that the
Dirichlet insulator family
associated to a geodesic $\delta\in N$ may be noncoalescable, while a
corona insulator family is coalescable and conversely (corona family
noncoalescable while Dirichlet family coalescable).
\vglue2pt
ii) If ${\rm tuberadius}(\delta)>\ln(3)/2,$ then the Dirichlet insulator
and corona insulator families coincide, and
 they are noncoalescable because ${\rm tuberadius}(\delta)>\ln(3)/2$
implies the relevant visual angles are less than 120 degrees, hence
tri-linking cannot occur
(see Example A.3, or Lemma 5.9 of [G]).  Note that these facts
together with Proposition 2.3 imply that
${\cal C}(u,v)<2\pi/3$ if $u>\ln(3)$.
\vglue2pt
iii)  The corona family is not uniquely defined for there is some
choice in constructing the $\kappa_{0j}$'s.
 


\nonumproclaim{Proposition 2.8} Let $\delta$ be a shortest
geodesic in the closed orientable
hyperbolic $3$\/{\rm -}\/manifold $N.$  Then either a corona insulator family is
noncoalescable or there exists a marked group $\{G,f,w\}$ where $G$ is a
subgroup of $\pi_1(N)$
and  the parameter associated to
$\{G,f,w\}$ lies in the exceptional box  
$X_0= X_{0a} \cup X_{0b} \subset {\cal W}.$
\endproclaim

\demo{Proof}
We will show that if $\{G',f,w'\}$ is any marked group in
$\pi_1(N)$\break with $f$ minimizing length,
$w'$ maximizing ${\cal C}(\distance (h(A_f), A_f))$
where\break $h\in G-\{f^k\},$ and with ${\cal
C}(\distance (w'(A_f), A_f)) \ge 2\pi/3,$
then there is a marked group $\{G,f,w\}$ 
(where $G$ is a subgroup of $G'$
and $w$ takes $A_f$ to a nearest translate) with
associated parameter
$\beta=(L_\beta', D_\beta', R_\beta')\in X_0.$
Here, as in Definition 1.2, $A_f$ denotes the axis of $f$.

We note that the parameter associated to
a length-minimizing, corona-maximizing marked
group as in the previous paragraph might, {\it a priori}, lie outside
the exceptional boxes 
$X_k$ for $k\in \{0,1,\ldots, 6\}$ described in Proposition 1.28.
So, the computer proof that we describe below starts afresh with
the same initial box ${\cal W}$ as in Definition 1.22.
The fundamental conditions  used to eliminate
parameter points will be a ``shortest" condition and
a ``maxcorona" condition.
Although we cannot use Proposition 1.28 to directly limit the
sub-boxes under consideration, we {\it can} carefully exploit
aspects of Proposition 1.28 to develop other useful conditions
for eliminating sub-boxes, as follows.

Recall that ${\cal L}(X_k)$ denotes the range
of $L'=\exp(L)$ values in the box $X_k.$

If $\delta$ is a shortest geodesic with
$\exp(\length(\delta))$ not in ${\cal L}(X_k)$ for any $k\in
\{0,1,\ldots, 6\}$ then
Corollary 1.29 implies that ${\rm tuberadius}(\delta) > \ln(3)/2.$
As in Remark 2.7ii), it follows that
such parameter values can be eliminated.

So, we restrict to parameter values $\beta$ with
$L_\beta'\in {\cal L}(X_k)$ for some $k\in
\{0,1,2,3,4,5,6\}.$  Now, if $k=0$ --- that is, if $\beta$ is
a parameter value with $L_\beta'\in {\cal L}(X_0)$ then we
can  find an affiliated parameter
$\beta_1$ that
lies in $X_0$
by the following argument.   If $\beta$ corresponds to the
marked group $\{G^\prime,f,w^\prime\}$ as in the first
paragraph, then there exists a marked 2-generator subgroup
$\{G,f,w\}$ of
$G^\prime$ with parameter $\beta_1=(L',D',R')$ such that $L'=L_\beta'$ and
such that
$d={\rm Re}(D)$ is minimal (recall that $\exp(D)=D'$),
that is,  $d$ is twice the tuberadius of the geodesic corresponding
to $f$. The proof of
Proposition 1.28 now implies that
$(L',D',R')\in X_0$.  So, in our computer
analysis of the parameter space ${\cal W}$ we can ignore parameters
$\beta$ with $L_\beta'\in {\cal L}(X_0).$

Now, we consider
$\beta$ with $L_\beta'\in {\cal L}(X_j)$ for some
$j\in\{1,2,3,4,5,6\}.$
Then, it turns out that  $D_\beta$ is subject to useful constraint.
Specifically,  given such a $j$,
we see that such parameter values
$\beta$ with associated value of $d$ below
the minimum $d$ value for the box $X_j$
(for example, roughly 1.059 for $X_5$; see Table 1.2)
are eliminated by the following argument.

Assume
$l_\beta = {\rm Re}(L_{\beta})$ is a minimum 
(otherwise we would have a ``shortest"
contradiction). Next observe that $d_\beta$ cannot be minimal:
otherwise we obtain a contradiction to Proposition 1.28 using
the fact that we are outside all of the $X_k.$
Because $d_\beta$ is not a
minimum, there is another
parameter point ${\beta}'$ with $L_{{\beta}'}'=L_\beta',$
$d_{{\beta}'}<d_\beta,$ and $d_{{\beta}'}$ a minimum.
Again by construction, ${\beta}'$ is outside the $X_k,$
which contradicts Proposition 1.28, and
eliminates $\beta.$
Hence, in summary,
$D_\beta$ values are constrained as follows:
first, the associated $d$ value must be above
the minimum $d$ bound for that $j$ and
second, the visual angle from the corona function
must be greater than or equal to $2\pi/3.$
These two constraints in the $X_5$ case are
shown graphically in Figure 2.4: $D_\beta$ must lie in
the decorated region.

Note that we have implicitly used the fact the
${\cal L}(X_k)$'s are disjoint.


Our proof is now similar to the proof of Proposition 1.28.
We partition the
initial box ${\cal W}$ into sub-boxes ${\cal W}_i$ and eliminate ${\cal
W}_i$ if either ${\cal W}_i$ lies outside $\exp({\cal P})$ (see Remark
1.23i), Definition 1.12, and Remark 2.7ii)),
or if any of the following conditions   hold:  
\vglue-20pt
\phantom{ho}
\begin{itemize}
\item[a)]  There exists no $\beta\in {\cal W}_i$ such that
$\exp(\length(f_\beta)) \in {\cal L}(X_j)$   for
$j\in\{1,2,3,4,5,6\}.$ 

\item[b)]  ${\cal W}_i$ has some  $L'$ values in (exactly one) ${\cal L}(X_j)$ but
there exists no $\beta\in {\cal W}_i$ such that distance$(w_\beta(A_f),
A_f)\in\ $ decorated region for that $j$.  Again $j\in\{1,2,3,4,5,6\}.$

\item[c)]  There exists a killerword $h$ in $f,w,f^{-1},w^{-1}$ such that
$\Relength(h_\beta) < \Relength(f_\beta)$  and $h_\beta \neq {\rm id}$
for all $\beta\in {\cal W}_i.$

\item[d)]  There exists a killerword $h$ in $f,w,f^{-1},w^{-1}$ such that
$${\cal C}(\distance(h_\beta(A_f), A_f)) >{\cal C}(\distance(w_\beta(A_f),
A_f))$$ and $h_\beta(A_f) \neq A_f$ for all $\beta\in {\cal W}_i.$
\end{itemize}

\figin{fig2.4}{525}
\begin{quote}
Figure 2.4. The $120$-degree contour for the corona function
${\cal C}(u,v),$ and a decorated region which corresponds to corona greater
than or equal to $120$ degrees and $u$ value greater than or equal to $1.059$.
\end{quote}

We have two files that contain the decomposition of ${\cal W}$ into
sub-boxes and associated conditions/killerwords.  The program {\it corona}
checks that these files do indeed eliminate all of
${\cal W} - X_0.$  \enddemo 

{\it Remarks} 2.9.  i)  The proof of Proposition 2.8
shows that the portion of the parameter space ${\cal W}$
which is not immediately eliminated (by either the
outside-of-$\exp({\cal P})$ condition,
or by conditions a) and b))
is considerably smaller than that of Proposition 1.28. Condition a)
implies that the parameter space is ``$(2+\epsilon)$-complex dimensional" and
condition b) implies that one of the parameters is greatly constrained.   This
suggests why it took so much longer to come up with the partition and the
associated killerwords for Proposition 1.28.
In fact, it took roughly 1500 CPU days to find the partition and the
associated killerwords for Proposition 1.28, versus roughly two CPU days for
Proposition 2.8.  Here, the term ``CPU day''  refers to 24 hours of running
an SGI Indigo 2 workstation with an R4400 chip, and the estimate of 1500
CPU days refers to 15 to 20 such machines running 80 to 90 percent of the
time for three to four months.
\vglue4pt
ii) We took pains to make {\it corona} as similar to {\it verify} as we
could, thereby lessening the amount of analysis needed to show the veracity
of {\it corona}.
\vglue4pt
iii)  When working with exponentiated co-ordinates (that is, in ${\cal W}$
rather than ${\cal P}$) the corona function changes as follows.  Let $X =
\exp(\alpha + i \beta)$ and
$ U = \exp(u + iv),$  then the corona function, when $-\pi/2 \le v \le
\pi/2$ (so that the positive choice in $\pm$ is made),
$$\cosh(\alpha + i \beta) = 1 - {4 \over 1 + \cosh(u+iv)}$$
becomes
$${X + X^{-1} \over 2} = 1 - {4 \over 1 + (U + U^{-1})/2}.$$

It is a pleasant exercise to solve this, and we find that
$$X = {(U^2 -6U + 1) \pm 4(U - 1)\sqrt{-U} \over (U+1)^2}.$$
The two answers are reciprocals, which implies their associated arguments
are opposites.  We choose $+$ or $-$ so that the argument is positive.

For $-\pi \le v \le -\pi/2$ and $\pi/2 \le v \le \pi$
(so that the negative choice of $\pm$ is made)
we find (replace $U$ by $-U$ in the above formula) that
$$X = {(U^2 + 6U + 1) \pm 4(U + 1)\sqrt{U} \over (U-1)^2}.$$

In {\it corona}, the exponentiated version of the corona function is the
function {\it horizon(ortho)}, which  takes in $U=\ ${\it ortho} and
computes the associated $X$ value.  The argument of $X,$ $\beta,$  is
implicitly obtained in the function {\it larger\/{\rm -}\/angle}.
\vglue4pt
iv) It is possible that by working purely in the context of the corona function, rather than first working with $\Redistance$ and attempting to prove
Proposition 1.28, the computer proof can be simplified.
After some initial investigations using a special version of {\it SnapPea}
created for us by J. Weeks, we proceeded on  this
project with the naive idea that perhaps Vol3 was the only manifold whose
shortest geodesic did not have a $\ln(3)/2$ tube.  The remarkable fact that this
naive idea is almost correct accounts for the fact that a proof of Theorem 0.2
can be obtained with only the mild extra effort detailed in this section
and the next.
 








