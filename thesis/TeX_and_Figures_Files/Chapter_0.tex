\def\Arccosh{{\mathrm Arccosh}}
\chapter{Introduction}

This paper documents a computer-assisted procedure
for rigorously analyzing small hyperbolic $3$-manifolds.
Briefly, we will define a compact six-dimensional space ${\mathcal P}$
that parameterizes pairs of elements of ${\mathrm Isom}({\mathbf H}^3)$,
and then construct a regular binary space partition (BSP) tree
which subdivides ${\mathcal P}$ into subregions ${\mathcal P_i}$
and whose leaves are --- with a few exceptions --- labeled with killerwords.
These killerwords will encode miniature proofs that ${\mathcal P_i}$
cannot contain any points which correspond to
particular choices of pairs of generators of
any torsion-free discrete group of ${\mathrm Isom}({\mathbf H}^3)$.
This tree of mini-proofs will then be used
to exhaustively isolate all possible manifolds
which can have properties related to the dimensions of ${\mathcal P}$.

The first two applications of the procedure
were used to prove a proposition from \cite{GMT}:

\begin{proposition}{\cite{GMT}}
Let $M$ be an orientable hyperbolic $3$-manifold, and let $\delta$ be
a shortest geodesic. Then, either ${\mathit tuberadius}(\delta) > \ln(3)/2$,
or $M$ lies within one of seven tightly constrained
exceptional shortest-geodesic-geometry regions.
\end{proposition}

and a related proposition; both of which were used in the proof of
the main technical result of \cite{GMT}:
\begin{theorem}{\cite{GMT}}\label{GMT 0.1}\label{GMT 0.2}
Every closed hyperbolic $3$-manifold
has a non-coalescible insulator family,
indeed one coming from a shortest geodesic.
As a consequence,
homotopy hyperbolic $3$-manifolds are hyperbolic.
\end{theorem}
More recently, the proposition has been sharpened:
\begin{theorem}{}
Let $M$ be an orientable hyperbolic $3$-manifold,
and let $\delta$ be a shortest geodesic.
Then, either ${\mathit tuberadius}(\delta) > \ln(3)/2$, or
$M$ is isometric to one of seven specific manifolds.
\end{theorem}
\begin{proof}{}
	Let $X_0, \cdots, X_6$ denote the exceptional regions of Proposition \ref{GMT 1.28}
with $N_i$ the corresponding conjectural manifold.
\cite{GMT} showed that $N_0$=Vol3 is the unique manifold in the region
and \cite{JR} showed that Vol3 covered no 3-manifold.
	\cite{JR} also proved that $N_5$ and $N_6$ are isometric.
\cite{CLLMR} and \cite{L} identify a manifold in each region
and \cite{CLLMR} show that these manifolds are the unique ones in its region.
\cite{CLLMR} show that $N_1$, $N_5$ and $N_6$ cover no manifold.
In \cite{GT} the proof is completed by showing that $N_3$ covers no manifold
and each of $N_2$ and $N_4$ 2-fold cover manifolds,
however the quotients are all non exceptional,
i.e. each shortest geodesic has a $ln(3)/2$ tube.
\end{proof}

Other applications of the procedure:
\vskip 8pt
Theorem \cite{G}  (Smale conjecture of hyperbolic 3-manifolds)
If N is a closed hyperbolic 3-manifold,
then the natural inclusion ${\mathrm Isom}(N)\to {\mathrm Diff}(N)$ is a homotopy equivalence.
\vskip 8pt
The proof made essential use of the fact that a shortest geodesic
of a closed hyperbolic 3-manifold satisfies the insulator condition \cite{GMT}.
\vskip 8pt
Theorem \cite{GMM}, \cite{Mi} The Week's manifold
is the unique closed orientable hyperbolic 3-manifold of minimal volume.

This result relies on 

Lemma \cite{ACS}  Suppose that M is a closed, orientable hyperbolic 3-manifold,
and that C is a shortest geodesic in M.
Set $N = \mathrm{drill}_C(M)$. If $\mathrm{tuberad}(C) \ge \ln(3)/2$ then
$\mathrm{vol} N < 3:0177 \mathrm{vol}M$.

This is based on a result of \cite{ADST},
that makes essential use of Perelman's [Pe] monotonicty result.
a key element of his proof of geometrization and the $\ln(3)/2$
theorem of \cite{GMT}.
Lemma \cite{ACS} is used to prove other results such as 

Theorem \cite{ACS} Suppose that M is a closed, orientable, hyperbolic
3-manifold such that ${\mathrm vol}(M) \le 1.22$.
Then $H_1(M:Zp)$ has dimension at most 3 for every prime p.

(The statement is sharpened slightly to take into account \cite{GT})

Theorem \cite{LM} Let M be a compact orientable 3-manifold with boundary a torus,
and with interior admitting a complete finite-volume hyperbolic structure.
Then the number of exceptional slopes on M is at most 10.

This result, known as the Gordon conjecture,
relies on rigorous computer assistance using the AffApprox technology.

Finally, the procedure is currently in use
by Gabai, Haraway, Meyerhoff, Thurston, and Yarmola
to rigorously analyze small orientable cusped hyperbolic $3$-manifolds,
and to rigorously analyze small non-orientable closed hyperbolic $3$-manifolds.


This paper is organized as follows.

In Section \ref{Ch. Technical Introduction} we introduce
\cite{GMT}; Proposition 1.28 as Theorem \ref{GMT 1.28},
and sketch its proof.

In Section 2 we provide a formal statement and proof of Theorem \ref{GMT 1.28}.

In Section 3,
the method for describing the decomposition
of the parameter space ${\mathcal W}$ into sub-regions is given,
and the conditions used to eliminate sub-regions are discussed.
Near the end of this section, the first part of a detailed example is given.

Eliminating 
a sub-region requires that a certain function is shown to be bounded 
appropriately over the entire sub-region.  This is carried out by using a 
first-order Taylor approximation of the function together with a 
remainder 
bound.
Our computer version of such a Taylor approximation with remainder bound 
is called an {\textit AffApprox} and in Section 4,
the relevant theory is developed.
At this point, the detailed example of Section 3 can be completed.

In Sections 5 and 6, round-off error analysis appropriate to our 
set-up is introduced.  Specifically, in Section 6, round-off error is 
incorporated 
into the {\textit AffApprox} formulas introduced in Section 4.  The proofs here 
require an analysis of round-off error for complex numbers, which is carried 
out in Section 5.

In Section 7 we give some hints about the search for a atree,
in the hope that they will help others endeavoring to apply these methods.

Finally, in Section 8, we present an updated version of
the code used to check that the proof is valid,
with self-contained copies of the proofs required for the reader
to check its own validity.

{\textit Prior publication}
Sections 2 through 6 originally appeared in \cite{GMT},
and are reproduced here with only minor revision.
Section 1 is an abridged version of the parts of the rest of \cite{GMT}.
Section 7 borrows from material which originally appeared in remarks in \cite{GMT}.

{\textit Acknowledgements}.
%TODO: Acknowledge
