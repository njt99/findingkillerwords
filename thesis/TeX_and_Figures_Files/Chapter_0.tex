\def\Arccosh{ \mathrm {Arccosh}}
\chapter{Introduction}\label{Ch.intro}

This paper documents a computer-assisted procedure
for rigorously analyzing small hyperbolic $3$-manifolds.
Briefly, we will define a compact six-dimensional space $\mathcal {P}$
that parameterizes pairs of elements of $\mathrm {Isom}(\mathbf {H}^3)$,
and then construct a regular binary space partition (BSP) tree
which subdivides $\mathcal {P}$ into subregions $\mathcal P_i$
and whose leaves are --- with a few exceptions --- labeled with killerwords.
These killerwords will encode miniature proofs that $\mathcal P_i$
cannot contain any points which correspond to
particular choices of pairs of generators of
any torsion-free discrete group of $\mathrm {Isom}(\mathbf {H}^3)$.
This tree of mini-proofs will then be used
to exhaustively isolate all possible manifolds
which can have properties related to the dimensions of $\mathcal {P}$.

The first two applications of the procedure
were used to prove a proposition from \cite{GMT}:

\begin{proposition}{\cite{GMT}}
Let $M$ be an orientable hyperbolic $3$-manifold, and let $\delta$ be
a shortest geodesic. Then, either $\mathit {tuberadius}(\delta) > \ln(3)/2$,
or $M$ lies within one of seven tightly constrained
exceptional geodesic-geometry regions.
\end{proposition}

A more precise form of this proposition is restated in the next chapter
as \ref{GMT 1.28}.
This proposition and a related one were used in the proof of
the main technical result of \cite{GMT}, the topological rigidity of hyperbolic 3-manifolds. 

\begin{theorem}{\cite{GMT}}\label{GMT 0.2}
Every closed hyperbolic $3$-manifold
has a non-coalescible insulator family,
indeed one coming from a shortest geodesic.
As a consequence,
homotopy hyperbolic $3$-manifolds are hyperbolic.
\end{theorem}

More recently, the proposition has been sharpened:

\begin{theorem}{}
Let $M$ be an orientable hyperbolic $3$-manifold,
and let $\delta$ be a shortest geodesic.
Then, either $\mathit {tuberadius}(\delta) > \ln(3)/2$, or
$M$ is isometric to one of seven specific manifolds.
\end{theorem}

\begin{proof}{}
	Let $X_0, \cdots, X_6$ denote the exceptional regions of Proposition \ref{GMT 1.28}
with $N_i$ the corresponding conjectural manifold.
\cite{GMT} showed that $N_0$=Vol3 is the unique manifold in the region
and \cite{JR} showed that Vol3 covered no 3-manifold.
	\cite{JR} also proved that $N_5$ and $N_6$ are isometric.
\cite{CLLMR} and \cite{L} identify a manifold in each region
and \cite{CLLMR} show that these manifolds are the unique ones in its region.
\cite{CLLMR} show that $N_1$, $N_5$ and $N_6$ cover no manifold.
In \cite{GT} the proof is completed by showing that $N_3$ covers no manifold
and each of $N_2$ and $N_4$ 2-fold cover manifolds,
however the quotients are all non exceptional,
i.e. each shortest geodesic has a $ln(3)/2$ tube.
\end{proof}

Other applications of the procedure:
\vskip 8pt
\begin{theorem} \cite{G1}  (Smale conjecture of hyperbolic 3-manifolds)
If N is a closed hyperbolic 3-manifold,
then the natural inclusion $\mathrm {Isom}(N)\to \mathrm {Diff}(N)$ is a homotopy equivalence.\end{theorem}
\vskip 8pt
The proof makes essential use of the fact that a shortest geodesic
of a closed hyperbolic 3-manifold satisfies the insulator condition \cite{GMT}.
\vskip 8pt
\begin{theorem} \cite{GMM}, \cite{Mi} The Weeks manifold
is the unique closed orientable hyperbolic 3-manifold of minimal volume.\end{theorem}

In addition to \cite{GMT} this result relies on 

\begin{lemma} \cite{ACS}\label{acs}  Suppose that M is a closed, orientable hyperbolic 3-manifold,
and that C is a shortest geodesic in M.
Set $N = \mathrm{drill}_C(M)$. If $\mathrm{tuberad}(C) \ge \ln(3)/2$ then
$\mathrm{vol} N < 3.0177 \mathrm{vol}M$.\end{lemma}

This is based on a result of \cite{ADST},
that makes essential use of Perelman's \cite{Pe1}, \cite{Pe2} Ricci flow with surgery and as well as his monotonicity result, which are key elements of his proof of geometrization.  Lemma \ref{acs} is used to prove other results such as:


%It  and the $\ln(3)/2$
%theorem of \cite{GMT}.


\begin{theorem}\cite{ACS} Suppose that M is a closed, orientable, hyperbolic
3-manifold such that $\mathrm {vol}(M) \le 1.22$.
Then $H_1(M:Zp)$ has dimension at most 3 for every prime p.\end{theorem}

That result also requires this work and \cite{GMT}.  \vskip 8pt

The methods of this work are crucial to the following result.

\begin{theorem} \cite{GHMTY} The figure-8 knot complement is the unique 1-cusped hyperbolic 3-manifold with 9 or more non hyperbolic fillings.  \end{theorem}
 
This result gives a positive proof of the long-standing Gordon exceptional filling conjecture that has attracted much attention.  See \cite{GHMTY} for a detailed history.  An important result in this direction was the theorem of Lackenby and Meyerhoff \cite{LM} who showed that a cusped hyperbolic 3-manifold has at most 10 exceptional fillings.  That result also relied on rigorous computer assistance using the AffApprox technology.
\vskip 8pt
For a survey of this work and further developments including other applications, see \cite{GMTY}.

\vskip 8pt

This paper is organized as follows.

In Chapter \ref{Ch.TechIntro} we introduce
\cite{GMT}; Proposition 1.28 as Theorem \ref{GMT 1.28},
and sketch its proof.

In Chapter \ref{Ch.WordsParams} we provide a formal statement and proof of Theorem \ref{GMT 1.28}.

In Chapter \ref{Ch.ConditionsTree},
the method for describing the decomposition
of the parameter space $\mathcal {W}$ into sub-regions is given,
and the conditions used to eliminate sub-regions are discussed.
Near the end of this section, the first part of a detailed example is given.

Eliminating 
a sub-region requires that a certain function is shown to be bounded 
appropriately over the entire sub-region.  This is carried out by using a 
first-order Taylor approximation of the function together with a 
remainder 
bound.
Our computer version of such a Taylor approximation with remainder bound 
is called an {\textit AffApprox} and in Chapter \ref{Ch.AffApprox},
the relevant theory is developed.
At this point, the detailed example of Chapter \ref{Ch.ConditionsTree} can be completed.

In Chapter \ref{Ch.XComplex} and \ref{Ch.AffWithEPS}, round-off error analysis appropriate to our 
set-up is introduced.  Specifically, in Chapter \ref{Ch.AffWithEPS}, round-off error is 
incorporated 
into the {\textit AffApprox} formulas introduced in Chapter \ref{Ch.AffApprox}.  The proofs here 
require an analysis of round-off error for complex numbers, which is carried 
out in Chapter \ref{Ch.XComplex}.

In Section \ref{Ch.Crumbs} we give some hints about the search for a tree,
in the hope that they will help others endeavoring to apply these methods.
\vskip 8pt

\noindent\textbf {Prior publication:} Chapter 2 includes material from section 0 of \cite{GMT}.
Chapters 3 through 7 originally appeared as sections 1 and 5 through 8 of \cite{GMT},
and are reproduced here with minor revision.  In particular, Chapter 3 is a somewhat abridged version of section 1 of \cite{GMT}.
Chapter 8, which includes new insights, also contains material which originally appeared in remarks in \cite{GMT}, and is related to material which originally appeared in \cite{GMTY}.

\vskip 8pt

\noindent\textbf {Acknowledgements:}
\begin{enumerate}
	\item David Gabai (relentless hope; orchestrating support; ruthless focus)
	\item Robert Meyerhoff (writing down AffApprox proofs)
	\item The Geometry Center (creative environment)
	\item UC Berkeley (patience)
\end{enumerate}
