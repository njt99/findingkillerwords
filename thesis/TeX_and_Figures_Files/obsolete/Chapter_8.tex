\section{AffApproxes with round-off error}
 
In Section 6, we saw how to do calculations with AffApproxes.  Here, we incorporate round-off error into
these calculations.  

\demo{Conventions {\rm 8.1}}
An AffApprox $x$ is a five-tuple
$(x.f; x.f_0,x.f_1,x.f_2; x.{\rm err})$
consisting of four complex numbers $(x.f, x.f_0,x.f_1, x.f_2)$ and one real number $x.{\rm err}.$  In Section 6, the
real number was denoted $x.e,$ but it seems preferable to use $x.{\rm err}$ here.  Recall (Definition 6.4)
that an AffApprox $x$ represents the set $S(x)$ of functions from $A = \{(z_0,z_1,z_2) \in {\bf C}^3 : |z_k| \le 1 \
{\rm for} \ k \in \{0,1,2\}\}$ to ${\bf C}$ that are $x.{\rm err}$-well-approximated by the affine function
$x.f + x.f_0 z_0 + x.f_1 z_1 + x.f_2 z_2.$
\enddemo

{\it Remark} 8.2.
A review of Definition 7.2 (XComplexes and AComplexes; loosely, exact and approximate complex numbers) might be helpful at this time.

One approach to round-off error for AffApproxes would be to replace the four complex numbers in the definition of AffApprox by four AComplex numbers complete with their round-off errors, and similarly for the one real number.  We will not do this because it would necessitate keeping track of five separate round-off-error terms when we do AffApprox calculations.

Instead, we will 
replace the four complex numbers by four XComplexes and push all the round-off error into the $.{\rm err}$ term.  
Thus, the definition of AffApprox remains 
essentially the same as in Section 6.
We note that, in doing an AffApprox calculation, our subsidiary calculations will generally be on XComplex numbers and produce an AComplex number whose $.e$ term will be plucked off and forced into the $.{\rm err}$ term of the final AffApprox.

\demo{Conventions {\rm 8.3}}
i)  In what follows, we will use Basic Properties 7.0 and Lemmas 7.0 and 7.1.  Also, the propositions in Section 6
will be utilized;  as such, the numbering of the propositions is the same in Sections 6 and 8 (for example,
Proposition 6.7 corresponds to Proposition 8.7).  
\vglue4pt
ii)  Some notational simplifications will be introduced: ${\rm dist}(x)$ before Proposition 8.6, $ax$ before Propositions 8.8, 8.9, 8.11
(the middle usage of $ax$ differs slightly from the other two), and $ay$ before Propositions 8.8, 8.9.
\vglue4pt
iii)  We will try to keep our notation fairly consistent with that of the computer programs {\it verify} and {\it corona} and this will produce some mildly peculiar notation.  In particular, in the operations pertaining to Propositions 8.2 and beyond, the resultant AffApproxes will be denoted $(r\_f.z; r\_f_1.z, r\_f_2.z, 
r\_f_3.z;\break r\_{\rm error})$
where the first four terms are XComplexes and the last term is a double (technically, $r\_f.z$ is the XComplex part of the AComplex  $r\_f$ and similarly for the $r\_f_k.z$ terms).  One break with the notation of the programs though is that AffApproxes are called (in the programs) ACJ's, which stands for ``Approximate Complex 1-Jets."
\vglue4pt
iv) The propositions that follow include in their statements the definitions of the various operations on AffApproxes (see Remark
6.6iii)).
\vglue4pt
v)  We remind the reader (see Section 7.4) that all machine operations act on machine numbers, and that the 
various variables appearing in the propositions are assumed to be doubles.
\enddemo

$-X$:

\nonumproclaim{Proposition 8.1}
If $x$ is an AffApprox 
then $S(-x) = -(S(x))$ where
$$-x \equiv (-x.f; -x.f_0,-x.f_1,-x.f_2; x.{\rm err}).$$ \endproclaim


$X+Y$:
We analyze the addition of the  {\rm AffApproxes} 
 $x = (x.f; x.f_0,x.f_1,\break x.f_2; x.{\rm err})$  and 
$y = (y.f; y.f_0,y.f_1,y.f_2; y.{\rm err}).$  To get the first term in $x+y$ we add the XComplex numbers $x.f$ and $y.f;$ which
produce the AComplex number $r\_f = x.f + y.f$ (see Proposition 7.4), and then we pluck off the XComplex part,
which we denote $r\_f.z.$  The round-off error part $r\_f.e$ will be foisted into the overall error term $r\_{\rm error}$
for $x+y.$ Similarly for the next three terms in $x + y.$

Abstractly, the overall error term $r\_{\rm error}$ comes from adding the round-off error contributions $r\_f.e,\  r\_f_0.e,\  r\_f_1.e, 
\ r\_f_2.e$ and the AffApprox error contributions $x.{\rm err},\  y.{\rm err}$.  Of course, we have to produce a machine version.

\nonumproclaim{Proposition 8.2} If $x$ and $y$ are 
{\rm AffApproxes, }
then $S(x + y) \supseteq S(x) + S(y)${\rm ,} where
$$x + y \equiv (r\_f.z; r\_f_0.z, r\_f_1.z, r\_f_2.z; r\_{\rm error})$$
with
\begin{eqnarray*}
r\_f &=& x.f + y.f,\\
r\_f_k &=& x.f_k + y.f_k,\\
r\_{\rm error}& =& (1 + 3EPS)\\
&&\otimes\ ((x.{\rm err} \oplus y.{\rm err}) \oplus ((r\_f.e \oplus
r\_f_0.e) \oplus (r\_f_1.e \oplus r\_f_2.e))).
\end{eqnarray*}
\endproclaim

\demo{Proof}
The error is given by 
\begin{eqnarray*}
&&\hskip-24pt (x.{\rm err} + y.{\rm err})  +\ ((r\_f.e + r\_f_0.e) + (r\_f_1.e + r\_f_2.e))\\
&\le& (1+EPS/2) (x.{\rm err} \oplus
y.{\rm err}) \\
&&+\ (1+EPS/2)((r\_f.e \oplus r\_f_0.e) + (r\_f_1.e \oplus r\_f_2.e))\\
&\le& (1+EPS/2)^3 ((x.{\rm err} \oplus
y.{\rm err}) \oplus ((r\_f.e \oplus r\_f_0.e) \oplus (r\_f_1.e \oplus r\_f_2.e)))\\
&\le& (1 + 3EPS) \otimes ((x.{\rm
err} \oplus y.{\rm err}) \oplus ((r\_f.e \oplus r\_f_0.e) \oplus (r\_f_1.e \oplus r\_f_2.e))).
\end{eqnarray*}
 To get the last line
we used Lemma 7.1. \enddemo

$X - Y$:

\nonumproclaim{Proposition 8.3:} If $x$ and $y$ are {\rm AffApproxes,} then $S(x - y) \supseteq S(x) - S(y)${\rm ,}
 where
$$x - y \equiv (r\_f.z; r\_f_0.z, r\_f_1.z, r\_f_2.z; r\_{\rm error})$$
with
\begin{eqnarray*}
r\_f& =& x.f - y.f,\\
r\_f_k &= &x.f_k - y.f_k,\\
r\_{\rm error} &=& (1 + 3EPS)\\
&& \otimes\ ((x.{\rm err} \oplus y.{\rm err}) \oplus ((r\_f.e
\oplus r\_f_0.e) \oplus (r\_f_1.e \oplus r\_f_2.e))).
\end{eqnarray*}
\endproclaim

$X + D$: 
Here, we add the AffApprox $x = (x.f; x.f_0,x.f_1,x.f_2; x.{\rm err})$  to the double $y$.  The only terms that change are the first and the last.

\nonumproclaim{Proposition 8.4}\hskip-8pt If $x$ is an 
{\rm AffApprox}  and $y$ is a double{\rm ,} 
then $S(x + y)\break \supseteq S(x) + S(y)${\rm ,} where
$$x + y \equiv (r\_f.z; r\_f_0.z, r\_f_1.z, r\_f_2.z; r\_{\rm error})$$
with
\begin{eqnarray*}
r\_f& =& x.f + y,\\
r\_f_k &=& x.f_k,\\
r\_{\rm error}& = &(1 + EPS) \otimes (x.{\rm err} \oplus r\_f.e).\end{eqnarray*}
\endproclaim

\demo{Proof}
The error is given by 
$$x.{\rm err} + r\_f.e \le (1 + EPS) \otimes (x.{\rm err} \oplus r\_f.e)$$
by Lemma 7.0. \enddemo

$X - D$:
\nonumproclaim{Proposition 8.5}\hskip-8pt If $x$ is an 
{\rm AffApprox}  and $y$ is a double{\rm ,}
then $S(x - y) \break\supseteq S(x) - S(y)${\rm ,} where
$$x - y \equiv (r\_f.z; r\_f_0.z, r\_f_1.z, r\_f_2.z; r\_{\rm error})$$
with
\begin{eqnarray*}
r\_f &= &x.f - y,\\
r\_f_k &=& x.f_k,\\
r\_{\rm error}& =& (1 + EPS) \otimes (x.{\rm err} \oplus r\_f.e).\end{eqnarray*}
\endproclaim

$X \times Y$: 
We multiply the AffApproxes $x$ and $y$ while pushing all error into the $.{\rm err}$ term.

We will use the functions (see Formulas 7.0 and 7.1, at the end of Section~7) ${\rm absUB}= (1 + 2EPS) \otimes
{\rm hypot}_o(x.re,x.im)$ and 
${\rm absLB}(x) = (1 - 2EPS) \otimes {\rm hypot}_o(x.re,x.im)$.

When $x$ is an AffApprox, we define ${\rm dist}(x)$ to be $$(1 + 2EPS) \otimes ({\rm absUB}(x.f_0) \oplus ({\rm absUB}(x.f_1) \oplus {\rm absUB}(x.f_2))).$$  This is the machine representation of the sum of the absolute values of the linear terms in the AffApprox $x$ 
(the proof is straightforward).

\nonumproclaim{Proposition 8.6} If $x$ and $y$ are 
{\rm AffApproxes,}
then $S(x \times y) \supseteq S(x) \times S(y)${\rm ,} where
$$x \times y \equiv (r\_f.z; r\_f_0.z, r\_f_1.z, r\_f_2.z; r\_{\rm error})$$
with
\begin{eqnarray*}
r\_f &=& x.f \times y.f,\\
r\_f_k &=& x.f \times y.f_k + x.f_k \times y.f,\\
r\_{\rm error} &=& 
(1 + 3EPS) \otimes (A \oplus (B \oplus C)),\end{eqnarray*}
and
\begin{eqnarray*}
A &=& ({\rm dist}(x) \oplus x.{\rm err})\otimes ({\rm dist}(y) \oplus y.{\rm err}),
\\
B& =& {\rm absUB}(x.f) \otimes
y.{\rm err} \oplus {\rm absUB}(y.f) \otimes x.{\rm err},\\
C& =& (r\_f.e \oplus r\_f_0.e) \oplus (r\_f_1.e \oplus
r\_f_2.e).\end{eqnarray*}
\endproclaim

\demo{Proof}
We add the non-round-off error term for $x \times y$ to the various round-off error terms that accumulated.
\begin{eqnarray*}
&&\hskip-48pt (({\rm dist}(x) + x.{\rm err}) \times\ ({\rm dist}(y) + y.{\rm err})) +
 (({\rm absUB}(x.f) \times y.{\rm err} 
\\
&& +\ 
{\rm absUB}(y.f) \times x.{\rm err}) +
 (r\_f.e + r\_f_0.e) + (r\_f_1.e + r\_f_2.e))
\\
&\le& 
(1+EPS/2)^3[({\rm dist}(x) \oplus x.{\rm err})\otimes ({\rm dist}(y) \oplus y.{\rm err})]\\
&& +\
 (1+EPS/2)^2\{({\rm absUB}(x.f) \otimes y.{\rm err} 
\\
& &\oplus\ 
{\rm absUB}(y.f) \otimes x.{\rm err}) +
((r\_f.e \oplus r\_f_0.e) \oplus (r\_f_1.e \oplus r\_f_2.e))\}
\\
&\le& (1+EPS/2)^3 A + (1+EPS/2)^3 (B \oplus C) \\
&\le& (1+3EPS) \otimes (A \oplus (B \oplus C) ). \\
\noalign{\vskip-36pt}
\end{eqnarray*}
\enddemo

$X \times D$: 

\nonumproclaim{Proposition 8.7} \hskip-8pt If $x$ is an 
AffApprox  and $y$ is a double{\rm ,} 
then $S(x \times y)\break \supseteq S(x) \times S(y)${\rm ,} where
$$x \times y = (r\_f.z; r\_f_0.z, r\_f_1.z, r\_f_2.z; r\_{\rm error})$$
with
\begin{eqnarray*}
r\_f& =& x.f \times y,\\
r\_f_k &=& x.f_k \times y,\\
r\_{\rm error} &=& (1+3EPS)\\
&& \otimes\ ( (x.{\rm err} \otimes |y|) \oplus
( (r\_f.e \oplus r\_f_0.e) \oplus (r\_f_1.e \oplus r\_f_2.e)) ).
\end{eqnarray*}
\endproclaim

$X/Y$: 
For convenience, let  $ax = {\rm absUB}(x.f),\ ay = {\rm absLB}(y.f).$ 
 
\nonumproclaim{Proposition 8.8} If $x$ and $y$ are 
{\rm AffApproxes}  with $D > 0$ {\rm (}\/see below{\rm ),} then
$S(x / y) \supseteq S(x) / S(y)${\rm ,} where
$$x / y \equiv (r\_f.z; r\_f_0.z, r\_f_1.z, r\_f_2.z; r\_{\rm error})$$
with
\begin{eqnarray*}
r\_f &=& x.f / y.f,
\\
r\_f_k& =& (x.f_k \times y.f - x.f \times y.f_k)/ (y.f \times y.f),\\
r.{\rm error} & =& (1 + 3EPS) \otimes (
((1 + 3EPS) \otimes A \ominus (1 - 3EPS) \otimes B )\oplus C ),\end{eqnarray*}
and
\begin{eqnarray*}
A &=& (ax \oplus ({\rm dist}(x) \oplus x.{\rm err})) \oslash D,\\
B &=& (ax \oslash ay \oplus {\rm dist}(x) \oslash ay)
\oplus  (({\rm dist}(y) \otimes ax) \oslash (ay \otimes ay)),\\
C &=& (r\_f.e \oplus r\_f_0.e) \oplus (r\_f_1.e \oplus
r\_f_2.e),\\
D &=& ay \ominus (1 + EPS) \otimes ({\rm dist}(y) \oplus y.{\rm err}).
\end{eqnarray*}
\endproclaim

\demo{Proof}
As usual, we add the round-off errors to the old AffApprox error, 
taking into account round-off error, working on it bit by bit.
\begin{eqnarray*}
 &&\hskip-16pt (ax + {\rm dist}(x) + x.{\rm err})/(ay - ({\rm dist}(y) + y.{\rm err}))\\
&&  \le (1 + EPS/2)^2 \\
&&\quad \times \ (ax \oplus ({\rm dist}(x)
\oplus x.{\rm err}))/
         (ay - (1 + EPS) \otimes ({\rm dist}(y) \oplus y.{\rm err}))\\
&&  \le (1 + EPS/2)^2 (ax \oplus ({\rm dist}(x) \oplus
x.{\rm err}))/
         \Big(\left({1 \over 1 + EPS/2}\right) \\
&&\quad\times \ (ay \ominus (1 + EPS) \otimes ({\rm dist}(y) \oplus y.{\rm err}))\Big)\\
&& \le (1 + EPS/2)^4 (ax
\oplus ({\rm dist}(x) \oplus x.{\rm err}))\\
&&\quad \oslash
         (ay \ominus (1 + EPS) \otimes ({\rm dist}(y) \oplus y.{\rm err}))\\
&& \le   (1 + 3EPS) \otimes A.
\end{eqnarray*}

The next term, being subtracted, requires opposite inequalities.
\begin{eqnarray*}
&&(ax/ay + {\rm dist}(x)/ay) + {\rm dist}(y) \times ax / (ay \times ay)\\
&&\quad \ge (1-EPS/2)(ax \oslash ay + {\rm
dist}(x)
\oslash ay) \\
&&\qquad +  (1-EPS/2) ({\rm dist}(y) \otimes ax )/ ({1 \over 1 - EPS/2})(ay \otimes ay)\\
&&\quad \ge ((1-EPS/2)^4 ((ax
\oslash ay \oplus {\rm dist}(x) \oslash ay) \oplus  (({\rm dist}(y) \otimes ax) \oslash (ay \otimes ay)))\\
&&\quad \ge (1 +
EPS/2)(1 - 3EPS) (B)  \ge (1 - 3EPS) \otimes B.\end{eqnarray*}

Finally, we do the round-off terms 
$$((r\_f.e + r\_f_0.e) + (r\_f_1.e + r\_f_2.e)) \le (1 + EPS/2)^2 C $$
and we put these three pieces together:
\begin{eqnarray*}
&&
(ax + {\rm dist}(x) + x.{\rm err})/(ay - ({\rm dist}(y) + y.{\rm err}))
\\
&&\qquad  - 
((ax/ay + {\rm dist}(x)/ay) + {\rm dist}(y) \times ax / (ay \times ay))\\
&&\qquad
+ ((r\_f.e + r\_f_0.e) + (r\_f_1.e + r\_f_2.e))
\\
&&\quad \le 
(1 + 3EPS) \otimes A - (1 - 3EPS) \otimes B + (1 + EPS/2)^2 C\\
&&\quad \le (1+EPS/2)^2 (
((1 + 3EPS) \otimes A \ominus (1 - 3EPS) \otimes B)
 + C
) \\
&&\quad \le (1+3EPS) \otimes (
((1 + 3EPS) \otimes A \ominus (1 - 3EPS) \otimes B)
 \oplus C
) . \\
\noalign{\vskip-36pt}
\end{eqnarray*}
\enddemo

$D/X$:
We   divide a double $x$ by an AffApprox $y$.
For convenience, let $ax = |x|,\ ay = {\rm absLB}(y.f).\ $  Having done division out in the previous proposition, we will skip the proof of Proposition 8.9.  See the {\it Annals}
web site for the proof.

\nonumproclaim{Proposition 8.9} If $x$ is a double and $y$ is an 
{\rm AffApprox}  with $D > 0$ {\rm (}\/see below{\rm ),}
 then $S(x / y) \supseteq S(x) / S(y)${\rm ,} where
$$x / y \equiv (r\_f.z; r\_f_0.z, r\_f_1.z, r\_f_2.z; r\_{\rm error})$$
with
 \begin{eqnarray*}
r\_f&\hskip-8pt =\hskip-8pt& x / y.f,\\
r\_f_k&\hskip-8pt =\hskip-8pt& -(x \times y.f_k)/ (y.f \times y.f),\\
r\_{\rm error}&\hskip-8pt =\hskip-8pt& (1+3EPS) \otimes ( 
((1 + 2EPS) \otimes (ax  \oslash D) \ominus (1 - 3EPS) \otimes B)
\oplus C),\\
 B&\hskip-8pt =\hskip-8pt& ax \oslash ay  \oplus ({\rm dist}(y) \otimes ax \oslash (ay \otimes ay)),\\
C& \hskip-8pt=\hskip-8pt& (r\_f.e \oplus r\_f_0.e)
\oplus (r\_f_1.e \oplus r\_f_2.e),\\
D &\hskip-8pt=\hskip-8pt& ay \ominus (1 + EPS) \otimes ({\rm dist}(y) \oplus y.{\rm err}).
\end{eqnarray*}
\endproclaim

$X/D$:
We   divide an AffApprox $x$ by a nonzero double $y$ (the computer will object if $y = 0$).  The proof is easy  and
so we delete it.

\nonumproclaim{Proposition 8.10} 
If $x$ is an {\rm AffApprox}  and $y$ is a nonzero double{\rm ,} then
$S(x / y) \supseteq S(x) / S(y)${\rm ,} where
$$x / y \equiv (r\_f.z; r\_f_0.z, r\_f_1.z, r\_f_2.z; r\_{\rm error}),$$
with
\begin{eqnarray*}
r\_f &\hskip-8pt=\hskip-8pt& x.f / y\\
r\_f_k &\hskip-8pt=\hskip-8pt& x.f_k / y\\
r\_{\rm error} &\hskip-8pt = \hskip-8pt&
(1+3EPS) \otimes ( 
(x.{\rm err}\oslash |y|)
\oplus 
[(r\_f.e \oplus r\_f_0.e) \oplus (r\_f_1.e \oplus r\_f_2.e)]
                                               ).\end{eqnarray*}
\endproclaim

$\sqrt X$: 
Here, $x$ is an AffApprox and we let $ax = {\rm absUB}(x.f)$.  There are two cases to consider depending on whether or not
$D = ax \ominus (1 + EPS) \otimes ({\rm dist}(x) \oplus x.{\rm err})$ is or is not greater than zero.

\nonumproclaim{Proposition 8.11{\rm a}} If $x$ is an {\rm AffApprox}
and
$ D = ax \ominus (1 + EPS) \otimes ({\rm dist}(x) \oplus x.{\rm err})$ is not greater than zero{\rm ,} then 
$S(\sqrt x) \supseteq \sqrt {S(x)}${\rm ,} where
we use the crude overestimate $$\sqrt x \equiv \left(0;0,0,0;(1 + 2EPS) \otimes
\root o \of {(ax \oplus (x\,{\rm dist} \oplus x.{\rm err}))} \right).$$
\endproclaim

\demo{Proof}
\begin{eqnarray*}\sqrt {ax + x\,{\rm dist} + x.{\rm err}}&\le &(1 + EPS/2) \sqrt { (ax \oplus (x\,{\rm dist} \oplus x.{\rm err}))}\\
&\le  &(1 + 2EPS) \otimes \root o \of { (ax \oplus (x\,{\rm dist} \oplus x.{\rm err}))}.
\\
\noalign{\vskip-36pt}
\end{eqnarray*}
\enddemo

\nonumproclaim{Proposition 8.11{\rm b}} If 
$x$  is an {\rm AffApprox} and
$ D = ax \ominus (1 + EPS) \otimes ({\rm dist}(x) \oplus x.{\rm err})$ is greater than zero{\rm ,}
then $S(\sqrt x) \supseteq \sqrt {S(x)}${\rm ,} where
 $$\sqrt{x} \equiv (r\_f.z; r\_f_0.z,r\_f_1.z,r\_f_2.z; r\_{\rm error})$$ 
with
\begin{eqnarray*}
r\_f& = &\sqrt {x.f},\\
t &=& r\_f + r\_f,\\
r\_f_k& = &{\rm AComplex} (x.f_k.re,x.f_k.im;0) / t.
\end{eqnarray*}
{\rm (}\/Simply put{\rm ,} $r\_f_k = x.f_k / (2\sqrt{x.f})$. The reason we have to fuss to define $r\_f_k$ is
 because
$\sqrt{x.f}$ is an {\rm AComplex.)}
\begin{eqnarray*}
  r\_{\rm error}& \hskip-8pt =\hskip-8pt &
(1+3EPS)\\ &\hskip-8pt\hskip-8pt& \otimes\hskip-1pt \left\{\hskip-1pt
(1+EPS) \otimes \root o \of {ax}\ominus
(1 - 3EPS) \otimes [{\rm dist}(x)\oslash(2 \times \root o \of {ax})
\oplus \root o \of {D} ]\right\}
\\
&\hskip-8pt\hskip-8pt&  \oplus  
((r\_f.e \oplus r\_f_0.e) \oplus (r\_f_1.e \oplus r\_f_2.e))
.\end{eqnarray*}
\endproclaim

\demo{Proof}
Let us work on the pieces.
$$\sqrt {ax} \le (1 + EPS) \otimes \root o \of {ax}.$$

Next,
\begin{eqnarray*}
&&{\rm dist}(x)/(2 \sqrt {ax}) + \sqrt {ax - ({\rm dist}(x) + x.{\rm err})}\\
&&\quad 
\ge  (1 - EPS/2)^2 {\rm dist}(x) \oslash (2 \root o \of {ax}) 
\\
&&\qquad + (1 - EPS/2)^{1/2}\sqrt {ax \ominus (1+EPS) \otimes ({\rm dist}(x) \oplus x.{\rm err})}
                                              \\
&&\quad \ge(1 - EPS/2)^3 \left[{\rm dist}(x) \oslash (2 \root o \of {ax}) \oplus \root o \of {D}\
\right]\\
&&\quad \ge (1 + EPS/2) (1 - 3EPS) \left[{\rm dist}(x) \oslash (2 \root o \of {ax}) \oplus \root o \of {D}\ \right]\\
&&\quad \ge (1 -
3EPS) \otimes \left[{\rm dist}(x) \oslash (2 \root o \of {ax}) \oplus \root o \of {D}\ \right].
\end{eqnarray*}

Adding in the usual term, we get as our error bound
\begin{eqnarray*}
&&\sqrt {ax} - ({\rm dist}(x)/(2 \sqrt {ax}) + \sqrt {ax - ({\rm dist}(x) + x.{\rm err})}\ )\\
&&\qquad+\ ((r\_f.e + r\_f_0.e) + (r\_f_1.e + r\_f_2.e))\\
&&\quad \le (1 + EPS) \otimes \root o \of {ax} 
- (1 - 3EPS) \otimes [{\rm dist}(x) \oslash (2 \root o \of {ax}) \oplus 
\root o \of {D}\ ]
\\
&&\qquad+\ (1 + EPS/2)^2 ((r\_f.e \oplus r\_f_0.e) \oplus (r\_f_1.e \oplus r\_f_2.e))
      \\
&&\quad \le (1 + EPS/2)^3 \big(\{(1 + EPS) \otimes \root o \of {ax} \\
&&\qquad 
\ominus\ (1 - 3EPS) \otimes [{\rm dist}(x) \oslash (2 \root o \of {ax}) \oplus \root o \of {D}\ ]\}
\\
&&\qquad \oplus\ ((r\_f.e \oplus r\_f_0.e) \oplus (r\_f_1.e \oplus r\_f_2.e))\big)
 \\
&&\quad \le (1 + 3EPS) \otimes (
\{(1 + EPS) \otimes \root o \of {ax} 
\\
&&\qquad\ominus\ (1 - 3EPS) \otimes [{\rm dist}(x) \oslash (2 \root o \of {ax})\oplus \root o \of {D}\ ]\}
\\
&&\qquad \oplus ((r\_f.e \oplus r\_f_0.e) \oplus (r\_f_1.e \oplus r\_f_2.e))
                                                ).\\
\noalign{\vskip-36pt}
\end{eqnarray*}
\enddemo



