\chapter{Technical Introduction}\label{Ch. Technical Introduction}
Here is a brief description of why Proposition \ref{GMT 0.2} might be amenable to 
computer-assisted proof.
If a shortest geodesic $\delta$ in a hyperbolic $3$-manifold $N$ does not have 
a $\ln(3)/2$
tube then there is a 2-generator subgroup $G$ of $\pi_1(N) = \Gamma$
which also does not have that property.
Specifically, take $G$ generated by $f$ and $w$,
with $f \in \Gamma$ a primitive hyperbolic isometry
whose fixed axis $\delta_0 \subset {\mathbf H}^3$ projects to $\delta$, and 
with $w \in \Gamma$ a hyperbolic isometry
which takes $\delta_0$ to a nearest translate.
Then, after identifying $N={\mathbf H}^3/\Gamma$ and letting 
$Z={\mathbf H}^3/G$,
we see that the shortest geodesic in $Z$ (which corresponds to $\delta$)
does not have 
a $\ln(3)/2$ tube. 
Thus, to understand solid tubes around shortest geodesics in hyperbolic 
$3$-manifolds, we need to understand appropriate 2-generator groups, and this 
can be done by a parameter space analysis as follows.  (Parameter space 
analyses are naturally amenable to 
computer proofs.)

The space of {\textit relevant}
(see
Definition \ref{GMT 1.12})
2-generator groups 
in ${\mathrm Isom}_+({\mathbf H}^3)$ is naturally
parametrized by a subset ${\mathcal P}$ of ${\mathbf C}^3.$  
Each parameter corresponds to a
2-generator group $G$ with specified generators $f$ and $w$, and we call 
such a group a {\textit marked group}. 
The marked groups of particular interest are those in which $G$ is
discrete, torsion-free, parabolic-free, $f$ corresponds to a shortest
geodesic $\delta$, and $w$ corresponds to a
covering translation of a particular lift of
$\delta$ to a nearest translate.  
We denote this set of particularly interesting marked groups by ${\mathcal T}.$
We show that if tuberadius($\delta) \le \ln(3)/2$ 
in a hyperbolic $3$-manifold $N,$ then 
$G$ must correspond to a parameter lying in one of seven small regions
${\mathcal R}_n,\ n=0,\ldots,6$ 
in ${\mathcal P}$.  
With respect to this notation, we have:

\vglue8pt 
\begin{proposition} \label{GMT 2.19}
 %\ref{GMT 2.19}}.
${\mathcal T} \cap ({\mathcal P} - \mathbf{\cup}_{n=0,\ldots,6}{\mathcal R}_n) = \emptyset.$
\vfill\end{proposition}

The full statement of Proposition \ref{GMT 2.19}
explicitly describes the seven small
regions of the parameter space as well as some associated data.

Here is the idea of the proof.
Roughly speaking, we subdivide ${\mathcal P}$ into a billion regions
of varying sizes,
and show that all but the seven exceptional regions cannot contain 
a parameter corresponding to a  
``shortest/nearest" marked group.
For example we would know that 
a region ${\mathcal R}$ contained no such group if we knew that for each 
point $\rho\in {\mathcal R}$,
Relength($f_\rho) > {\mathrm Relength}(w_\rho).$  
(Here Relength($f_\rho$) (resp.\ Relength($w_\rho^{\phantom{|}}$))
denotes the real translation length of the isometry of ${\mathbf H}^3$ 
corresponding to
the element $f$ (resp.\ $w$) in the marked group with parameter~$\rho.$)
This inequality would contradict the fact that $f$ corresponds to $\delta$ 
which is a {\textit shortest} geodesic.  Similarly, there are {\textit nearest}
contradictions.
