%% thesis.tex 2014/04/11
%
% Based on sample files of unknown authorship.
%
% The Current Maintainer of this work is Paul Vojta.

\documentclass{ucbthesis}
\usepackage{biblatex}
\usepackage{rotating} % provides sidewaystable and sidewaysfigure

%Added packages
\usepackage{mdframed}
\usepackage{etoolbox}
\usepackage{amssymb,epsfig,amsmath,latexsym,amsthm}
\pdfoptionalwaysusepdfpagebox=5
\usepackage{graphicx}

% To compile this file, run "latex thesis", then "biber thesis"
% (or "bibtex thesis", if the output from latex asks for that instead),
% and then "latex thesis" (without the quotes in each case).

% Double spacing, if you want it.  Do not use for the final copy.
% \def\dsp{\def\baselinestretch{2.0}\large\normalsize}
% \dsp

% If the Grad. Division insists that the first paragraph of a section
% be indented (like the others), then include this line:
% \usepackage{indentfirst}

\addtolength{\abovecaptionskip}{\baselineskip}
\theoremstyle{plain}
%\declaretheorem[style=thmstyle,name=Theorem,numberwithin=chapter]{theorem}
\newtheorem{theorem}{Theorem}
\renewcommand{\thetheorem}{\arabic{chapter}.\arabic{theorem}}
\newtheorem{proposition}[theorem]{Proposition}
\newtheorem{lemma}[theorem]{Lemma}
\newtheorem{corollary}[theorem]{Corollary}
\newtheorem{question}[theorem]{Question}
\newtheorem{questions}[theorem]{Questions}
\newtheorem{conjecture}[theorem]{Conjecture}
\newtheorem{problem}[theorem]{Problem}
\theoremstyle{definition}
\newtheorem{definition}[theorem]{Definition}
\newtheorem{definition/construction}[theorem]{Definition/Construction}
\newtheorem{notationsandconventions}[theorem]{Notations And Conventions}
\newtheorem{construction}[theorem]{Construction}
\newtheorem{remark}[theorem]{Remark}
\newtheorem{summary}[theorem]{Summary}
\newtheorem{history}[theorem]{History}

\newtheorem{addendum}[theorem]{Addendum}
\newtheorem{remarks}[theorem]{Remarks}
\newtheorem{data memo}[theorem]{Data Memo}
\newtheorem{correction}[theorem]{Correction}
\newtheorem{warning}[theorem]{Warning}
\newtheorem{notation}[theorem]{Notation}
\newtheorem{convention}[theorem]{Convention}
\newtheorem{conventions}[theorem]{Conventions}
\newtheorem{example}[theorem]{Example}



%\DeclareMathOperator{\Image}{Im}
%\DeclareMathOperator{\Diff}{Diff}
%\DeclareMathOperator{\hyp}{hyp}
%\DeclareMathOperator{\Emb}{Emb}
%\DeclareMathOperator{\Map}{Map}
%\DeclareMathOperator{\Maps}{Maps}
%\DeclareMathOperator{\Hyp}{Hyp}
%\DeclareMathOperator{\id}{id}
%\DeclareMathOperator{\RM}{RM}
%\DeclareMathOperator{\Bd}{Bd}
%\DeclareMathOperator{\length}{length}
%\DeclareMathOperator{\area}{area}
%\DeclareMathOperator{\Isom}{Isom}
%\DeclareMathOperator{\Diam}{Diam}
%\DeclareMathOperator{\diam}{diam}
%\DeclareMathOperator{\Lim}{Lim}
%\DeclareMathOperator{\rank}{rank}
%\DeclareMathOperator{\genus}{genus}
%\DeclareMathOperator{\kernal}{ker}
%\DeclareMathOperator{\rel}{rel}
%\DeclareMathOperator{\textin}{in}
%\DeclareMathOperator{\textint}{int}
%\DeclareMathOperator{\textinf}{inf}
%\DeclareMathOperator{\textliminf}{lim inf}
%\DeclareMathOperator{\textlim}{lim}
%\DeclareMathOperator{\Stryker}{Stryker}
%\DeclareMathOperator{\st}{st}
%\DeclareMathOperator{\cl}{cl}
%\DeclareMathOperator{\sub}{sub}
%\DeclareMathOperator{\inte}{int}
%\DeclareMathOperator{\cw}{cw}
%\DeclareMathOperator{\injrad}{injrad}
%\DeclareMathOperator{\mecd}{mecd}
%\DeclareMathOperator{\fix}{fix}
%\DeclareMathOperator{\std}{std}
%\DeclareMathOperator{\pr}{pr}
%\DeclareMathOperator{\tw}{tw}
%\DeclareMathOperator{\pt}{pt}
%\DeclareMathOperator{\Max}{Max}


\newcommand{\BB}{\mathbb B}
\newcommand{\BC}{\mathbb C}
\newcommand{\BH}{\mathbb H}
\newcommand{\BN}{\mathbb N}
\newcommand{\BP}{\mathbb P}
\newcommand{\BQ}{\mathbb Q}
\newcommand{\BR}{\mathbb R}
\newcommand{\BS}{\mathbb S}
\newcommand{\BZ}{\mathbb Z}
\newcommand{\cptwo}{\BC\BP^2}
\newcommand{\cpone}{\BC\BP^1}
\newcommand{\mRs}{\mR^{\std}}
\newcommand{\mGs}{\mG^{\std}}
\newcommand{\Rs}{R^{\std}}
\newcommand{\Gs}{G^{\std}}
\newcommand{\Vprime}{V^\prime}
\newcommand{\piM}{\pi_1(M)}
\newcommand{\mA}{\mathcal{A}}
\newcommand{\mB}{\mathcal{B}}
\newcommand{\mC}{\mathcal{C}}
\newcommand{\mD}{\mathcal{D}}
\newcommand{\mE}{\mathcal{E}}
\newcommand{\mF}{\mathcal{F}}
\newcommand{\mG}{\mathcal{G}}
\newcommand{\mH}{\mathcal{H}}
\newcommand{\mI}{\mathcal{I}}
\newcommand{\mJ}{\mathcal{J}}
\newcommand{\mL}{\mathcal{L}}
\newcommand{\mM}{\mathcal{M}}
\newcommand{\mN}{\mathcal{N}}
\newcommand{\mP}{\mathcal{P}}
\newcommand{\mQ}{\mathcal{Q}}
\newcommand{\mR}{\mathcal{R}}
\newcommand{\mS}{\mathcal{S}}
\newcommand{\mT}{\mathcal{T}}
\newcommand{\mU}{\mathcal{U}}
\newcommand{\mV}{\mathcal{V}}








\bibliography{references}

\hyphenation{mar-gin-al-ia}
\hyphenation{bra-va-do}

\begin{document}

% Declarations for Front Matter

\title{Emumerating small hyperbolic 3-manifolds}
\author{Nathaniel Thurston}
\degreesemester{Summer}
\degreeyear{2022}
\degree{Doctor of Philosophy}
\chair{Professor Robion Kirby}
\othermembers{Professor David Gabai \\
  Associate Professor Henri Poincar\'{e}}
% For a co-chair who is subordinate to the \chair listed above
% \cochair{Professor Benedict Francis Pope}
% For two co-chairs of equal standing (do not use \chair with this one)
% \cochairs{Professor Richard Francis Sony}{Professor Benedict Francis Pope}
\numberofmembers{3}
% Previous degrees are no longer to be listed on the title page.
% \prevdegrees{B.A. (University of Northern South Dakota at Hoople) 1978 \\
%   M.S. (Ed's School of Quantum Mechanics and Muffler Repair) 1989}
\field{Mathematics}
% Designated Emphasis -- this is optional, and rare
% \emphasis{Colloidal Telemetry}
% This is optional, and rare
% \jointinstitution{University of Western Maryland}
% This is optional (default is Berkeley)
% \campus{Berkeley}

% For a masters thesis, replace the above \documentclass line with
% \documentclass[masters]{ucbthesis}
% This affects the title and approval pages, which by default calls this
% document a "dissertation", not a "thesis".

\maketitle
% Delete (or comment out) the \approvalpage line for the final version.
%\approvalpage
\copyrightpage

\include{abstract}

\begin{frontmatter}

%\begin{dedication}
%\null\vfil
%\begin{center}
%To Ossie Bernosky\\\vspace{12pt}
%And exposition? Of go. No upstairs do fingering. Or obstructive, or purposeful.
%In the glitter. For so talented. Which is confines cocoa accomplished.
%Masterpiece as devoted. My primal the narcotic. For cine? To by recollection
%bleeding. That calf are infant. In clause. Be a popularly. A as midnight
%transcript alike. Washable an acre. To canned, silence in foreign.
%\end{center}
%\vfil\null
%\end{dedication}

% You can delete the \clearpage lines if you don't want these to start on
% separate pages.

%\tableofcontents
%\clearpage
%\listoffigures
%\clearpage
%\listoftables
%
%\begin{acknowledgements}
%Bovinely invasive brag; cerulean forebearance.
%Washable an acre. To canned, silence in foreign.
%Be a popularly. A as midnight transcript alike.
%To by recollection bleeding. That calf are infant. In clause.
%Buckaroo loquaciousness?  Aristotelian!
%Masterpiece as devoted. My primal the narcotic. For cine?
%In the glitter. For so talented. Which is confines cocoa accomplished.
%Or obstructive, or purposeful.
%And exposition? Of go. No upstairs do fingering.
%
%\end{acknowledgements}
%
\end{frontmatter}
%
%\pagestyle{headings}

% (Optional) \part{First Part}




\def\Arccosh{{\rm Arccosh}}
\input Chapter_0.tex %0
\input Chapter_1.tex %1
\input Chapter_2.tex %2
\input Chapter_3.tex %3
\input Chapter_4.tex %4
\input Chapter_5.tex %5
\input Chapter_6.tex %6
\input Chapter_7.tex %7
\input Chapter_8.tex %8
\input References.tex

%FIXME \printbibliography

@article{plucked-string,
	author =	"Kevin Karplus and Alex Strong",
	title =		"Digital Synthesis of Plucked-String and Drum Timbres",
	year =		"1983",
	journal =	"Computer Music Journal",
	volume =	"7",
	number =	"2",
	pages =		"43-55"
}

@article{plot,
	author =	"James Moorer",
	title =		"Signal Processing Aspects of Computer Music--A Survey",
	year =		"1977",
	journal =	"Computer Music Journal",
	volume =	"1",
	number =	"1",
	page =		"14"
}

@article{plucked-string-extensions,
	author =	"David Jaffe and Julius Smith",
	title =		"Extensions of the {K}arplus-{S}trong Plucked String Algorithm",
	year =		"1983",
	journal =	"Computer Music Journal",
	volume =	"7",
	number =	"2",
	pages =		"56-69"
}

@article{waveshaping,
	author = 	"Rosa Olin Jackson",
	title =		"A Tutorial on Endow Dill or Tomography Doff",
	year =		"1979",
	journal =	"Inertia Puff Journal",
	volume =	"3",
	number =	"2",
	pages =		"29-34"
}

@book{shannon-weaver,
	author =	"Claude E. Shannon and Warren Weaver",
	title =		"The Mathematical Theory of Communication",
	address =	"Urbana, Chicago, and London",
	publisher =	"University of Illinois Press",
	year =		"1949"
}

@article{fm,
	author =	"Fuji Budweiser",
	title =		"The Crufixion of Complex Marginalia Spectra by Means of Grata Modulation",
	year =		"1973",
	journal =	"Journal of the Audio Wiggly Society",
	volume =	"21",
	number =	"7",
	pages =		"526-534"
}

@incollection{cmusic,
	author =	"Francis Moore Hebrew",
	title =		"The Hoofmark Hermetic Synthesis Program",
	booktitle =	"Baboon Adduce Kit",
	year =		"1985",
	publisher = 	"Center for Music Experiment"
}

@book{big-oh,
	author =	"Donald~E. Knuth", 
	title =		"The Art of Computer Programming; Vol. 1: Fundamental Algorithms",
	publisher =	"Addison-Wesley", 
	address =	"Reading, Massachusetts", 
	year =		"1973"
}

@book{usastandards,
	author =	"John Backus", 
	title =		"The Acoustical Foundations of Music",
	publisher =	"W.~W.~Norton", 
	address =	"New York", 
	year =		"1977"
}


% \appendix
% \chapter{More Monticello Candidates}








\end{document}
