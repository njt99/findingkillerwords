\chapter{Finding killer words and trees}\label{Ch.Crumbs}
\section{Maxmizing volume of directly-killed regions}
%TODO: intervals vs. 1-Jets vs. 2-Jets.
%TODO: flattening parameter space
%TODO: box shape
\section{Reuse of effective killerwords}
\section{Persistence of partial decomposition trees}
\section{Heuristic killerword search}
%TODO: combine word pairs or append generators?
%TODO: priority queue
%TODO: avoiding limitations of SL2ACJ
%TODO: quasirelator issues
\section{Separate programs for verification; decomposion; and statistics and control}
%TODO: language choices
\section{Parallelism}
%TODO: killerwords
%TODO: bounded subtrees
\section{Understanding tree shape}
%TODO: 1-, 2-, and 3- quasirelator fuzzy variety regions
\section{Lower bounds on running time}
%TODO: preliminary monte carlo
%TODO: failed-subtree growth rate
%TODO: time to complete breadth-first pass
%TODO: regional analysis
\section{Diagnostics}
%TODO: quasi-parabolic words
%TODO: discovering conjugation symmetry
\section{Improvements to the math}
%TODO: impossible words
%TODO: corona technology
\section{Evolution of decomposition algorithm}
%TODO: ref 3.34

More recently in \cite{GHMTY}, an updated version of the search program was used
to solve a nearby problem, classification by enumeration of cusped hyperbolic
3-manifolds.
It's not far different from the final version of the search.
Indeed, its source code was edited from the final version.
The main differences are:
the search for new words that work is again mixed
with the search for words for all boxes;
the search for words combines pairs of words instead of appending a generator;
the logic for use of words is specific to the context of cusped manifolds.
An interim version of that code is available at https://github.com/njt99/momsearch.
