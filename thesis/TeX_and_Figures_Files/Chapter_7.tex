\chapter{Hints from behind the curtain}\label{Ch.Crumbs}
\section{Introduction}
The goal of this thesis is
to introduce new computational methods
and use them to prove Theorem \ref{GMT 1.28}.
As noted in \ref{GMT 5.3} to prove this theorem it sufficed
to produce a data set which is then shown
by relatively simple methods to satisfy the desired conclusion.
However, constructing said data set required an enormous effort,
both computationally and personally,
that required us to develop efficient tools,
organization and approaches to both
recognize that the project is tractable
and to carry it through completion.
The goal of this chapter is to briefly state and discuss some of these elements.
Some such as affine approximation, are central parts of the text,
others are just casually used while others are totally suppressed.

\section{Choosing The Computational Setup}
It's all about controlling the error,
i.e.  maximizing the impact of the boxes you kill.
Note that if the error is increased by $10\%$,
you need at least roughly $1.1^6$ as many boxes (about twice as many).
Moreover, if an error component compounds faster than linearly
when you compose isometries, the situation is much worse.
We briefly describe some technique to control error.
\begin{enumerate}
	\item
		We discovered that standard interval arithmetic
		has far greater error compared with $1$-jet arithmetic,
		so we went with that.
		We considered 2-jet arithmetic,
		however the effort to set the foundations
		seemed untenable,
		especially considering that the $1$-jet material
		comprises 30+ pages of this thesis.
	\item
		We exponentiated the parameter space
		to reduce the required arithmetic operations
		to +, -, *, /, and square root.
		In particular we eliminated the need to work
		with functions such as cosh and arccosh from the formulas.
	\item
		We worked with three complex parameters
		rather than six real parameters.
		Among other things when multiplying
		two complex numbers using $1$-jets
		the error does not grow exponentially under complex rotation.
	\item
		We worked with box shapes that were as round as possible:
		By coordinating box shape with cyclical subdivision,
		and interleaving real and imaginary parts,
		we arranged to make all boxes similar.
		
		In our early exploratory stage,
		we tried adaptively adjusting the shape based on
		the error corresponding to different parameters.
		We ran into trouble, possibly due to a feedback loop,
		and shelved the idea.
\end{enumerate}

\section{Getting Started}
One must confront the question of whether or not a computer
assisted project has an acceptable chance of success.
To address this we did a preliminary Monte Carlo probe:
choosing a few (later a few hundred) points in the parameter space,
finding a killerword for the point,
finding the largest box containing the point that was killed,
and finally measuring box volume.

By examining the sampled distribution of volume of these killed boxes,
it seemed likely that it had a long tail,
suggesting that any data-driven estimate of run time for a complete search
would likely be a lower bound.
We had no idea how optimistic that lower bound was!
Nonetheless, it took us a month to learn enough to get through that stage.

\section{Engineering - Mathematics Feedback}
The search was aided
by mathematical and experimental results
that were discovered by the search
as well as previously known ones.
Here are some examples.
\begin{enumerate}
	\item 
		We ran into early trouble in our search, with many thousands
		of problematic boxes present after
		several cycles of six subdivisions each, and the number growing
		rapidly as we progressed deeper.
		Investigation revealed a pattern:
		most of these problem boxes had strange quasi-relators
		like $fwf^{-1}w^{-1}$.
		
		Happily, a relator of this form
		implies that $f$ and $w$ share an axis
		which is impossible in our setup.

		After realizing this, it was easy to modify the search to
		take advantage of the insight,
		kill all the problem boxes,
		and proceed deeper into the box tree.
	\item Conjugation Symmetry: See \ref{GMT 5.2}.
		This was discovered by using Geomview and ndview
		to visualize six-dimensional wireframes of unsolved boxes.
		A bilateral symmetry in their arrangement caught our eye...
		
	\item Mathematical Software:
		Snappea, Heegaard and Poincare were essential tools.
		For example Snappea gave us confidence that a result along the lines we hoped for might be true.
		However, it wasn't perfect because the search discovered exceptional regions.
		But then Heegaard and Poincare experimentally showed
		that manifolds existed in our exceptional regions,
		giving confidence
		that we indeed found seven needles in
		our five hundred million box haystack.

	\item The Shimizu-Leutbecher Theorem:
		This theorem (not used in this thesis) is very useful
		for finding buffer regions about varieties
		defined by quasi-relators
		and was successfully used in our paper \cite{GHMTY}
		to eliminate regions. See Lemma 2.13 of that paper.

\end{enumerate}

\section{The Computational Trilogy}
We developed three separate programs to complete this
project.
\begin{enumerate}
	\item Verify:
		This clean and simple program written in CWEB
		rigorously checks that
		our parameter space subdivision with killer words functions.
	\item Decomposition:
		This program written in C++ finds the decomposition of
		parameter space into exceptional boxes and boxes with killer words.
		It was frequently altered as the search progressed,
		to take advantage of knowledge gained during the search.
		See remark \ref{GMT1.34} for details.
	\item Statistics and Control:
		This collection of small programs written in Perl (later Python)
		does everything else,
		from estimating a lower bound on remaining search effort
		to orchestrating multi-host parallelism
		to validating tree structure to rebalancing the tree.
\end{enumerate}
