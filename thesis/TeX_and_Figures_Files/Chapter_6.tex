\chapter{AffApproxes with round-off error}\label{Ch.AffWithEPS} % =~ GMT \S 8
 
In Chapter \ref{Ch.AffApprox}, we saw how to do calculations with AffApproxes.  Here, we incorporate round-off error into
these calculations.  

\begin{conventions}\label{GMT 8.1}
An AffApprox $x$ is a five-tuple
$(x.f; x.f_0,x.f_1,x.f_2; x.{\textrm err})$
consisting of four complex numbers $(x.f, x.f_0,x.f_1, x.f_2)$ and one real number $x.{\textrm err}.$  In
	Chapter \ref{Ch.AffApprox}
, the
real number was denoted $x.e,$ but it seems preferable to use $x.{\textrm err}$ here.  Recall (Definition \ref{GMT 6.4}
that an AffApprox $x$ represents the set $S(x)$ of functions from $A = \{(z_0,z_1,z_2) \in \mathbf {C}^3 : |z_k| \le 1 \
{\textrm for} \ k \in \{0,1,2\}\}$ to $\mathbf {C}$ that are $x.{\textrm err}$-well-approximated by the affine function
$x.f + x.f_0 z_0 + x.f_1 z_1 + x.f_2 z_2.$
\end{conventions}

\begin{remarks} \label{GMT 8.2}
A review of Definition \ref{GMT def7.2} (XComplexes and AComplexes; loosely, exact and approximate complex numbers) might be helpful at this time.

One approach to round-off error for AffApproxes would be to replace the four complex numbers in the definition of AffApprox by four AComplex numbers complete with their round-off errors, and similarly for the one real number.  We will not do this because it would necessitate keeping track of five separate round-off-error terms when we do AffApprox calculations.

Instead, we will 
replace the four complex numbers by four XComplexes and push all the round-off error into the $.{\textrm err}$ term.  
Thus, the definition of AffApprox remains 
essentially the same as in
	\ref{Ch.AffApprox}
We note that, in doing an AffApprox calculation, our subsidiary calculations will generally be on XComplex numbers and produce an AComplex number whose $.e$ term will be plucked off and forced into the $.{\textrm err}$ term of the final AffApprox.\end{remarks}

\begin{conventions}\label{GMT 8.3}
i)  In what follows, we will use Basic Properties \ref{GMT basic7.0} and Lemmas \ref{GMT 7.0} and \ref{GMT lemma7.1}.  Also, the corresponding propositions in
	Chapter \ref{Ch.AffApprox}
will be utilized.
(for example, Proposition \ref{GMT prop6.7} corresponds to Proposition \ref{GMT prop8.7}.  
\vglue4pt
	ii)  Some notational simplifications will be introduced: ${\textrm dist}(x)$ before Proposition \ref{GMT prop8.6}, $ax$ before Propositions \ref{GMT prop8.8}, \ref{GMT prop8.9}, \ref{GMT prop8.11a}, 
	(the middle usage of $ax$ differs slightly from the other two), and $ay$ before Propositions \ref{GMT prop8.8}, \ref{GMT prop8.9}.
\vglue4pt
	iii)  We will try to keep our notation fairly consistent with that of the {\textit verify} computer program, and this will produce some mildly peculiar notation.  In particular, in the operations pertaining to Propositions \ref{GMT prop8.2} and beyond, the resultant AffApproxes will be denoted $(r\_f.z; r\_f_1.z, r\_f_2.z, 
r\_f_3.z; r\_{\textrm error})$
where the first four terms are XComplexes and the last term is a double (technically, $r\_f.z$ is the XComplex part of the AComplex  $r\_f$ and similarly for the $r\_f_k.z$ terms).  One break with the notation of the programs though is that AffApproxes are called (in the programs) ACJ's, which stands for ``Approximate Complex 1-Jets."
\vglue4pt
iv) The propositions that follow include in their statements the definitions of the various operations on AffApproxes (see Remark
	\ref{GMT 6.6} iii).
\vglue4pt
	v)  We remind the reader
	(see Conventions \ref{GMT cs7.4}
	that all machine operations act on machine numbers, and that the 
various variables appearing in the propositions are assumed to be doubles.
\end{conventions}

$-X$:

\begin{proposition}\label{GMT prop8.1}
If $x$ is an AffApprox 
then $S(-x) = -(S(x))$ where
$$-x \equiv (-x.f; -x.f_0,-x.f_1,-x.f_2; x.{\textrm err}).$$ \end{proposition}


$X+Y$:
We analyze the addition of the  {\textrm AffApproxes} 
 $x = (x.f; x.f_0,x.f_1, x.f_2; x.{\textrm err})$  and 
$y = (y.f; y.f_0,y.f_1,y.f_2; y.{\textrm err}).$  To get the first term in $x+y$ we add the XComplex numbers $x.f$ and $y.f;$ which
produce the AComplex number $r\_f = x.f + y.f$ (see Proposition \ref{GMT prop7.4}), and then we pluck off the XComplex part,
which we denote $r\_f.z.$  The round-off error part $r\_f.e$ will be foisted into the overall error term $r\_{\textrm error}$
for $x+y.$ Similarly for the next three terms in $x + y.$

Abstractly, the overall error term $r\_{\textrm error}$ comes from adding the round-off error contributions $r\_f.e,\  r\_f_0.e,\  r\_f_1.e, 
\ r\_f_2.e$ and the AffApprox error contributions $x.{\textrm err},\  y.{\textrm err}$.  Of course, we have to produce a machine version.

\begin{proposition}\label{GMT prop8.2} If $x$ and $y$ are 
{\textrm AffApproxes, }
then $S(x + y) \supseteq S(x) + S(y)${\textrm ,} where
$$x + y \equiv (r\_f.z; r\_f_0.z, r\_f_1.z, r\_f_2.z; r\_{\textrm error})$$
with
\begin{eqnarray*}
r\_f &=& x.f + y.f,\\
r\_f_k &=& x.f_k + y.f_k,\\
r\_{\textrm error}& =& (1 + 3EPS)\\
&&\otimes\ ((x.{\textrm err} \oplus y.{\textrm err}) \oplus ((r\_f.e \oplus
r\_f_0.e) \oplus (r\_f_1.e \oplus r\_f_2.e))).
\end{eqnarray*}
\end{proposition}

\begin{proof}{}
The error is given by 
\begin{eqnarray*}
&&\hskip-24pt (x.{\textrm err} + y.{\textrm err})  +\ ((r\_f.e + r\_f_0.e) + (r\_f_1.e + r\_f_2.e))\\
&\le& (1+EPS/2) (x.{\textrm err} \oplus
y.{\textrm err}) \\
&&+\ (1+EPS/2)((r\_f.e \oplus r\_f_0.e) + (r\_f_1.e \oplus r\_f_2.e))\\
&\le& (1+EPS/2)^3 ((x.{\textrm err} \oplus
y.{\textrm err}) \oplus ((r\_f.e \oplus r\_f_0.e) \oplus (r\_f_1.e \oplus r\_f_2.e)))\\
&\le& (1 + 3EPS) \otimes ((x.\mathrm {err} \oplus y.{\textrm err}) \oplus ((r\_f.e \oplus r\_f_0.e) \oplus (r\_f_1.e \oplus r\_f_2.e))).
\end{eqnarray*}
 To get the last line
we used Lemma \ref{GMT lemma7.1}. \end{proof}

$X - Y$:

\begin{proposition}\label{GMT prop8.3} If $x$ and $y$ are {\textrm AffApproxes,} then $S(x - y) \supseteq S(x) - S(y)${\textrm ,}
 where
$$x - y \equiv (r\_f.z; r\_f_0.z, r\_f_1.z, r\_f_2.z; r\_{\textrm error})$$
with
\begin{eqnarray*}
r\_f& =& x.f - y.f,\\
r\_f_k &= &x.f_k - y.f_k,\\
r\_{\textrm error} &=& (1 + 3EPS)\\
&& \otimes\ ((x.{\textrm err} \oplus y.{\textrm err}) \oplus ((r\_f.e
\oplus r\_f_0.e) \oplus (r\_f_1.e \oplus r\_f_2.e))).
\end{eqnarray*}
\end{proposition}

$X + D$: 
Here, we add the AffApprox $x = (x.f; x.f_0,x.f_1,x.f_2; x.{\textrm err})$  to the double $y$.  The only terms that change are the first and the last.

\begin{proposition}\label{GMT prop8.4}\hskip-5pt If $x$ is an 
{\textrm AffApprox}  and $y$ is a double{\textrm ,} 
then $S(x + y) \supseteq S(x) + S(y)${\textrm ,} where
$$x + y \equiv (r\_f.z; r\_f_0.z, r\_f_1.z, r\_f_2.z; r\_{\textrm error})$$
with
\begin{eqnarray*}
r\_f& =& x.f + y,\\
r\_f_k &=& x.f_k,\\
r\_{\textrm error}& = &(1 + EPS) \otimes (x.{\textrm err} \oplus r\_f.e).\end{eqnarray*}
\end{proposition}

\begin{proof}{}
The error is given by 
$$x.{\textrm err} + r\_f.e \le (1 + EPS) \otimes (x.{\textrm err} \oplus r\_f.e)$$
by Lemma \ref{GMT 7.0}. \end{proof}

$X - D$:
\begin{proposition}\label{GMT prop8.5}\hskip-5pt If $x$ is an 
{\textrm AffApprox}  and $y$ is a double{\textrm ,}
then $S(x - y) \supseteq S(x) - S(y)${\textrm ,} where
$$x - y \equiv (r\_f.z; r\_f_0.z, r\_f_1.z, r\_f_2.z; r\_{\textrm error})$$
with
\begin{eqnarray*}
r\_f &= &x.f - y,\\
r\_f_k &=& x.f_k,\\
r\_{\textrm error}& =& (1 + EPS) \otimes (x.{\textrm err} \oplus r\_f.e).\end{eqnarray*}
\end{proposition}

$X \times Y$: 
We multiply the AffApproxes $x$ and $y$ while pushing all error into the $.{\textrm err}$ term.

We will use the functions (see Formulas \ref{GMT formula7.0} and \ref{GMT formula7.1}, at the end of Chapter \ref{Ch.XComplex}) ${\textrm absUB}= (1 + 2EPS) \otimes
{\textrm hypot}_o(x.re,x.im)$ and 
${\textrm absLB}(x) = (1 - 2EPS) \otimes {\textrm hypot}_o(x.re,x.im)$.

When $x$ is an AffApprox, we define ${\textrm dist}(x)$ to be $$(1 + 2EPS) \otimes ({\textrm absUB}(x.f_0) \oplus ({\textrm absUB}(x.f_1) \oplus {\textrm absUB}(x.f_2))).$$  This is the machine representation of the sum of the absolute values of the linear terms in the AffApprox $x$ 
(the proof is straightforward).

\begin{proposition}\label{GMT prop8.6} If $x$ and $y$ are 
{\textrm AffApproxes,}
then $S(x \times y) \supseteq S(x) \times S(y)${\textrm ,} where
$$x \times y \equiv (r\_f.z; r\_f_0.z, r\_f_1.z, r\_f_2.z; r\_{\textrm error})$$
with
\begin{eqnarray*}
r\_f &=& x.f \times y.f,\\
r\_f_k &=& x.f \times y.f_k + x.f_k \times y.f,\\
r\_{\textrm error} &=& 
(1 + 3EPS) \otimes (A \oplus (B \oplus C)),\end{eqnarray*}
and
\begin{eqnarray*}
A &=& ({\textrm dist}(x) \oplus x.{\textrm err})\otimes ({\textrm dist}(y) \oplus y.{\textrm err}),
\\
B& =& {\textrm absUB}(x.f) \otimes
y.{\textrm err} \oplus {\textrm absUB}(y.f) \otimes x.{\textrm err},\\
C& =& (r\_f.e \oplus r\_f_0.e) \oplus (r\_f_1.e \oplus
r\_f_2.e).\end{eqnarray*}
\end{proposition}

\begin{proof}{}
We add the non-round-off error term for $x \times y$ to the various round-off error terms that accumulated.
\begin{eqnarray*}
&&\hskip-48pt (({\textrm dist}(x) + x.{\textrm err}) \times\ ({\textrm dist}(y) + y.{\textrm err})) +
 (({\textrm absUB}(x.f) \times y.{\textrm err} 
\\
&& +\ 
{\textrm absUB}(y.f) \times x.{\textrm err}) +
 (r\_f.e + r\_f_0.e) + (r\_f_1.e + r\_f_2.e))
\\
&\le& 
(1+EPS/2)^3[({\textrm dist}(x) \oplus x.{\textrm err})\otimes ({\textrm dist}(y) \oplus y.{\textrm err})]\\
&& +\
 (1+EPS/2)^2\{({\textrm absUB}(x.f) \otimes y.{\textrm err} 
\\
& &\oplus\ 
{\textrm absUB}(y.f) \otimes x.{\textrm err}) +
((r\_f.e \oplus r\_f_0.e) \oplus (r\_f_1.e \oplus r\_f_2.e))\}
\\
&\le& (1+EPS/2)^3 A + (1+EPS/2)^3 (B \oplus C) \\
&\le& (1+3EPS) \otimes (A \oplus (B \oplus C) ). \\
\noalign{\vskip-36pt}
\end{eqnarray*}
\end{proof}

$X \times D$: 

\begin{proposition}\label{GMT prop8.7} \hskip-4pt If $x$ is an 
AffApprox  and $y$ is a double{\textrm ,} 
then $S(x \times y) \supseteq S(x) \times S(y)${\textrm ,} where
$$x \times y = (r\_f.z; r\_f_0.z, r\_f_1.z, r\_f_2.z; r\_{\textrm error})$$
with
\begin{eqnarray*}
r\_f& =& x.f \times y,\\
r\_f_k &=& x.f_k \times y,\\
r\_{\textrm error} &=& (1+3EPS)\\
&& \otimes\ ( (x.{\textrm err} \otimes |y|) \oplus
( (r\_f.e \oplus r\_f_0.e) \oplus (r\_f_1.e \oplus r\_f_2.e)) ).
\end{eqnarray*}
\end{proposition}

$X/Y$: 
For convenience, let  $ax = {\textrm absUB}(x.f),\ ay = {\textrm absLB}(y.f).$ 
 
\begin{proposition}\label{GMT prop8.8} If $x$ and $y$ are 
{\textrm AffApproxes}  with $D > 0$ {\textrm (}\/see below{\textrm ),} then
$S(x / y) \supseteq S(x) / S(y)${\textrm ,} where
$$x / y \equiv (r\_f.z; r\_f_0.z, r\_f_1.z, r\_f_2.z; r\_{\textrm error})$$
with
\begin{eqnarray*}
r\_f &=& x.f / y.f,
\\
r\_f_k& =& (x.f_k \times y.f - x.f \times y.f_k)/ (y.f \times y.f),\\
r.{\textrm error} & =& (1 + 3EPS) \otimes (
((1 + 3EPS) \otimes A \ominus (1 - 3EPS) \otimes B )\oplus C ),\end{eqnarray*}
and
\begin{eqnarray*}
A &=& (ax \oplus ({\textrm dist}(x) \oplus x.{\textrm err})) \oslash D,\\
B &=& (ax \oslash ay \oplus {\textrm dist}(x) \oslash ay)
\oplus  (({\textrm dist}(y) \otimes ax) \oslash (ay \otimes ay)),\\
C &=& (r\_f.e \oplus r\_f_0.e) \oplus (r\_f_1.e \oplus
r\_f_2.e),\\
D &=& ay \ominus (1 + EPS) \otimes ({\textrm dist}(y) \oplus y.{\textrm err}).
\end{eqnarray*}
\end{proposition}

\begin{proof}{}
As usual, we add the round-off errors to the old AffApprox error, 
taking into account round-off error, working on it bit by bit.
\begin{eqnarray*}
 &&\hskip-16pt (ax + {\textrm dist}(x) + x.{\textrm err})/(ay - ({\textrm dist}(y) + y.{\textrm err}))\\
&&  \le (1 + EPS/2)^2 \\
&&\quad \times \ (ax \oplus ({\textrm dist}(x)
\oplus x.{\textrm err}))/
         (ay - (1 + EPS) \otimes ({\textrm dist}(y) \oplus y.{\textrm err}))\\
&&  \le (1 + EPS/2)^2 (ax \oplus ({\textrm dist}(x) \oplus
x.{\textrm err}))/
         \Big(\left({1 \over 1 + EPS/2}\right) \\
&&\quad\times \ (ay \ominus (1 + EPS) \otimes ({\textrm dist}(y) \oplus y.{\textrm err}))\Big)\\
&& \le (1 + EPS/2)^4 (ax
\oplus ({\textrm dist}(x) \oplus x.{\textrm err}))\\
&&\quad \oslash
         (ay \ominus (1 + EPS) \otimes ({\textrm dist}(y) \oplus y.{\textrm err}))\\
&& \le   (1 + 3EPS) \otimes A.
\end{eqnarray*}

The next term, being subtracted, requires opposite inequalities.
\begin{eqnarray*}
&&(ax/ay + {\textrm dist}(x)/ay) + {\textrm dist}(y) \times ax / (ay \times ay)\\
&&\quad \ge (1-EPS/2)(ax \oslash ay + \mathrm {dist}(x)
\oslash ay) \\
&&\qquad +  (1-EPS/2) ({\textrm dist}(y) \otimes ax )/ ({1 \over 1 - EPS/2})(ay \otimes ay)\\
&&\quad \ge ((1-EPS/2)^4 ((ax
\oslash ay \oplus {\textrm dist}(x) \oslash ay) \oplus  (({\textrm dist}(y) \otimes ax) \oslash (ay \otimes ay)))\\
&&\quad \ge (1 +
EPS/2)(1 - 3EPS) (B)  \ge (1 - 3EPS) \otimes B.\end{eqnarray*}

Finally, we do the round-off terms 
$$((r\_f.e + r\_f_0.e) + (r\_f_1.e + r\_f_2.e)) \le (1 + EPS/2)^2 C $$
and we put these three pieces together:
\begin{eqnarray*}
&&
(ax + {\textrm dist}(x) + x.{\textrm err})/(ay - ({\textrm dist}(y) + y.{\textrm err}))
\\
&&\qquad  - 
((ax/ay + {\textrm dist}(x)/ay) + {\textrm dist}(y) \times ax / (ay \times ay))\\
&&\qquad
+ ((r\_f.e + r\_f_0.e) + (r\_f_1.e + r\_f_2.e))
\\
&&\quad \le 
(1 + 3EPS) \otimes A - (1 - 3EPS) \otimes B + (1 + EPS/2)^2 C\\
&&\quad \le (1+EPS/2)^2 (
((1 + 3EPS) \otimes A \ominus (1 - 3EPS) \otimes B)
 + C
) \\
&&\quad \le (1+3EPS) \otimes (
((1 + 3EPS) \otimes A \ominus (1 - 3EPS) \otimes B)
 \oplus C
) . \\
\noalign{\vskip-36pt}
\end{eqnarray*}
\end{proof}

$D/X$:
We   divide a double $x$ by an AffApprox $y$.
For convenience, let $ax = |x|,\ ay = {\textrm absLB}(y.f).\ $  Having done division out in the previous proposition, we will skip the proof of Proposition \ref{GMT prop8.9}.  See the {\textit Annals}
web site for the proof.

\begin{proposition}\label{GMT prop8.9} If $x$ is a double and $y$ is an 
{\textrm AffApprox}  with $D > 0$ {\textrm (}\/see below{\textrm ),}
 then $S(x / y) \supseteq S(x) / S(y)${\textrm ,} where
$$x / y \equiv (r\_f.z; r\_f_0.z, r\_f_1.z, r\_f_2.z; r\_{\textrm error})$$
with
 \begin{eqnarray*}
r\_f&\hskip-8pt =\hskip-8pt& x / y.f,\\
r\_f_k&\hskip-8pt =\hskip-8pt& -(x \times y.f_k)/ (y.f \times y.f),\\
r\_{\textrm error}&\hskip-8pt =\hskip-8pt& (1+3EPS) \otimes ( 
((1 + 2EPS) \otimes (ax  \oslash D) \ominus (1 - 3EPS) \otimes B)
\oplus C),\\
 B&\hskip-8pt =\hskip-8pt& ax \oslash ay  \oplus ({\textrm dist}(y) \otimes ax \oslash (ay \otimes ay)),\\
C& \hskip-8pt=\hskip-8pt& (r\_f.e \oplus r\_f_0.e)
\oplus (r\_f_1.e \oplus r\_f_2.e),\\
D &\hskip-8pt=\hskip-8pt& ay \ominus (1 + EPS) \otimes ({\textrm dist}(y) \oplus y.{\textrm err}).
\end{eqnarray*}
\end{proposition}

$X/D$:
We   divide an AffApprox $x$ by a nonzero double $y$ (the computer will object if $y = 0$).  The proof is easy  and
so we delete it.

\begin{proposition}\label{GMT prop8.10} 
If $x$ is an {\textrm AffApprox}  and $y$ is a nonzero double{\textrm ,} then
$S(x / y) \supseteq S(x) / S(y)${\textrm ,} where
$$x / y \equiv (r\_f.z; r\_f_0.z, r\_f_1.z, r\_f_2.z; r\_{\textrm error}),$$
with
\begin{eqnarray*}
r\_f &\hskip-8pt=\hskip-8pt& x.f / y\\
r\_f_k &\hskip-8pt=\hskip-8pt& x.f_k / y\\
r\_{\textrm error} &\hskip-8pt = \hskip-8pt&
(1+3EPS) \otimes ( 
(x.{\textrm err}\oslash |y|)
\oplus 
[(r\_f.e \oplus r\_f_0.e) \oplus (r\_f_1.e \oplus r\_f_2.e)]
                                               ).\end{eqnarray*}
\end{proposition}

$\sqrt X$: 
Here, $x$ is an AffApprox and we let $ax = {\textrm absUB}(x.f)$.  There are two cases to consider depending on whether or not
$D = ax \ominus (1 + EPS) \otimes ({\textrm dist}(x) \oplus x.{\textrm err})$ is or is not greater than zero.

\begin{proposition}\label{GMT prop8.11a} If $x$ is an {\textrm AffApprox}
and
$ D = ax \ominus (1 + EPS) \otimes ({\textrm dist}(x) \oplus x.{\textrm err})$ is not greater than zero{\textrm ,} then 
$S(\sqrt x) \supseteq \sqrt {S(x)}${\textrm ,} where
we use the crude overestimate $$\sqrt x \equiv \left(0;0,0,0;(1 + 2EPS) \otimes
\root o \of {(ax \oplus (x\,{\textrm dist} \oplus x.{\textrm err}))} \right).$$
\end{proposition}

\begin{proof}{}
\begin{eqnarray*}\sqrt {ax + x\,{\textrm dist} + x.{\textrm err}}&\le &(1 + EPS/2) \sqrt { (ax \oplus (x\,{\textrm dist} \oplus x.{\textrm err}))}\\
&\le  &(1 + 2EPS) \otimes \root o \of { (ax \oplus (x\,{\textrm dist} \oplus x.{\textrm err}))}.
\\
\noalign{\vskip-36pt}
\end{eqnarray*}
\end{proof}

\begin{proposition}\label{GMT prop8.11b} If 
$x$  is an {\textrm AffApprox} and
$ D = ax \ominus (1 + EPS) \otimes ({\textrm dist}(x) \oplus x.{\textrm err})$ is greater than zero{\textrm ,}
then $S(\sqrt x) \supseteq \sqrt {S(x)}${\textrm ,} where
 $$\sqrt{x} \equiv (r\_f.z; r\_f_0.z,r\_f_1.z,r\_f_2.z; r\_{\textrm error})$$ 
with
\begin{eqnarray*}
r\_f& = &\sqrt {x.f},\\
t &=& r\_f + r\_f,\\
r\_f_k& = &{\textrm AComplex} (x.f_k.re,x.f_k.im;0) / t.
\end{eqnarray*}
{\textrm (}\/Simply put{\textrm ,} $r\_f_k = x.f_k / (2\sqrt{x.f})$. The reason we have to fuss to define $r\_f_k$ is
 because
$\sqrt{x.f}$ is an {\textrm AComplex.)}
\begin{eqnarray*}
  r\_{\textrm error}& \hskip-8pt =\hskip-8pt &
(1+3EPS)\\ &\hskip-8pt\hskip-8pt& \otimes\hskip-1pt \left\{\hskip-1pt
(1+EPS) \otimes \root o \of {ax}\ominus
(1 - 3EPS) \otimes [{\textrm dist}(x)\oslash(2 \times \root o \of {ax})
\oplus \root o \of {D} ]\right\}
\\
&\hskip-8pt\hskip-8pt&  \oplus  
((r\_f.e \oplus r\_f_0.e) \oplus (r\_f_1.e \oplus r\_f_2.e))
.\end{eqnarray*}
\end{proposition}

\begin{proof}{}
Let us work on the pieces.
$$\sqrt {ax} \le (1 + EPS) \otimes \root o \of {ax}.$$

Next,
\begin{eqnarray*}
&&{\textrm dist}(x)/(2 \sqrt {ax}) + \sqrt {ax - ({\textrm dist}(x) + x.{\textrm err})}\\
&&\quad 
\ge  (1 - EPS/2)^2 {\textrm dist}(x) \oslash (2 \root o \of {ax}) 
\\
&&\qquad + (1 - EPS/2)^{1/2}\sqrt {ax \ominus (1+EPS) \otimes ({\textrm dist}(x) \oplus x.{\textrm err})}
                                              \\
&&\quad \ge(1 - EPS/2)^3 \left[{\textrm dist}(x) \oslash (2 \root o \of {ax}) \oplus \root o \of {D}\
\right]\\
&&\quad \ge (1 + EPS/2) (1 - 3EPS) \left[{\textrm dist}(x) \oslash (2 \root o \of {ax}) \oplus \root o \of {D}\ \right]\\
&&\quad \ge (1 -
3EPS) \otimes \left[{\textrm dist}(x) \oslash (2 \root o \of {ax}) \oplus \root o \of {D}\ \right].
\end{eqnarray*}

Adding in the usual term, we get as our error bound
\begin{eqnarray*}
&&\sqrt {ax} - ({\textrm dist}(x)/(2 \sqrt {ax}) + \sqrt {ax - ({\textrm dist}(x) + x.{\textrm err})}\ )\\
&&\qquad+\ ((r\_f.e + r\_f_0.e) + (r\_f_1.e + r\_f_2.e))\\
&&\quad \le (1 + EPS) \otimes \root o \of {ax} 
- (1 - 3EPS) \otimes [{\textrm dist}(x) \oslash (2 \root o \of {ax}) \oplus 
\root o \of {D}\ ]
\\
&&\qquad+\ (1 + EPS/2)^2 ((r\_f.e \oplus r\_f_0.e) \oplus (r\_f_1.e \oplus r\_f_2.e))
      \\
&&\quad \le (1 + EPS/2)^3 \big(\{(1 + EPS) \otimes \root o \of {ax} \\
&&\qquad 
\ominus\ (1 - 3EPS) \otimes [{\textrm dist}(x) \oslash (2 \root o \of {ax}) \oplus \root o \of {D}\ ]\}
\\
&&\qquad \oplus\ ((r\_f.e \oplus r\_f_0.e) \oplus (r\_f_1.e \oplus r\_f_2.e))\big)
 \\
&&\quad \le (1 + 3EPS) \otimes (
\{(1 + EPS) \otimes \root o \of {ax} 
\\
&&\qquad\ominus\ (1 - 3EPS) \otimes [{\textrm dist}(x) \oslash (2 \root o \of {ax})\oplus \root o \of {D}\ ]\}
\\
&&\qquad \oplus ((r\_f.e \oplus r\_f_0.e) \oplus (r\_f_1.e \oplus r\_f_2.e))
                                                ).\\
\noalign{\vskip-36pt}
\end{eqnarray*}
\end{proof}



